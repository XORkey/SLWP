\section{Scalability}
There are several degrees of scalability with this login scheme.
\subsection{Keyring}
Instead of a nice and round 100,
you can put any number of keys in a keyring file.
The value of 100 is just convenient;
all keys are numbered with a two-digit number,
which can easily be remembered.
\par
Any number will do,
from zero
(which leaves only the keyring identifier,
and allows you to login to zero websites)
up to any amount of keys you bother to carry around with you.
Having more keys than sites you want to login to is just a way to make things a bit harder for those that do not own the keyring to choose keys.
If you are not happy with this,
you can just add new keys as you acquire them,
just as with real keys.
(Nothing stops you from adding bogus brass and steel keys to the keyring that holds your carkeys.
But I don't believe it is common practice to add BMW and Ford keys to a keyring of a Toyota owner.)
\subsection{Key length}
The length of keys is generally considered as an important aspect.
The more bits, the better the key.
\par
That is true if such a key is used to encrypt data that is in any way predictable,
like, for example, a piece of text.
If the encrypted data has patterns of any kind,
you can directly work with intermediate decryption attempts.
Statistical analysis of resulting bit patterns can reveil if a certain key or method is getting close or closer.
\par
But all values encrypted with our keys are comprised of random bits only.
This implies that any result of decryption has to be tried,
to establish if the decryption was in any way successful.
That would mean many login attempts
(millions, billions, or more)
which is infeasable.
And that is just to crack a user key,
which only gives you one login.
\par
With a Secret key fixed on 256 bits to accommodate the \AES\ encryption,
all other keys can be a lot smaller than that.
Theoretically,
a 1-bit key could suffice,
but this obviously is not strong enough,
as it would take only two attempts to test all possible values.
\par
To rule out any feasable brute force attempts
(supposing that a site does not stop countless consecutive failed attempts)
a key length of 24 bits should be enough.
Assuming that a single login attempt,
exhausting all secret keys each time,
could be done in a second
(taking into account the network traffic to download and upload webpages and values),
a 16 bit key would be cracked in at most 18 hours.
365 days have 8760 hours,
and if Secret keys are replaced every half year,
then,
if an attempt takes more than 4380 hours,
it must be considered futile.
Doubling the time with each bit,
a 24 bit key would take 4608 hours of continuous effort,
which is well over half a year.
\subsection{Old Secret keys}
At least one old Secret key is required,
if Secret keys are to be replaced every now and then.
An HSM can store many keys,
so there is no real practical limit.
\subsection{Encryption algorithms}
In this article,
the basis for generating the keypairs for the site and the user is \AES.
All site keys and user keys are related using this algorithm,
so it has to be secure.
The encryption algorithm has to be a block cypher,
using a secret key.
A hashing function does not do the trick,
as this would directly reveil user keys once you have the site keys.
But any other block cypher may be employed for this task.
\par
Hashes are computed using \SHA\ in this article.
The hash values that are exchanged between website and user are incomplete by design,
and just contain some of the least significant bits of the hash.
The keylength of the site and user keys
should not exceed the lenght of the hash value,
nor have equal length,
as that would weaken the login algorithm.
Using 128-bit keys, a hash function like \MDV\ is not recommended, as this produces a 128-bit hash.
\par
\clearpage
