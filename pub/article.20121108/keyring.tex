\section{Storing user keys in a keyring}
\label{sec:keyring}
Keys for the user are collected and stored in a keyring.
The keyring is a block of at least 100 random numbers with 128 bits, most of them dummy keys.

\subsection{Creating a keyring}
A keyring is created by using a random generator of sufficient quality,
which generates 100 random numbers of 128 bits.
These random numbers are then written to a file on a storage device.
Key number zero is used to identify the keyring and will never change after its creation.
The other 99 keys are dummies (random bits with no meaning whatsoever).
\par
Nothing else is stored in a keyring, as to minimize the information stored there.
Since all bits are entirely random%
---for dummies and real keys---%
a keyring stores no information whatsoever.

\subsection{Encrypting a keyring}
Optionally but preferably,
the keyring on the storage device can be encrypted, to further enhance security.
The unencrypted values should never be written to a storage device and only be kept in volatile memory.
An encrypted keyring is of no value to a hacker without knowledge with which keys to decrypt it.
It is up to the user to remember this.
\par
The keyring can be encrypted by using the keys in the keyring itself.
The user may select any number of different random key numbers from its ring,
which are numbered from 1 to 99.
Suppose two keys are used.
Then,
prepending a zero for key numbers less than 10,
this yields a 4 digit PIN code for the keyring.
\par
Using the keys (in selected order) to encrypt the keyring, the values of the keys are obscured.
Select the first key from the keyring by specifying its index $j$.
For each key $K[i]$ (with $i$ running from 0 to 99), except $K[j]$, use XOR:
\[K[i]=\XOR{K[i]}{K[j]}\]
\par
To decrypt, reverse the order of the selected keys and do the same.
This will render the original keyring.

\subsection{Copying keys and keyrings}
A keyring can be freely copied to other devices, so all keys are available there as well.
The keys of one keyring can be copied to other keyrings, in different locations.
It is up to the user to keep a registration of this.
\par
Encrypted keyrings can be stored in a public place, for easy access, and act as a backup.
A dedicated webserver can be employed for storing encrypted keyrings,
which can be used to restore lost keyrings.
They can also be used to login when away from home with no access to your own keyring.
These temporary keyrings should be discarded when logging out.

\subsection{Irretrievably lost keys and keyrings}
When a keynumber is forgotten and no record of it can be found,
the key must be considered lost.
\par
As the user has supplied websites with personal information,
it may be easy to regain login capabilities by just asking for a new key,
provided a username and maybe some other information can be given.
Selecting a free keynumber on the keyring and replacing that key with the new one should restore the ability to login.
\par
Keyrings can become unusable or lost in two situations:
the device holding it is defective (like a harddisk crash) and no backup has been made,
or the encryption cannot be reversed (PIN forgotten).
In both cases, the user has to start over and create a new keyring, containing only dummy keys.
