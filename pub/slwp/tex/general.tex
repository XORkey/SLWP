%\section{Login for dummies}
%Not that you are one if you decide to read this section.
\Section{General description}
It serves as a general description of the login process,
without going into technical details too much.

\Subsection{A new login scheme}
Userids tend to be in short supply for people with first names like John, Peter, and Chris,
or surnames like Jones or Smith.
This is true for people in almost any country.
And the larger the site, the rarer available userids.
\par
Passwords tend to be weak,
as people choose short, real-life words, like things or names.
Not that people are not willing, but remembering passwords that are too long or too difficult is not easy.
It is very frustrating when you cannot login because you have forgotten your password,
or,
just when you are starting to remember it,
you need to change it again.
\par
As a radical measure, we do away with all this!
\par
Instead of userids, the website holds hashes based on key ring identifiers.
Instead of a password for an account, the website holds one part of a key, of which you have the other part,
like the broken locket Annie clings to in the musical that bears her name.
Your key is different from the key the website has, but they are related to each other.
It is not possible to login with the key that is stored on the website;
you can only login with your personal key, which is verified with the key from the website.
\par
We will let the computers
(yours, the one running the website, and another one dedicated for logins)
do what they do best: compute.
Give them some random bits to chew on, and they are in heaven.
We will let humans (you, your neighbor and your fellow earthlings) do what they do best:
remembering small ordinal numbers.

\Subsection{Storing your keys}
Keys are just long strings of random bits with no information whatsoever.
Your keys are personal and are stored locally.
\par
Imagine a key ring,
	not unlike your own key ring on which you have keys for your car,
		your house,
			shed,
				or locker.
This key ring has a label and 99 keys on it,
	numbered 1 through 99.
Instead of brass or steel,
	these keys are made of 128 random bits each.
The label is another 128 bits long,
	and as random as possible,
		like the keys.
Instead of being on a steel ring,
	these keys,
		with the key ring label,
			are written to a file on disk;
				a blob of 100 strings of 128 bits.
\par
The use of keys on this key ring is not all that different from using real physical keys.
The bits in a key are comparable with the teeth or holes of a physical key you use to unlock your home.
You do not need to remember exactly how far the teeth need to protrude or exactly where and how deep the holes in your key need to be,
	to be able to unlock the door;
		you just select the right key
			(the whole physical thing at once, with all the right teeth or holes)
				by recognizing its form or its label.

\input{tex/computers-and-links.subsec}

\SubsectionWithoutText{Logging in}
\subsubsection{Part I}
Before starting the login process you select a key from your key ring;
the one you know to belong to the website you are trying to login to.
\par
To login you don't send your key to the server.
Instead, you generate a secret random value,
which you encrypt with your key using a simple \XOR\ operation.
Your key is totally random and the secret random value you encrypt with it is also, well\ldots totally random.
Mixing these random bits gives another totally random value.
The website will receive this value and will try to decrypt this with the key that belongs to your account
%(we will keep you in the dark for now, about how this works).
(as explained in detail below).
When it has decrypted the random login value,
it will know which secret random value you generated.
Instead of telling you the secret random number,
the website sends a hash value of it,
because otherwise someone able to see the network traffic will instantly know your key.
\par
If the website returns a hash value matching your secret random number, you know two things.
First, you know that the website you are talking to can successfully decrypt your random value.
It can only do this if it has a matching key.
Second, you know you are talking to the same site that sent you your key when you applied for an account.
This implies that if you trusted that site then, you can trust this site now, as it can only be the same site.

\subsubsection{Part II}
%This is all great and very sophisticated, but the website wouldn't dare share your wonder about this,
%as you may well fake your enthusiasm, and flatly lie about the correctness of the hash it has sent you.
%To see if your claims hold,
To confirm that you are who you claim to be,
the website will do the same as you and send you an encrypted secret random value.
This cryptogram can only be decrypted by someone owning the right key for the account.
Using \XOR\ with the key you have originally selected from your key ring,
you decrypt it.
Then,
using \XOR\ again,
you combine the two random values you now have,
and return this value to the website.
\par
When the website receives a value matching its secret random number, it knows two things.
First, the user on the other side can successfully decrypt the secret random number.
It can only do this if it has a matching key.
Second, the website knows that this key is from the same key ring that was used when applying for an account.
This implies that if it trusted this user then, it can trust this user now, as it can only be the same user.

\SubsectionWithoutText{Keys}
\subsubsection{On site keys and user keys}
Beware, the icky parts start here (a bit).
\par
When applying for an account, a site key is created by a random generator.
The user key is then computed as the encryption of the site key with a Secret Key, with a block cypher like \AES.
\par
When the user encrypts its secret random value with its key,
it uses \XOR.
This is a very simple and reversible encryption method.
But as both key and value are utterly random, no information is stored in the encrypted value.
The website needs to decrypt the value, and this can only be done with the user key.
But the website does not store user keys,
it stores only site keys\ldots
\par
%The solution to this may be a bit of a dissappointment and a cheat:
The user key is temporarily regenerated by encrypting it again with the Secret Key.
Once the user key is available, the random value can easily be decrypted by \XOR-ing it.
Also, the cryptogram that is sent to the user is a random value \XOR-red with this regenerated user key.
\par
%The clever part,
Importantly,
however,
is that all the cryptographic functions the website is required to perform take place inside a Hardware Security Module,
or HSM for short.
%Secret Keys can be put in the HSM and made to good use there,
Secret Keys are stored and used in the HSM,
but can never be retrieved from it.
The HSM computes all values needed for the login process,
without ever revealing the keys that are used,
not even to a hacker with full control of the HSM.
\par
In the end,
only random bits enter the HSM,
and only random bits leave it.

\subsubsection{Key lifetime}
%Every key has its lifetime, so Secret Keys need changing every so often, as a good security measure.
As a good security measure, Secret Keys need changing every so often.
The same goes for keys on the user's key ring.
\par
The login protocol is capable of changing keys on the website and keys on the key ring,
without any effort on the user's side.
%Well\ldots, a little program will do it for you, without you noticing.
\par
Logins can be granted,
but new keys may not be,
which results in the expiration of an account.
A website may deny a user new keys when payment is due,
giving users limited login rights.
At any time new keys can be provided again.
\par\vspace{10mm}\noindent
How this all works,
how keys are used,
what a website can do with logins,
and more,
can all be read in the next chapters.
