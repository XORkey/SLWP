%
% $Header: /usr/local/src/slwp/article/tex/login.tex,v 1.9 2014/02/01 13:15:42 timo Exp $
%
% AUTHOR:	ing. T.M.C. Ruiter
%
\Section{Logging in}
\label{logging_in}
After the user has unencrypted his key ring and selected a key number ($k$),
and given its means of identification,
the login process can commence.
\Subsection{User Identification}
\label{sec:user_ids}
When logging in,
the user will need something as a means of identification.
Instead of sending a traditional alphanumeric userid,
we send a specially crafted user hash value,
henceforth called $U_h$.
This value will be used by the web server to identify the user,
and the web server will only store hashes in its user database.
It is chosen once per website and can remain the same as long as the key ring exists.
\subsubsection{The identication value}
The user identification value is the combination of several obligatory and optional values:
the key ring identifier,
the hostname of the URL,
a user identification string
(comparable with the traditional ``userid''),
a biometric value,
and the value supplied by a physical device like a hardware token.
The first two ingredients are always used;
the identification string is optional if a biometric value is used, but mandatory otherwise;
the last two may be added for more flavor.
If a biometric value is used,
it may be used by itself to augment the hash value,
or as a replacement of the user identification string.
\par
The most basic form of the user identification string is to give a name
(real or imaginary);
in this case in the form of a string of letters.
There are no restrictions or constraints as to the contents of this value,
except that it must not be empty.
This means that it may be the same and conveniently reused for each and every website.
Another form may include biometric features,
like an iris scan,
a fingerprint,
or hand geometry.
\subsubsection{Basic construction}
The hash that is sent to the website to identify the user,
is constructed as follows.
Compute the \SHA\ hash of the key ring identifier $Z[0]$
\[H_0=\F{\SHA}{Z[0]}\]
Then,
compute the \SHA\ of a concatenation of the user identification string
and the domain name part of the URL.
\[H_1=\F{\SHA}{\mathtt{ID+domain\_name}}\]
Then, compute the final 256-bit value $U_h$ by means of an \XOR\ operation:
\[U_h=\xor{H_0}{H_1}\]
This is $U_h$ in its basic form.
\subsubsection{Fortifying $U_h$}
The $U_h$ value may be strengthened in several ways, always computing $H_0$ as before:
\[H_0=\F{\SHA}{Z[0]}\]
\par
If a biometric value is used it may replace the user identification string when calculating $H_1$:
\[H_1=\F{\SHA}{\mathtt{bio\_value+domain\_name}}\]
\[U_h=\xor{H_0}{H_1}\]
Instead of replacing the userid,
it can also be added as a separate value:
\[H_1=\F{\SHA}{\mathtt{ID+domain\_name}}\]
\[H_2=\F{\SHA}{\mathtt{bio\_value}}\]
\[U_h=\xor{\xor{H_0}{H_1}}{H_2}\]
\par
A value from a hardware token can be added as well:
\[H_1=\F{\SHA}{\mathtt{ID+domain\_name}}\]
\[H_2=\F{\SHA}{\mathtt{token\_value}}\]
\[U_h=\xor{\xor{H_0}{H_1}}{H_2}\]
\par
Finally,
things can be combined multiple times
to physically bind an account to two people:
\[H_1=\F{\SHA}{\mathtt{ID+domain\_name}}\]
\[H_2=\F{\SHA}{\mathtt{bio\_value1+bio\_value2}}\]
\[U_h=\xor{\xor{H_0}{H_1}}{H_2}\]
Or strenghten it further by adding a hardware token also:
\[H_1=\F{\SHA}{\mathtt{ID+domain\_name}}\]
\[H_2=\F{\SHA}{\mathtt{bio\_value1+bio\_value2}}\]
\[H_3=\F{\SHA}{\mathtt{token\_value}}\]
\[U_h=\xor{\xor{\xor{H_0}{H_1}}{H_2}}{H_3}\]
\subsubsection{Multiple $U_h$ values for an account}
Websites may be accessed from different devices like a tablet, a smart phone, or a PC.
Each device may offer different means of collecting biometric values, if at all.
Therefore, it may be convenient for a website to offer the user to have multiple $U_h$ values for his account.
\par
Most PCs are equipped with a USB port; a smart phone may have a fingerprint reader.
A website may allow the user to either use a hardware token plugged into the PC,
or to scan its fingerprint when using its mobile device.
Both values yield different $U_h$ values,
but both may be linked to the same account.
\subsubsection{$U_h$ security aspects}
This hash value is a combination of something the user has,
or has access to
(the key ring)
and something the user knows
(the chosen userid)
or is uniquely his
(some biometric value)
and optionally something the user physically possesses
(a hardware token).
\par
Using the key ring identifier $Z[0]$ ties $U_h$ with the key ring.
This way, you need the complete key ring for logging in, not just the right key for the website.
Combining $H_0$ with other values makes it hard to match a stolen set of hashes from a web server
with a set of key rings from a key ring server.
\par
The userid can be anything, but must not be empty; otherwise the key ring identifier is exposed.
The value
\[\xor{\xor{U_h}{\F{\SHA}{\mathtt{domain\_name}}}}{\F{\SHA}{\mathtt{other\_domain\_name}}}\]
would be a valid $U_h$ for ${\mathtt{other\_domain\_name}}$.
Although it is more convenient for the user to have no userid at all,
and a stolen $U_h$ by itself is not enough for logging in,
the possibility to infer other $U_h$ values from a given $U_h$ is not a desired feature.
\par
Combining a hash of the domain name from the URL with the user identification string
(even if these are the same for both sites)
ensures three things.
\begin{itemize}
\item It makes the resulting $U_h$ value unique for each website.
An $U_h$ value obtained from one site is always invalid for another site.
\item Accounts on different websites cannot be correlated and traced back to a single user,
so colluding websites cannot link accounts.
\item A $U_h$ value for a phishing site will be computed using \emph{that} site's URL,
instead the URL of the site it mimics;
the $U_h$ values will be useless,
and therefore phishing itself.
\end{itemize}
\par
If a hardware token is used there are two possibilities.
If the value provided by the token is fixed by nature then the resulting hash is the same each time it is used.
In that case the resulting $U_h$ may be used as a key of the user database.
If the token value is different each time
(e.g. time based)
then the value $\xor{U_h}{H_2}$ should be used as the key,
where $H_2$ is recomputed at the server side.

%%%%%%%%%%%%%%%%%%%%%%%%%%%%%%%%%%%%%%%%%%%%%%%%%%%%%%%%%%%%%%%%%%%
\SubsectionWithoutText{Applying for an account}
\label{sec:applying}
%See section~\vref{scheme:new_account} for an accompanying flow diagram.
\subsubsection{Step 1: A new user makes an application}
\label{sec:apply_step1}
When a new user applies for an account,
apart from any personal details the website is interested in,
he presents the web server with its hash value $U_h$.
Furthermore,
the user should select the index ($d$) of a hitherto unused key.
%and use $n$ bits of it,
%at the discretion of,
%and indicated by,
%the web server.
%\[K_d=\mod{Z[d]}{2^n}\]
%Alternatively, the user may just generate a new dummy key on the spot:
Then, the user generates a new dummy key
\[K_d=\F{Random}{n}\]
and remember it and the key index $d$ for later use.
\par
Although this is a dummy key that is not used for logins
(it will even be replaced after a successful application for an account)
it still needs to be secured,
since it is used in the application process.
Eventually,
this key needs to find its way to the login server,
so we will encrypt it with its public encryption key $K_{le}$:
\[K_h=\Enc{K_d}{K_{le}}\]
The user will then send $U_h$, $K_h$, and its age, shoe size, and gender to the web server.

\subsubsection{Step 2: Response by the web server}
\label{sec:apply_step2}
The user's particulars and his identification hash value $U_h$ are relayed to the accounts server,
which will create the account.
The web server will request values for a new user from the login server,
by forwarding the dummy key ($K_h$),
and indicating how long the keys should be
($n$ bits long).

\subsubsection{Step 3: What have we here?}
\label{sec:apply_step3}
The login server will do the following, all inside the HSM:
\begin{enumerate}
\item Generate a $n$-bit pseudo-random site key $K_s$ for the user.
\[K_s=\F{Random}{n}\]
This key is then prepaired for use by the accounts server.
This server needs to send keys encrypted with the $K_{le}$, the public encryption key of the login server,
which we will do here beforehand:
\[K_y=\Enc{K_s}{K_{le}}\]
\item Encrypt the $K_s$ with the current Secret key $S[0]$ to yield the new user key $K_u$.
\[K_u=\aes{K_s}{S[0]}\]
\item The (2048 bit) $K_h$ value is decrypted with the private key $K_{LE}$ to yield $K_d$:
\[K_d=\Dec{K_h}{K_{LE}}\]
which is subsequently used to encrypt $K_u$:
\[K_x=\xor{K_u}{K_d}\]
\end{enumerate}
Both values $K_x$ and $K_y$ are returned to the web server.
\subsubsection{Step 4: Here are the results}
\label{sec:apply_step4}
The web server will then relay the new encrypted key $K_y$ and $U_h$ to the accounts server.
The accounts server will take $K_y$
and update the new account with it.
These encrypted keys can be sent to the login server by the web server during logins,
without further processing.

\subsubsection{Step 5: New account confirmed to the user}
\label{sec:apply_step5}
The encrypted user key $K_x$ is presented to the user,
so it can store it in its key ring.
The user can decrypt its new key with by using $K_d$ it used originally:
\[K_u=\xor{K_x}{K_d}\]
Finally, $K_u$ is imported in the key ring with something like
\[Z[d]=\xor{(\F{Random}{128-n}\ \mathbf{<<}\ n)}{K_u}\]
The dummy key at $Z[d]$ is overwritten with 128 random bits,
of which the last $n$ bits contain the new key $K_u$.

\Subsection{Login scheme}
\label{sec:full_login}
The procedure described below to login is the full login scheme.
%See sections~\vref{scheme:simplest_login},
%\vref{scheme:login_with_old_key},
%and \vref{scheme:login_with_wrong_key} for explanatory flow diagrams.

%%%%%%%%%%%%%%%%%%%%%%%%%%%%%%%%%%%%%%%%%%%%%%%%%%%%%%%%%%%%%%%%%%%
\subsubsection{Step 1: Starting a login procedure}
\label{sec:login_step1}
The user's login program has to compute the following:
\begin{enumerate}
\item The userid hash $U_h$ (see section \ref{sec:user_ids}).
\item An $n$-bit%
\footnote{Here, $n$ is conveyed to the user through the login webpage.}
pseudo-random number $R_u$:
\[R_u=\F{Random}{n}\]
\item The user key $K_u$ is taken from the key ring $Z$ at index $k$.
As this is 128-bit value,
take the least significant $n$ bits of it.
The pseudo-random number is encrypted with it to get the value $A_u$:
\[K_u=\mod{Z[k]}{2^n}\]
\[A_u=\xor{R_u}{K_u}\]
\item The web server will return a hash value of the user's pseudo-random value.
To be able to verify this hash,
we compute our own hash $B_u$ to verify it with:
\[B_u=\xor{K_u}{\sha{R_u}\]}
\end{enumerate}
The special userid hash $U_h$ and the encrypted pseudo-random number $A_u$ are sent to the web server.

%%%%%%%%%%%%%%%%%%%%%%%%%%%%%%%%%%%%%%%%%%%%%%%%%%%%%%%%%%%%%%%%%%%
\subsubsection{Step 2: Site key lookup and key matching attempts}
\label{sec:login_step2}
The web server runs a login procedure%
%, see Algorithm~\vref{alg:web server_login}
.
\par
After receiving $U_h$ from the user, the encrypted site key $K_y$ needs to be looked up.
A query with $U_h$ is sent by the web server to the accounts server.
If a match is found, $K_y$ is returned;
otherwise, $K_y$ is set to zero, indicating that no such record exists.
In the latter case, the values $B_s$ and $P_s$ are both set to zero as well, and returned to the user.
The user needs to rethink his actions and start over.
\par
Now,
an iterative process starts,
trying to find the right Secret key to log the user in.
The web server will send five values to the login server:
$A_u$;
$K_y$;
the number of bits in the user and site keys ($n$);
a boolean value indicating whether the user is capable of performing a verify operation using a public signing key ($v$);
and a Secret key index $i$.
This will be repeated
(increasing index $i$),
until a match is found,
or no more keys can or will be tried.
\paragraph{Account status}
Along with $K_y$ the account status $s$ may also be returned.
This status may indicate whether we allow the user to login or not.
If something is required from the user
(payment, or otherwise)
we might want to expire the account until the requirement is met.
We use the value $\mathtt{MAX\_ACTIVE\_KEYS}$ to limit the number of keys we want to try.
\par
If $\mathtt{MAX\_ACTIVE\_KEYS}$ is set to 1 then accounts will expire immediately when a new Secret key is inserted.
When set to 3,
there will be a grace period of 2 times the Secret keys replacement interval.
This means that login is granted for this time,
in which the user can pay his debts.
If that does not happen,
the account will expire.
\par
Users that have not logged in for a while,
but have paid their monthly fees,
can still login without problems because all Secret keys will be tried for such logins.
\paragraph{Computing index starting value}
Finally, we could consider reducing the number of login attempts.
Instead of always starting with index $0$ for Secret key $S[0]$,
the date of the last login of the user can be taken into account.
If the query with $U_h$ as key,
that is sent to the accounts database,
would also return the last login date,
a starting key index $i$ could be computed.

%%%%%%%%%%%%%%%%%%%%%%%%%%%%%%%%%%%%%%%%%%%%%%%%%%%%%%%%%%%%%%%%%%%
\subsubsection{Step 3: Login server actions}
\label{sec:login_step3}
The login server is a processor of login values and calls a function of the HSM to compute them
%(see Algorithm~\vref{alg:hsm})%
.
\par
In case $i$ is less than $m$
(the number of stored Secret keys, see section~\vref{sec:secret_keys})
the login server calculates the following, all within the HSM:
\begin{enumerate}
\item Decrypt the site key, using the private encryption key $K_{LE}$:
\[K_s=\mod{\Dec{K_y}{K_{LE}}}{2^n}\]
\item Temporarily, regenerate a user key, using the same algorithm as when the key was originally generated:
\[K_u=\aes{K_s}{S[i]}\]
with $S[i]$ the $i$-th Secret key stored in the HSM.
\item With the user key $K_u$, decrypt the pseudo-random the user has sent:
\[R_u=\xor{A_u}{K_u}\]
\item Calculate a hash with which the user can verify we own the site key $K_s$ and a corresponding Secret key $S[i]$.
Hide this value behind $K_u$:
\[B_s=\xor{K_u}{\sha{R_u}}\]
\item If the user is able to verify the signature of the login server,
then the value $B_s$ will be replaced with a signature thereof with the signing key $K_{LS}$:
\[B_s=\Sgn{B_s}{K_{LS}}\]
We use and send only the signature, not the $B_s$ value also.
\item To be able to verify that the user owns and knows his user key $K_u$,
is not just sending a replay of some earlier successfull login sequence,
and will be unable to lie about the correctness of the hash of the pseudo-random the web server will send,
we calculate a challenge for the user.
\par
If the index $i$ equals zero, generate a pseudo-random value
\[R_s=\F{Random}{n}\]
otherwise, compute
\[R_s=\aes{K_s}{S[0]}\]
which will be the new key for the user.
\par
We then encrypt $R_s$ to a value $P_s$ the user can decrypt using its key:
\[P_s=\xor{R_s}{K_u}\]
This value will be different for each time the loop is executed,
even if the same new key is transmitted.
This is because $K_u$ will change each time,
as it is the encryption of $K_s$ with a different secret key $S[i]$.
\item We now know both random values.
If the user knows them also
(by successfully decrypting $P_s$)
he can prove this by sending the \XOR\ of $R_u$ and the hash of $R_s$.
\[Q_s=\xor{R_u}{\sha{R_s}}\]
This is the value that the web server should compare.
\end{enumerate}
The HSM will produce the values $B_s$,
$P_s$,
and $Q_s$ as a result of the calculations and the login server will send them back to the web server,
as an answer to the five values it was given ($A_u$, $K_y$, $n$, $v$, and $i$).
The HSM will delete all temporary values from memory immediately afterwards,
and remember only its Secret keys.
\par
Should index $i$ equal $m$, then the array of Secret keys is exhausted.
If we reach this situation, we cannot log the user in since all possible attempts have failed.
Return zero values for $B_s$, $P_s$, and $Q_s$ to indicate this.

%------------------------------------------------------------------
\subsubsection{Step 4: User verifies site key}
\label{sec:login_step4}
The web server sends $B_s$ and $P_s$ to the user.
If both are zero, the login has failed and the user should return to step 1.
It is advisable for the user to choose different values for the next try.
\par
If $B_s$ is nonzero, the user verifies whether it matches with $B_u$.
\par
If the user is not able to verify a digital signature,
just compare values $B_u$ and $\xor{K_u}{B_s}$.
Otherwise,
it should try to verify the combination of $B_u$
(the hash over $R_u$)
and the signature thereof,
which is in $B_s$:
\[\Ver{B_u+\xor{K_s}{B_s}}{K_{ls}}\]
\par
If the two values do not match,
the user either has selected the wrong key or uses an old key.
Logging in with an old key every now and then is inherent to this scheme,
as most Secret keys will be changed at regular intervals.
\par
To indicate that no match has been found, we send
\[Q_u=\mathtt{0x0}\]
($n$ zeroes) to the web server.
The web server will need to start over, and send values $A_u$, $K_y$, $n$, $v$, and $i+1$ to the login server.
So, back to step 3.
\par
After two unsuccessful attempts the user may be presented a question
whether it likes to abort the login procedure or continue trying with this key.
To abort, use:
\[Q_u=\mathtt{0x1}\]
and send this to the web server
(instead of $\mathtt{0x0}$).
We return to step 1.

%------------------------------------------------------------------
\subsubsection{Step 5: Site verifies user key}
\label{sec:login_step5}
If $B_s$ matches $B_u$, then the web server has found a Secret key for the user.
The user can now prove it has control over its user key by computing $R_s$ and $Q_u$:
\[R_s=\xor{P_s}{K_u}\]
\[Q_u=\xor{R_u}{\sha{R_s}}\]
which is sent to the web server as proof.
If the web server accepts $Q_u$ as correct it will log the user in.
\par
If the web server receives a value $Q_u$ that does \emph{not} match,
something fishy is going on.
Further attempts for this account, or from that source should be scrutinized.

%------------------------------------------------------------------
\subsubsection{Step 6: User key replacement}
\label{sec:login_step6}
\label{sec:conditional_key_replacement}
If this attempt to login was not the first with this key,
the web server may have sent the user a new key in $R_s$ (instead of a pseudo-random value).
The user should store this new key in the key ring, overwriting the old key the user has just used:
\[Z[k]=\xor{(\F{Random}{128-n}\ \mathbf{<<}\ n)}{R_s}\]
where all bits of $Z[k]$ are replaced.
\par
For most websites,
restoring user keys that refer to the most recent Secret key $S[0]$ will be done without hesitation.
For some,
refreshing keys will be done only when certain conditions are met,
like payment of monthly fees,
a certain number of reviews written,
or some amount of data uploaded.
Until then,
logging in with a valid
(but in this context a typically `old')
key is granted.
But if, for instance, payment is overdue, the key will expire and logging in is no longer possible.
\par
To deny a user a new key, and to indicate to the user that no new key is sent,
the website will replace the new key in $P_s$ with $A_u$
\[P_s = A_u\]
and return this value to the user.
In this case,
no keys should be decrypted or overwritten.
