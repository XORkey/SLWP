\clearpage
\section{Implementation}

%%%%%%%%%%%%%%%%%%%%%%%%%%%%%%%%%%%%%%%%%%%%%%%%%%%%%%%%%%%%%%%%%%%
\subsection{Algorithms}
The several algorithms explained in Section \ref{logging_in} are represented here in a consise manner.
\subsubsection{Userid hashes}
The hash that is used in the login procedure is composed of something you have
(the keyring)
and something you know
(the userid).
Together,
they are one part of the values needed to login.
It is calculated with Algorithm~\vref{alg:userid_hash}.
\begin{algorithm}
\caption{Computing the hash of the userid.}
\label{alg:userid_hash}
\begin{algorithmic}[1]
\Procedure{UserIDHash}{$Z,userid$}
\State $H_0\gets\F{\SHA}{Z[0]}$\Comment{Hash the keyring identifier.}
\State $H_1\gets\F{\SHA}{\texttt{userid}}$\Comment{Hash the userid.}
\State \Return{$\xor{H_0}{H_1}$}\Comment{This will be value $U_h$.}
\EndProcedure
\end{algorithmic}
\end{algorithm}
%%%%%%%%%%%%%%
\subsubsection{New account}
Algorithm~\vref{alg:new_account} is run by the login server to get the login values for a new user.
It is called by the webserver when the user has provided all necessary data.
The values returned will be relayed to the accounts server.
\begin{algorithm}
\caption{Generate values for a new account.}
\label{alg:new_account}
\begin{algorithmic}[1]
\Procedure{NewAccount}{$K_h$,$n$}
\State $K_s\gets\Call{Random}{n}$\Comment{Generate n-bit pseudo-random number.}
\State $K_y\gets\Enc{K_s}{K_{ae}}$	\Comment{Encrypt $K_s$ with accounts server public key.}
\State $K_u\gets\aes{K_s}{S[0]}$ \Comment{This is the new user key.}
\State $K_x\gets\xor{K_u}{\Dec{K_h}{K_{LE}}}$ \Comment{Encrypt this with the decrypted dummy key.}
\State \Return{$K_x,K_y$}
\EndProcedure
\end{algorithmic}
\end{algorithm}
\par
%%%%%%%%%%%%%%
\subsubsection{User login program}
A user must complete Algorithm~\vref{alg:user_login} successfully to login.
\begin{algorithm}
\caption{The login program of the user.}
\label{alg:user_login}
\begin{algorithmic}[1]
\Procedure{UserLogin}{$U_h,Z,k,K_{ls}$}
\State $R_u\gets\Call{Random}{n}$\Comment{Generate n-bit pseudo-random number.}
\State $K_u\gets Z[k]$\Comment{Take key from keyring.}
\State $A_u\gets\xor{R_u}{K_u}$\Comment{Compute a challenge.}
\State $B_u\gets\sha{R_u}$\Comment{And the response also.}
\State $Q_u,j\gets 0,0$\Comment{Initialize.}
\State $B_s,P_s\gets$\Call{SendToWebserver}{$U_h,A_u$}\Comment{Wait for $B_s$ and $P_s$.}
\While{$B_s\ne 0$}\Comment{Website is trying keys for us.}
\If{$\Ver{B_u+B_s}{K_{ls}}$}	\Comment{Verify $B_s$. Website found the right key!}
\State $R_s\gets\xor{P_s}{K_u}$
\State $Q_u\gets\xor{R_s}{R_u}$\Comment{Compute response to $P_s$.}
\Else
\State $Q_u\gets\Call{AskToContinue}{j}$\Comment{Return 0 to continue; 1 to stop.}
\EndIf
\State $B_s,P_s\gets$\Call{SendToWebserver}{$Q_u$}\Comment{Answer to million dollar question.}
\State $j\gets j+1$ \Comment{One more attempt.}
\EndWhile
\If{$Q_u>1\ \mathbf{and}\ P_s>1$} \Comment{Login succeeded.}
\If{$j>1$} \Comment{Not the first attempt with this key.}
\If{$P_s\ne A_u$} \Comment{We are not denied a new key.}
\State\Call{UpdateKeyring}{$Z,k,R_s$} \Comment{New key is sent with $R_s$.}
\EndIf
\EndIf
\EndIf
\EndProcedure
\end{algorithmic}
\end{algorithm}
It is called with the hash, the keyring, an index, and the decryption key from the certificate of the login server.
The function \textsc{AskToContinue} is called with the attempts counter as parameter.
Only after two attempts should the user be asked if further attempts should be tried.
It is up to the implementer if this question is asked once,
at every further attempt,
or at some other interval.
If no actual question is asked,
the function can return $\mathtt{0}$ directly.
%%%%%%%%%%%%%%
\subsubsection{Webserver program}
With Algorithm~\vref{alg:webserver_login} the webserver determines whether a user should be granted access.
This simple algorithm does not calculate anything,
it just compares values and sends data around.
\par
The accounts server can indicate to the webserver
(through account status $s$,
and the site key $K_s$)
what is required of the user;
either now or in the near future.
\begin{algorithm}
\caption{The login program of the webserver.}
\label{alg:webserver_login}
\begin{algorithmic}[1]
\Procedure{WebserverLogin}{$U_h,A_u,v$}
\State $F,Q_s,Q_u\gets 0,41,43$ \Comment{Initialize.}
\State $k\gets\mathtt{MAX\_KEYS}$ \Comment{Use all keys.}
\State $K_a,s,D\gets\Call{GetAccountInfo}{U_h}$ \Comment{Query the accounts server.}
\If{$s>0$} \Comment{Something required from the user.}
\State $k\gets\mathtt{MAX\_ACTIVE\_KEYS}$ \Comment{Use only some keys.}
\EndIf
\If {$K_a\ne 0$}
\State $i\gets\Call{GetIndex}{s,D}$	\Comment{Use account status and last login date to get $i$.}
%\Repeat
\While{$i<k\ \mathbf{and}\ Q_u\neq Q_s$}
\State $B_s,P_s,Q_s\gets$\Call{HSM}{$A_u,K_a,n,v,i$} \Comment{Call this function on login server.}
\If{$Q_s\ne 0$}
\State $Q_u\gets$\Call{SendToUser}{$B_s,P_s$}\Comment{Wait for $Q_u$.}
\Else\Comment{Array S exhausted.}
\State $Q_u\gets Q_s$\Comment{No point going on: terminate loop.}
\EndIf
\If{$Q_u=1$}\Comment{User aborted login.}
\State $Q_s\gets Q_u$\Comment{Terminate loop at user's request.}
\EndIf
%\If{$s>0$} \Comment{Something required from the user.}
%\If{$i<\mathtt{MAX\_ACTIVE\_KEYS}$}
\State $i\gets i+1$
%\Else
%\State $Q_u\gets Q_s$\Comment{Terminate loop ($Q_s = 0$).}
%\State $i\gets i-1$ \Comment{Deny user new keys.}
%\Else \Comment{All is OK.}
%\State $i\gets i+1$ \Comment{Normal key index.}
%\EndIf
%\EndIf
\EndWhile
%\Until{$Q_u=Q_s$}
\State $F\gets Q_s$ \Comment{Return $Q_s$ by default.}
\If {$Q_s>1\ \mathbf{and}\ s>0$} \Comment{Login succeeded but something required.}
\State $F\gets A_u$ \Comment{Login granted for now.}
\EndIf
\EndIf
\State \Call{SendToUser}{$0,F$} \Comment{Indicate login state.}
\EndProcedure
\end{algorithmic}
\end{algorithm}
%%%%%%%%%%%%%%
\subsubsection{Login server program}
Algorithm~\vref{alg:hsm} computes values for the webserver to check.
It uses private key $K_{LE}$ for decryption of the site key.
Private key $K_{LS}$ is used to sign the user random $R_u$ and yield $B_s$.
\nopagebreak
\begin{algorithm}[H]
\caption{The program of the login server, running inside the HSM.}
\label{alg:hsm}
\begin{algorithmic}[1]
\Procedure{HSM}{$A_u$, $K_a$, $n$, $v$, $i$}
%\State $I\gets\Call{Abs}{i}$ \Comment{Index $i$ may be negative to deny new keys.}
\If{$i< \mathtt{MAX\_KEYS}$}\Comment{Use all keys from array $S$.}
\State $K_s\gets\mod{\Dec{K_a}{K_{LE}}}{2^n}$	\Comment{Decrypt the site key $K_s$.}
\State $K_u\gets\aes{K_s}{S[i]}$	\Comment{Temporarily regenerate user key.}
\State $R_u\gets\xor{A_u}{K_u}$	\Comment{Calculate the user random.}
\State $B_s\gets\sha{R_u}$	\Comment{Compute the hash over the random.}
\If{$v = \mathtt{true}$}	\Comment{User can verify using certifcate.}
\State $B_s\gets\Sgn{B_s}{K_{LS}}$	\Comment{Replace hash with signature.}
\EndIf
\If{$i>0$}\Comment{Not the first attempt with $K_s$.}
\State $R_s\gets\aes{K_s}{S[0]}$\Comment{Send user a new key.}
\Else\Comment{First attempt or not allowed a new key.}
\State $R_s\gets\Call{Random}{n}$\Comment{Generate n-bit pseudo-random number.}
\EndIf
\State $P_s\gets\xor{R_s}{K_u}$\Comment{Compute a challenge.}
\State $Q_s\gets\xor{R_s}{R_u}$\Comment{And the response also.}
\Else\Comment{Array $S$ is exhausted.}
\State $B_s,P_s,Q_s\gets0,0,0$\Comment{It's game over.}
\EndIf
\State \Return $B_s, P_s, Q_s$\Comment{Return these to the webserver.}
\EndProcedure
\end{algorithmic}
\end{algorithm}
\subsection{Login webpages}
The login algorithms for both the user and the webserver are presented as contiguous programs.
Since there are several exchanges of values,
and the fact that the user has no login program at its disposal,
the algorithms need to be broken apart.
\par
The website can present login code to the user through the inclusion of JavaScript in the HTML login pages.
At the server side, PHP can be used to generate HTML with JavaScript.
%In this example implementation we use 32-bit random values.
\subsubsection{First page}
This can be a normal HTML page.
It must contain JavaScript code to start the login process,
by calculating $Au$, $Uh$, and the key index $k$, which the user must select.
It also calculates $Bu$, and its value is stored in the sessionStorage of the browser.
%See Listing \ref{login1} on page~\pageref{login1}.
\begin{algorithm}[H]
\caption{The function to start the login, included in the initial HTML page.}
\label{alg:startlogin}
\begin{algorithmic}[1]
\Procedure{StartLogin}{$k$, $\mathtt{userid}$}
\State $Ru\gets\Call{Random}{n}$
\State $H0\gets\F{\SHA}{Z[0]}$
\State $H1\gets\F{\SHA}{\mathtt{userid}}$
\State $Au\gets\xor{Ru}{Z[k]}$
\State $Uh\gets\xor{H0}{H1}$
\State $\mathtt{sessionStorage.Bu}\gets\sha{Ru}$
\State $\mathtt{sessionStorage.j}\gets 0$
\State \Call{Submit}{$Au$,$Uh$}
\EndProcedure
\end{algorithmic}
\end{algorithm}
%\lstset{language=HTML,basicstyle=\footnotesize\ttfamily}
%\begin{lstlisting}[breaklines=true,frame=lines,float=tbhp,label=login1,title=login.html,caption=Initial Login Page]
%<!DOCTYPE html>
%<html>
%  <head>
%    <title>Login 32-bit implementation</title>
%    <meta charset="utf-8">
%    <link rel="stylesheet" type="text/css" href="style/login.css" />
%    <script src="http://localhost:8888/slwp/sha256.js"></script>
%    <script src="http://localhost:8888/slwp/keyring.js"></script>
%    <script>
%      function StartLogin(keynumber)
%      {
%        var Ru = new Number(Math.floor(Math.random()*Math.pow(2, 31)));
%        var H0 = SHA256(Z[0]);
%        document.forms['login']['k'].value = keynumber;
%        document.forms['login']['Au'].value = XorKu(Ru, keynumber);
%        document.forms['login']['Uh'].value =
%		  parseInt(SHA256(UserId + H0.substr(UserId.length - 1)).substr(-8), 16);
%        sessionStorage.Bu =
%		  parseInt(SHA256(Ru.toString()).substr(-8), 16);
%        sessionStorage.j = Number(0);
%        document.login.submit();
%      }
%    </script>
%  </head>
%  <body>
%    <h1>Secure Login without Passwords</h1>
%    <input name="keyring" type="file">
%    <form name="login" method="post" action="login2.php">
%      <input id="k"  name="k"  type="password" hidden>
%      <input id="Uh" name="Uh" type="password" hidden>
%      <input id="Au" name="Au" type="password" hidden>
%      <input id="ID" name="ID" type="text" value="default userid" readonly>
%    </form>
%    <div class=keypad>
%      <script>DisplayKeys(true);</script>
%    </div>
%  </body>
%</html>
%\end{lstlisting}
\subsubsection{Initial PHP page}
The initial PHP page queries the accounts database for a user with hash $U_h$.
If such user hash exists the site key $K_a$ is returned and stored in the session array.
Otherwise a page is displayed to inform the user the login process has failed due to a mismatch.
\par
From that moment on,
the webserver sends webpages to the browser that calculate the value $Q_u$.
These pages are identical, except for the values of $B_s$ and $P_s$ that are sent along.
After computation of $Q_s$, the form,
containing an input element that will hold the $Q_u$ value,
is automatically submitted.
\par
The initial PHP page sends the first of these
(almost identical)
webpages;
the action option in the form will call the second PHP page for all subsequent calculations.
%See Listing \ref{login2} on page~\pageref{login2}.
\begin{algorithm}[H]
\caption{First PHP page.}
\label{alg:phplogin1}
\begin{algorithmic}[1]
\Procedure{PHPlogin1}{$Uh$, $Au$}
\State $\mathtt{\$\_SESSION[Ks]}\gets\Call{getAccountInfo}{Uh}$
\State $\mathtt{\$\_SESSION[Au]}\gets Au$
\State $\mathtt{\$\_SESSION[i]}\gets 0$
\State $Bs,Qs,Ps\gets 0,0,0$
\If{$\mathtt{\$\_SESSION[Ks]}\neq 0$}
\State \Call{SendBsPs}{}
\Else
\State \Call{LoginFailed}{"No such user."}
\EndIf
\EndProcedure
\end{algorithmic}
\end{algorithm}
%\lstset{language=PHP}
%\begin{lstlisting}[frame=lines,float=tbhp,label=login2,title=login2.php,caption=Second Login Page]
%<?php
%  session_start();
%  require 'user_pages.php';
%  $_SESSION["Ks"] = getAccountInfo($_POST["Uh"]);
%  $_SESSION["Au"] = $_POST["Au"];
%  $_SESSION["k"] = $_POST["k"];
%  $_SESSION["i"] = 0;
%  $Bs = $Qs = $Ps = 0;
%  if ($_SESSION["Ks"] != 0)
%    send_bs_ps();
%  else
%    login_failed("No such user");
%?>
%\end{lstlisting}
%%%%%
\subsubsection{Second PHP page}
The second,
and only other,
PHP page will compare values and send one of three possible webpages as a result of this:
login succeeded,
login failed,
or another copy of the calculation page for $Q_s$.
%See Listing \ref{login3} on page~\pageref{login3}.
%\lstset{language=PHP}
%\begin{lstlisting}[frame=lines,float=tbhp,label=login3,title=login3.php,caption=Final Login Page]
%<?php
%  session_start();
%  require 'user_pages.php';
%  if ($_POST["Qu"] == 1)
%    $_SESSION["Qs"] = $_POST["Qu"];
%  if ($_SESSION["s"] > 0)
%    $_SESSION["i"] -= 1;
%  else
%    $_SESSION["i"] += 1;
%  if ($_POST["Qu"] == $_SESSION["Qs"])
%  {
%    $F = $_SESSION["Qs"];
%    if ($F > 1 && $_SESSION["s"] > 0)
%      $F = $_SESSION["Au"];
%    if ($_POST["Qu"] > 1 && $F > 1)
%    {
%      $_SESSION["login_ok"] = $F;
%      login_succeeded("welcome.php");
%    }
%    else
%      login_failed("cancelled");
%  }
%  else
%    send_bs_ps();
%?>
%\end{lstlisting}
%\lstset{language=PHP}
%\begin{lstlisting}[frame=lines,float=tbhp,label=login_succeeded,title=user_pages.php,caption=login-succeeded]
%<?php
%  function login_succeeded($ref)
%  {
%    global $F;
%    ?>
%<!DOCTYPE HTML>
%<html>
%  <head>
%    <meta charset="utf-8">
%    <script src="http://localhost:8888/slwp/keyring.js"></script>
%    <title>Login successful</title>
%  </head>
%  <body>
%    <h1>LOGIN SUCCESS!!!</h1>
%    <a href="<?php echo $ref ?>">Go to the welcome page</a>
%    <a href="logout.php">Logout</a>
%    <script>
%      <?php
%      if ($F != $_SESSION["Au"])
%        echo "if (sessionStorage.j > 1)\n  UpdateKeyring(sessionStorage.Rs, " . $_SESSION["k"] . ");\n";
%      else
%        echo "document.write(\"Key update denied.\");\n";
%      ?>
%    </script>
%  </body>
%</html>
%    <?php
%  }
%?>
%\end{lstlisting}
%\lstset{language=PHP}
%\begin{lstlisting}[frame=lines,float=tbhp,label=login_failed,title=user_pages3.php,caption=login-failed]
%<?php
%  function login_failed($reason)
%  {
%    ?>
%<!DOCTYPE HTML>
%<html>
%  <head>
%    <meta charset="utf-8">
%    <title>Login FAILED</title>
%  </head>
%  <body>
%    <h1>Login FAILED (<?php echo $reason ?>)</h1>
%    <p>Combination of userid and key number did not yield valid login credentials.</p>
%      <a href="login.html">Try again...</a>
%  </body>
%</html>
%    <?php
%  }
%?>
%\end{lstlisting}
%\lstset{language=PHP}
%\begin{lstlisting}[frame=lines,float=tbhp,label=send_bs_ps,title=user_pages3.php,caption=send-bs-ps]
%<?php
%  require 'hsm.php';
%  function send_bs_ps()
%  {
%    global $Qs, $Bs, $Ps;
%    HSM($_SESSION["Au"], $_SESSION["Ks"], $_SESSION["i"]);
%    if ($Qs != 0)          // Found something to test.
%    {
%      $_SESSION["Qs"] = $Qs;      // Save Qs.
%    ?>
%<!DOCTYPE HTML>
%<html>
%  <head>
%    <meta charset="utf-8">
%    <script src="http://localhost:8888/slwp/sha256.js"></script>
%    <script src="http://localhost:8888/slwp/keyring.js"></script>
%    <title>Login step</title>
%    <script>
%      function LoginStep()
%      {
%        var Rs = new Number(XorKu(<?php echo $Ps . ", " . $_SESSION["k"] ?>));
%        sessionStorage.Rs = Rs;
%        if (<?php echo "\"" . $Bs . "\"" ?> == sessionStorage.Bu)
%          document.getElementById("Qu").value =
%		    parseInt(SHA256(Rs.toString()).substr(-8), 16);
%        else
%          document.getElementById("Qu").value =
%		    AskToContinue(sessionStorage.j);
%        sessionStorage.j = Number(sessionStorage.j) + 1;
%        document.login_step.submit();
%      }
%    </script>
%  </head>
%  <body onload=LoginStep()>
%    <form name="login_step" method=post action=login3.php>
%      <input type=text id=Qu name=Qu required>
%    </form>
%  </body>
%</html>
%    <?php
%    }
%    else              // No more keys to try.  End of the line.
%    {
%      login_failed("No more server keys");
%    }
%  }
%?>
%\end{lstlisting}
%\lstset{language=PHP}
%\begin{lstlisting}[frame=lines,float=tbhp,label=HSM,title=hsm.php,caption=HSM function]
%<?php
%  require 'crypto.php';  // Random(), AES(), SHA().
%  function HSM($Au, $Ks, $i)
%  {
%    global $Bs, $Ps, $Qs;
%    static $S = array(0x4452ff1e, 0x0351cb58, 0x34fd12bc, 0x425c3f62,
%                      0xffeeddcc, 0x12345678, 0xaaaabbbb, 0x00ff1111);
%    $I = abs($i);
%    if ($I < count($S))
%    {
%      $Ku = AES($Ks, $S[$I]);
%      $Ru = $Au ^ $Ku;
%      $Bs = (int)SHA($Ru);
%      if ($i > 0)
%        $Rs = AES($Ks, $S[0]);
%      else
%        $Rs = Random(32);
%      $Ps = $Rs ^ $Ku;
%      $Qs = (int)SHA($Rs);
%    }
%    else
%      $Bs = $Qs = $Ps = 0;
%    $Ku = $Ru = $Rs = 0; // Forget all temporary values.
%  }
%?>
%\end{lstlisting}
%\lstset{language=PHP}
%\begin{lstlisting}[frame=lines,float=tbhp,label=crypto,title=crypto.php,caption=Crypto functions]
%<?php
%  function Random($bits)
%  {
%    return mt_rand();  // Will give 32-bits random regardless of $bits.
%  }
%  function AES($v, $k)
%  {
%    return $v ^ $k;  // Secretly use XOR for this.
%  }
%  function SHA($s)
%  {
%    return intval("0x" . substr(hash('sha256', $s), -8), 0);
%  }
%?>
%
%\end{lstlisting}
