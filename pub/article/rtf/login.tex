%
% $Header: /usr/local/src/slwp/article/tex/login.tex,v 1.9 2014/02/01 13:15:42 timo Exp $
%
% AUTHOR:	ing. T.M.C. Ruiter
%
\section{Logging in}
\label{logging_in}
Having unencrypted its keyring, selected a key number ($k$),
and given its means of identification,
the login process can commence.
\subsection{User Identification}
\label{sec:user_ids}
Instead of sending a traditional alphanumeric userid,
we send a combination of hashes.
This value will be used by the webserver to identify the user,
and the webserver will only store hashes in its user database.
This value is specially crafted,
so guessing will be hard.
%\subsubsection{Identication value}
\par
The user will still need something as a means of identification.
The most basic form of this is to give a name (real or imaginary);
in this case in the form of a string of letters.
Another form may include biometric features,
like an iris scan,
a fingerprint,
or hand geometry.
Or,
in contrast,
a hardware token may provide a certain value.
\par
The identification value is chosen once per website and can remain the same as long as the keyring exists.
There are no restrictions or constraints as to the contents of this identification value,
apart from it to be empty.
This means that it may be the same and reused for each and every website.
\par
The hash that is sent to the website to identify the user,
will be constructed using a combination of the keyring identifyer $Z[0]$ and the identification value ($\mathtt{ID}$),
and is constructed as follows.
Compute the \SHA\ hash of the keyring identifier and the ID:
\[H_0=\F{\SHA}{Z[0]}\]
\[H_1=\F{\SHA}{\mathtt{ID}}\]
Then, compute the final 256-bit value $U_h$ by means of an \XOR\ operation:
\[U_h=\xor{H_0}{H_1}\]
\par
This hash value is a combination of something the user has,
or has access to
(the keyring)
and something the user knows
(its chosen name)
or uniquely is
(some biometric value)
or phisically posesses
(a hardware token).
This makes it hard to match a stolen set of hashes from a webserver with a set of keyrings from a keyring server.

%%%%%%%%%%%%%%%%%%%%%%%%%%%%%%%%%%%%%%%%%%%%%%%%%%%%%%%%%%%%%%%%%%%
\subsection{Applying for an account}
\label{sec:applying}
See section~\vref{scheme:new_account} for an accompanying flow diagram.
\subsubsection{Step 1: Excuse me...}
\label{sec:apply_step1}
When a new user applies for an account,
apart from any personal details the website is interested in,
it presents the webserver with its hash value $U_h$.
Furthermore,
the user should select the index ($d$) of a hitherto unused key and use $n$ bits of it,
at the discretion of,
and indicated by,
the webserver.
\[K_d=\mod{Z[d]}{2^n}\]
Although this is a dummy key that is not used for logins
(it will even be replaced after a successful application for an account)
it still needs to be secured,
since it is used in the application process.
Eventually,
this key needs to find its way to the login server,
so we will encrypt it with its public encryption key $K_{le}$:
\[K_h=\Enc{K_d}{K_{le}}\]
The user will then send $U_h$, $K_h$, and its age, shoe size, and gender to the webserver.

\subsubsection{Step 2: Howdy partner!}
\label{sec:apply_step2}
The users' particulars and its identification hash value $U_h$ are relayed to the accounts server,
which will create the account.
The webserver will request values for a new user from the login server,
by forwarding the dummy key ($K_h$),
and indicating how long the keys should be
($n$ bits long).

\subsubsection{Step 3: What have we here?}
\label{sec:apply_step3}
The login server will do the following:
\begin{enumerate}
\item Generate a $n$-bit pseudo-random site key $K_s$ for the user,
which is then encrypted with the public key $K_{ae}$ of the accounts server:
\[K_s=\F{Random}{n}\]
\[K_y=\Enc{K_s}{K_{ae}}\]
\item Encrypt the $K_s$ with the current Secret key $S[0]$ to yield the new user key $K_u$.
\[K_u=\aes{K_s}{S[0]}\]
\item The (2048 bit) $K_h$ value is decrypted with the private key $K_{LE}$ to yield $K_d$:
\[K_d=\Dec{K_h}{K_{LE}}\]
which is subsequently used to encrypt $K_u$:
\[K_x=\xor{K_u}{K_d}\]
Both values $K_x$ and $K_y$ are returned to the webserver.
\end{enumerate}

\subsubsection{Step 4: Here are the results}
\label{sec:apply_step4}
The webserver will then relay the new encrypted key $K_y$ and $U_h$ to the accounts server.
The accounts server will decrypt $K_y$ with its private key $K_{AE}$ to get the site key,
but encrypt it again on the spot with the public key of the login server:
\[K_s=\Dec{K_y}{K_{AE}}\]
\[K_a=\Enc{K_s}{K_{le}}\]
and update the new account with it.
These encrypted keys can be sent to the login server by the webserver during logins,
without further processing.

\subsubsection{Step 5: Thanks mate!}
\label{sec:apply_step5}
The encrypted user key $K_x$ is presented to the user,
so it can store it in its keyring.
The user can decrypt its new key with by using $K_d$ it used originally:
\[K_u=\xor{K_x}{K_d}\]
Finally, $K_u$ is imported in the keyring with something like
\[Z[d]=\xor{(\F{Random}{128-n}\ \mathbf{<<}\ n)}{K_u}\]
The dummy key $K_d$ at $Z[d]$ is overwritten with 128 random bits,
of which the last $n$ bits contain the new key $K_u$.

\subsection{Login scheme}
\label{sec:full_login}
The procedure described below to login is the full login scheme.
See sections~\vref{scheme:simplest_login},
\vref{scheme:login_with_old_key},
and \vref{scheme:login_with_wrong_key} for explanatory flow diagrams.

%%%%%%%%%%%%%%%%%%%%%%%%%%%%%%%%%%%%%%%%%%%%%%%%%%%%%%%%%%%%%%%%%%%
\subsubsection{Step 1: Knock, knock\ldots}
\label{sec:login_step1}
The user's login program has to compute the folowing:
\begin{enumerate}
\item The userid hash $U_h$ (see section \ref{sec:user_ids}).
\item An $n$-bit pseudo-random number $R_u$:
\[R_u=\F{Random}{n}\]
\item The user key $K_u$ is taken from the keyring $Z$ at index $k$.
As this is 128-bit value,
take the least significant $n$ bits of it.
The pseudo-random number is encrypted with it to get the value $A_u$:
\[K_u=\mod{Z[k]}{2^n}\]
\[A_u=\xor{R_u}{K_u}\]
\item The webserver will return a hash value of the user's pseudo-random value.
To be able to verify this hash,
we compute our own hash $B_u$ to verify it with:
\[B_u=\sha{R_u}\]
\end{enumerate}
The special userid hash $U_h$ and the encrypted pseudo-random number $A_u$ are sent to the webserver.

%%%%%%%%%%%%%%%%%%%%%%%%%%%%%%%%%%%%%%%%%%%%%%%%%%%%%%%%%%%%%%%%%%%
\subsubsection{Step 2: Site key lookup and key matching attempts}
\label{sec:login_step2}
The webserver runs a login procedure,
see Algorithm~\vref{alg:webserver_login}.
\par
After receiving $U_h$ from the user, the encrypted site key $K_a$ needs to be looked up.
A query with $U_h$ is sent by the webserver to the accounts server.
If a match is found, $K_a$ is returned;
otherwise, $K_a$ is set to zero, indicating that no such record exists.
In the latter case, the values $B_s$ and $P_s$ are both set to zero as well, and returned to the user.
The user needs to rethink its actions and start over.
\par
Along with $K_a$ the account status $s$ may also be returned.
This status may indicate whether we want to allow the user to login or not.
If something is required from the user
(payment, or otherwise)
we might want to expire the account until the requirement is met.
We use the value $\mathtt{MAX\_ACTIVE\_KEYS}$ to limit the number of keys we want to try.
\par
If $\mathtt{MAX\_ACTIVE\_KEYS}$ is set to 1 then accounts will expire immetiately when a new Secret key is inserted.
When set to 3,
there will be a grace period of 2 times the Secret keys replacement interval.
This means that login is granted for this time,
in which the user can fulfill its debts.
If that does not happen,
the account will expire.
\par
Users that have not logged in for a while,
but have payed their monthly fees,
can still login without problems because all Secret keys will be tried for such logins.
\par
Now,
an iterative process starts,
trying to find the right Secret key to log the user in.
The webserver will send five values to the login server:
$A_u$;
$K_a$;
the number of bits in the user and site keys ($n$);
a boolean value indicating if the user is capable of performing a verify operation using a public signing key ($v$);
and a Secret key index $i$.
This will be repeated
(increasing index $i$),
until a match is found,
or no more keys can or will be tried.
\par
Finally, we could consider reducing the number of login attempts.
Instead of always starting with index $0$ for Secret key $S[0]$,
the date of the last login of the user can be taken into account.
If the query with $U_h$ as key,
that is sent to the accounts database,
would also return the last login date,
a starting key index $i$ could be computed.

%%%%%%%%%%%%%%%%%%%%%%%%%%%%%%%%%%%%%%%%%%%%%%%%%%%%%%%%%%%%%%%%%%%
\subsubsection{Step 3: Login server actions}
\label{sec:login_step3}
The login server is a processor of login values and calls a function of the HSM to compute them
(see Algorithm~\vref{alg:hsm}).
\par
In case $i$ is less than $m$
(the number of stored Secret keys, see section~\vref{sec:secret_keys})
the login server calculates the following, all within the HSM:
\begin{enumerate}
\item Decrypt the site key, using the private encryption key $K_{LE}$:
\[K_s=\mod{\Dec{K_a}{K_{LE}}}{2^n}\]
\item Temporarily, regenerate a user key, using the same algorithm as when the key was originally generated:
\[K_u=\aes{K_s}{S[i]}\]
with $S[i]$ the $i$-th Secret key stored in the HSM.
\item With the user key $K_u$, decrypt the pseudo-random the user has sent:
\[R_u=\xor{A_u}{K_u}\]
\item Calculate a hash with which the user can verify we own the site key $K_s$ and a corresponding Secret key $S[i]$:
\[B_s=\sha{R_u}\]
\item If the user is able and wants to verify the signature of the login server,
then the value $B_s$ will be replaced with a signature thereof with the signing key $K_{LS}$:
\[B_s=\Sgn{B_s}{K_{LS}}\]
We use and send only the signature, not the $B_s$ value also.
\item To be able to verify that the user owns and knows its user key $K_u$,
is not sending a replay of some earlier successfull login sequence,
and will be unable to ly about the correctness of the hash of the pseudo-random the webserver will send,
we calculate a challenge for the user.
\par
If the index $i$ equals zero, generate a pseudo-random value
\[R_s=\F{Random}{n}\]
otherwise, compute
\[R_s=\aes{K_s}{S[0]}\]
which will be the new key for the user.
\par
We then encrypt $R_s$ to a value $P_s$ the user can decrypt using its key:
\[P_s=\xor{R_s}{K_u}\]
This value will be different for each time the loop is executed,
even if the same new key is transmitted.
This is because $K_u$ will change each time,
as it is the encryption of $K_s$ with a different secret key $S[i]$.
\item We now know both random values.
If the user knows them also
(by successfully decrypting $P_s$)
it can prove this by sending the \XOR\ of both values:
\[Q_s=\xor{R_s}{R_u}\]
This is the value that the webserver should compare.
\end{enumerate}
The HSM will produce the values $B_s$,
$P_s$,
and $Q_s$ as a result of the calculations and the login server will send them back to the webserver,
as an answer to the five values it was given ($A_u$, $K_a$, $n$, $v$, and $i$).
The HSM will delete all temporary values from memory directly afterwards, and remember only its Secret keys.
\par
Should index $i$ equal $m$, then the array of Secret keys is exhausted.
If we reach this situation, we cannot log the user in since all possible attempts have failed.
Return zero values for $B_s$, $P_s$, and $Q_s$ to indicate this.

%------------------------------------------------------------------
\subsubsection{Step 4: User verifies site key}
\label{sec:login_step4}
The webserver sends $B_s$ and $P_s$ to the user.
If both are zero, the login has failed and the user should return to step 1.
It is advisable for the user to choose different values for the next try.
\par
If $B_s$ is nonzero, the user verifies whether this value matches with $B_u$.
\par
If the user is not able to verify a digital signature,
just compare $B_u$ and $B_s$.
Otherwise,
it should try to verify the combination of $B_u$
(the hash over $R_u$)
and the signature thereof,
which is in $B_s$:
\[\Ver{B_u+B_s}{K_{ls}}\]
\par
If the two values do not match,
the user either has selected the wrong key or uses an old key.
Logging in with an old key every now and then is inherent in this scheme,
as most Secret keys will be changed at regular intervals.
\par
To indicate that no match has been found, we send
\[Q_u=\mathtt{0x0}\]
($n$ zeroes) to the webserver.
The webserver will need to start over, and send values $A_u$, $K_a$, $n$, $v$, and $i+1$ to the login server.
So, back to step 3.
\par
After two unsuccesful attempts the user may be presented a question
whether it likes to abort the login procedure or continue trying with this key.
To abort, use:
\[Q_u=\mathtt{0x1}\]
and send this to the webserver
(instead of $\mathtt{0x0}$).
We return to step 1.

%------------------------------------------------------------------
\subsubsection{Step 5: Site verifies user key}
\label{sec:login_step5}
If $B_s$ matches $B_u$, then the webserver has found a Secret key for the user.
The user can now prove it has control over its user key by computing $R_s$ and $Q_u$:
\[R_s=\xor{P_s}{K_u}\]
\[Q_u=\xor{R_s}{R_u}\]
which is sent to the webserver as proof.
If the webserver accepts $Q_u$ as correct it will log the user in.
\par
If the webserver receives a value $Q_u$ that does \emph{not} match,
something fishy is going on.
Further attempts for this account, or from that source should be scrutinized.

%------------------------------------------------------------------
\subsubsection{Step 6: User key replacement}
\label{sec:login_step6}
\label{sec:conditional_key_replacement}
If this attempt to login was not the first with this key,
the webserver may have sent the user a new key in $R_s$ (instead of a pseudo-random value).
The user should store this new key in the keyring, overwriting the old key the user has just used:
\[Z[k]=\xor{(\F{Random}{128-n}\ \mathbf{<<}\ n)}{R_s}\]
where all bits of $Z[k]$ are replaced.
\par
For most websites,
restoring user keys that refer to the most recent Secret key $S[0]$ will be done without hesitation.
For some,
refreshing keys will be done only when certain conditions are met,
like payment of monthly fees,
a certain number of reviews written,
or some amount of data uploaded.
Until then,
logging in with a valid
(but in this context a typically `old')
key is granted.
But if, for instance, payment is overdue, the key will expire and logging in is no longer possible.
\par
Instead of incrementing index $i$ in step 2,
the index is decremented.
The HSM will just be sending random values for $P_s$ in that case, for all values of index $i$.
To indicate to the user that no new key is sent,
the website will replace it with $A_u$
\[P_s = A_u\]
and return this value to the user.
In this case,
no keys should be decrypted or overwritten.
