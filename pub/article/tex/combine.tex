\section{Combinations and simplifications}
The solution presented in the previous chapters is the most elaborate solution possible.
All roles assigned to different servers and all values encrypted.
In some situations roles can be combined or encryption skipped,
at the cost of reduced security.
\subsection{Combining server roles}
There is no technical restriction whatsoever combining all server roles
(webserver, account server, login server---and even the client, for that matter)
into a single computer.
For security reasons it is advisable that those roles be split.
Yet costs or lower security requirements may incline people to decide to combine things.
\par
To protect the webserver from harm,
this solution aims at significantly reducing the data value for hackers.
The less valueable the data items,
the less likely they are stolen.
\subsubsection{Login and accounts server}
\label{sec:login+accounts}
If the login server is operated locally,
it could be combined with the accounts server,
both somewhere in the back office.
This server could use the HSM for storing the private key for the accounts database as well.
\subsubsection{Webserver and accounts server}
In this case, a hacker that owns this server can collect site keys,
because they are decrypted here.
This in itself does not help the hacker much,
but he or she can get its hands on unencrypted keys.
\subsubsection{Webserver and login server}
X
\subsubsection{Webserver, accounts server and login server}
Y

%%%%%%%%%%%%%%%%%%%%%%%%%%%%%%%%%%%%%%%%%%%%%%%%%%%%%%%%%%%%%%%%%%%
\subsection{Simplifications}
The login scheme
(starting with section \vref{sec:full_login})
is complete and offers the whole set of security measures you would want for a safe login scheme.
Several measures are taken to establish this level of security.
This section explores what happens when you leave them out.
\subsubsection{Storing plain site keys}
Instead of storing encrypted site keys in the accounts database
(step 4 of the application process,
see section~\vref{sec:apply_step4}),
site keys could be encrypted by the webserver just before sending them to the login server,
in step 2 of the login process (see section~\vref{sec:login_step2}).
Storing plain keys can be considered if the webserver and accounts server are combined
(see section~\vref{sec:login+accounts})
or when the key refactoring that is needed when the public key of the login server expires is found to be too cumbersome.
\par
Security penalty: low.
\subsubsection{Sending plain site keys}
Sending plain site keys does not require a certificate from the login server.
\par
Security penalty: medium.
\subsubsection{Using traditional userids}
Instead of using a value dependent of $Z[0]$,
you could stick to the old fashioned userid to indicate as which user you want to log in.
You will need a userid that is unique for the website.
What's more:
you will need a userid,
period.
\subsubsection{Getting rid of the HSM}
\begin{dialogue}
\speak{Jen}		\direct{Holding an invoice.}
				That's quite an expensive set of toys you have there, lads, your\ldots
				\direct{Reads.}
				H-S-Ms.
\speak{Moss}	Thank you for taking an interest!
				For a whole week I was wondering if you'd notice.
\speak{Jen}		Why on earth did you order two of those?
\speak{Moss}	One is none, I always say.
				\direct{Pauses.}
				I keep wondering why this sounds contradictory\ldots
\speak{Roy}		Could we not have three? We all would sleep better.
				\direct{\refer{Moss} nods affirmative.}
\speak{Jen}		Could we do without, instead?
				\direct{With a firm look,
				\refer{Jen} strides purposefuly towards the apparatus,
				with her her hand reaching for the power plug.}
\speak{Moss}	\direct{Looks up in horror.}
				Impossible!
\speak{Roy}		\direct{Lunges forward with a wild look to block \refer{Jen}'s path.}
				Over my dead body!
\speak{Jen}		\direct{Mildly disgusted.}
				So that's where you keep your porn, then?
\speak{Roy}		\direct{Rolls his eyes.}
				Yeah, you've got us.
\attrib{The IT Crowd (not).}
\end{dialogue}
Yes, you can do without.
But you will probably be hacked.
And hacked again,
most likely;
just for sport.
Your secret keys on your harddrive are a prize,
even if publishing them does not directly harm your security.
You have a lot of explaining to do when this happens for the second time.
The third time\ldots, oh, don't bother.
\par
The HSM is used for generating good quality random numbers.
Using some other pseudo-random generator will probably yield numbers less random,
which may be an attack vector.
\par
An HSM gives you physical and logical security,
at the cost of personnel.
The hardware costs are minimal compared to the people involved in securely handling your sensitive Secret keys.
But do it right,
and you are safe,
even if you are hacked.
\subsubsection{Eliminating Secret keys}
Instead of having different keys for website and user,
you may opt to store the user key in the website's user database.
Doing so will yield usable login data when the accounts of the website will be stolen.
\par
This is not recommended,
but not necessarily wrong.
Still,
no replay of logins can occur,
as there is still the use of random numbers for this.
\par
A possible bonus for this large omission in security could be that you don't need to store any Secret keys anywhere,
except the keys belonging to the users.
See if your users will put up with this for very long.
\subsubsection{Skipping website authentication}
A website may be offering user authentication only.
When Secret keys have been dismissed as well,
this could be considered the most rudimentary login scheme.
\par
In this case,
a user only sends its user id.
The website uses the corresponding user key to encrypt a random value.
All the user has to do is return the right hash value to log in.
\par
With the absence of the site verification,
the user loses the ability to verify that it is talking to the website that offered him its key originally.
It could be any other website,
possibly allowing him to login anyway and lure him to reveil sensitive or valuable information.
\par
Of course there are certificates that indicate the trustworthyness of a webserver,
but site authentication makes almost absolutely sure,
and you cannot annoyingly ``click this away'' like a warning for a bogus website certificate.
