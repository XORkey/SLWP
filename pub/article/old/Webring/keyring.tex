\section{Storing user keys in a keyring}
\label{sec:keyring}
Keys for the user are collected and stored in a keyring.
The keyring is a block of at least 100 random numbers with 128 bits, most of them dummy keys.

\subsection{Creating a keyring}
A keyring is created by using a random generator of sufficient quality,
which generates 100 random numbers of 128 bits.
These random numbers are then written to a file on a storage device.
Key number zero is used to identify the keyring and will never change after its creation.
The other 99 keys are dummies (random bits with no meaning whatsoever).
\par
Nothing else is stored in a keyring, as to minimize the information stored there.
Since all bits are entirely random%
---for dummies and real keys---%
a keyring stores no information whatsoever.

\subsection{Applying for a userid}
\label{sec:applying}
When a new user applies for an account, it should choose a userid%
---one that the website will accept as unique.
Furthermore, it should select the index of a hitherto unused key
(therefore, a dummy).
The user will send its $\E{SHA}$ hash of its userid,
along with its dummy key $K_d$ to the website.
\par
The website will send the dummy key $K_d$ to an SDPS hosting an HSM (see section \ref{sec:SDPS}),
which will do the following:
\begin{enumerate}
\item Generate a random
\[K_s=\mathtt{Random}()\]
which will be the new site key for this user.
\item Then it will encrypt the $K_s$ with the current Secret key $S[0]$ to yield the new user key $K_u$
\[K_u=\E{AES}(K_s, S[0])\]
\item Finally, the new user key is encrypted with the user's dummy key $K_d$
with an XOR operation:
\[K_x=K_u \oplus K_d\]
\end{enumerate}
The SDPS will then send the new key $K_s$ and the $\E{SHA}$ hash of the userid to the server hosting the accounts database.
It will send $K_x$ in a mail to the new user (optionally with the index the user supplied).
\par
The user's keyring program can import the new key $K_x$ into the keyring by calculating
\[K_u=K_x \oplus K_d\]
using the index it has originally selected for the slot in the keyring.
The dummy key $K_d$ is overwritten with the new key $K_u$.

\subsection{Encrypting a keyring}
Optionally but preferably,
the keyring on the storage device can be encrypted, to further enhance security.
The unencrypted values should never be written to a storage device and only be kept in volatile memory.
An encrypted keyring is of no value to a hacker without knowledge with which keys to decrypt it.
It is up to the user to remember this.
\par
Encrypted keyrings can be stored in a public place, for easy access, and act as a backup.
A dedicated webserver can be employed for storing encrypted keyrings.
Harvesting keyrings from these servers will not help hackers.
\par
The keyring can be encrypted by using the keys in the keyring itself.
The user may select any number of different random key numbers from its ring,
which are numbered from 1 to 99.
Suppose two keys are used.
Then,
prepending a zero for key numbers less than 10,
this yields a 4 digit PIN code for the keyring.
\par
Using the keys (in selected order) to encrypt the keyring, the values of the keys are obscured.
Select the first key from the keyring by specifying its index $j$.
For each key $K[i]$ (with $i$ running from 0 to 100), except $K[j]$, do
\[K[i]=K[i]\oplus K[j]\]
\par
To decrypt, reverse the order of the selected keys and do the same.
This will render the original keyring.

\subsection{Copying keys and keyrings}
A keyring can be freely copied to other devices, so all keys are available there as well.
The keys of one keyring can be copied to other keyrings, in different locations.
It is up to the user to keep a registration of this.
