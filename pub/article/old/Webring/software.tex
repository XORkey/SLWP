\section{Software}
Several little pieces of software are needed to make things happen.

\subsection{SDPS functions}
\label{sec:sdps_functions}
The SDPS (see section \ref{sec:SDPS}) needs to perform the following functions (in relatively random order):
\begin{itemize}
\item Since the SDPS hosts an HSM,
it could offer to generate a new keyring for a user
(just a blob of 100 random 128 bit numbers).
\item Generate a set of keys for a new user ($K_s$ and $K_x$,
see section \ref{sec:applying}) given a dummy key $K_d$ and a \E{SHA} hash of a userid.
\par
It should send the site key $K_s$ and the hash to the server that has all account information in a database using an insert query.
The encrypted user key $K_x$ should be sent back to the website, so it can transport it to the user.
\item Store a new Secret key,
given the new key and \texttt{MAX\_KEYS};
the latter may change,
due to a policy change.
\par
The new key should be stored in slot 0 of the array, by first moving all existing keys one position.
If there are already \texttt{MAX\_KEYS} Secret keys stored in the array,
all keys with an index greater or equal to \texttt{MAX\_KEYS} should be discarded.
\item Calculate the values $B_s$, $P_s$, and $Q_s$,
in response to the three values $A_u$, $K_s$,
and an index $i$ (see section \ref{sec:login_step3}).
\par The input values come from the website;
the output values should be returned.
\end{itemize}

\subsection{HSM}
The HSM in the SDPS should just present the functions needed for the SDPS,
and nothing more.
This way,
misuse can be prevented.
Anyway, it should be able to perform the functions described in section \ref{sec:sdps_functions}.

\subsection{Accounts Database}
The database that is used for holding all accounts should be able to return the site key $K_s$ in response to a \E{SHA} hash of a userid the user has sent (see section \ref{sec:login_step2}).

\subsection{Website}
The website should be able to do some basic tasks:
\begin{itemize}
\item Relay values received by the user to the accounts database and accept the site key $K_s$ (see section \ref{sec:login_step2}).
\item Iteratively send values $A_u$, $K_s$, and an index $i$, to the SDPS, and relay the results back to the user (see section \ref{sec:login_step3}).
\item Act according the replies of the user to find the right Secret key (see section \ref{sec:login_step4}).
\item Verify that the user has sent the right hash $Q_u$ and log the user in (see section \ref{sec:login_step5}).
\end{itemize}

\subsection{Browser}
The software for the browser has to do some diverse tasks:
\begin{itemize}
\item Present all keys in some graphical form, so the user can select one.
\item Compute the \E{SHA} hash of the userid,
generate a random value,
XOR this with the key selected by the user,
and send this to the website (see section \ref{sec:login_step1}).
\item Verify the \E{SHA} hashes the website will send (see section \ref{sec:login_step4}).
\item Decrypt the challenge from the website,
compute the \E{SHA} hash of the result,
and send it back (see section \ref{sec:login_step5}).
\item Optionally, replace keys on the keyring (see section \ref{sec:login_step6}).
\end{itemize}
%%%
