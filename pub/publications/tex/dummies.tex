%
\section{Secure login for dummies}
This piece of text is intended as an introduction to the most basic part of the protocol.
It does away with encryption of transmitted values, optional biometric values, key replacement, etcetera, to focus entirely on the core principle.
\subsection{Computer roles}
We distinguish three computer roles:
the website
(with an accounts database),
the user
(with a browser),
and a login server
(with a hardware security module or HSM which stores secret keys safely).
The account database holds the particulars of all users,
and,
instead of a userid and a password,
it holds a User ID hash
(called $U_h$)
and a site key
(called $K_s$).
\subsection{Keys}
There are four important keys:
the site key
(called $K_s$) which is held by the website,
the user key
(called $K_u$) held by the user on a key ring,
and the Secret Key
(called $S[0]$) held by the login server.
The user key $K_u$ is the site key $K_s$ encrypted with the secret key $S[0]$:
\[K_u=\aes{K_s}{S[0]}\]
\par
User keys are put on a digital key ring;
the user selects one of those keys to act as $K_u$.
This key ring can be seen as an array of keys,
of which the first is the key ring identifier.
This identifier is called $Z[0]$ henceforth and is the
fourth important key.
\par
For this introduction we assume all keys are static.
\subsection{User ID hash}
The login starts by the user sending the hash value $U_h$ to the website,
which is used to lookup the corresponding site key $K_s$.
If no match is found,
the login process halts immediately.
\par
The $U_h$ value is compiled from several values:
the key ring identifier $Z[0]$,
the domain name of the URL,
and a user id, freely chosen by the user.
All these value together provide a 256-bit hash value,
which is used to find the corresponding site key $K_s$ for this user.
\subsection{Logging in}
\subsubsection{Step 1}
To start the login sequence,
the user first compiles the $U_h$ value.
Then it calculates a second value, called $A_u$, as follows.
Choose one of the user keys $K_u$ from the key ring.
Generate a random number $R_u$ and combine the two with XOR:
\[A_u=\xor{R_u}{K_u}\]
We also need to calculate the value we like to hear from the website:
\[B_u=\sha{R_u}\]
which is the hash value of our random $R_u$.
\subsubsection{Step 2}
The website looks up $U_h$ and hopefully finds $K_s$.
If it does, the website sends $K_s$ and $A_u$ to the login server.
If not, we come to a full stop.
\subsubsection{Step 3}
The login server takes the $K_s$ value and temporarily turns this in to $K_u$:
\[K_u=\aes{K_s}{S[0]}\]
It then recovers the random the user has sent ($R_u$) by:
\[R_u=\xor{A_u}{K_u}\]
The user is expecting to get a hash of this random, so we calculate:
\[B_s=\xor{K_u}{(\sha{R_u})}\]
Furthermore, we will introduce a value which the user must respond to.
For this we generate a random value $R_s$.
The user can prove it owns the right user key $K_u$ if it is able to decrypt this random number $R_s$.
We therefore compute
\[P_s=\xor{R_s}{K_u}\]
so $R_s$ is easily recoverable by the user using XOR again.
The website needs to verify the results of the user's calculation,
so it is provided by the login server:
\[Q_s=\xor{R_s}{(\sha{R_u})}\]
The values $B_s$, $P_s$, and $Q_s$ are returned to the webserver.
\subsubsection{Step 4}
The website sends $B_s$ and $P_s$ to the user, and keeps $Q_s$ in memory.
The user first compares the value $B_s$ with its own value $B_u$.
If they are the same,\footnote{They may differ in the full version, as more secret keys may be used there.}
the website has the right key
(which proves the website is genuine).
\subsubsection{Step 5}
The user recovers the site random $R_s$ by computing
\[R_s=\xor{P_s}{K_u}\]
and calculates the answer for the website:
\[Q_u=\xor{R_s}{(\sha{R_u})}\]
which is returned to the website.
\par
The website compares values $Q_s$ and $Q_u$ to see if they are the same
(which proves the user is genuine).
If this is the case,\footnote{$Q_s$ and $Q_u$ may differ also, but in that case something fishy is going on.}
the user is logged in.
