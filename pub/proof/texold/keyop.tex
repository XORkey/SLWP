\Section{Key operations}
All server side operations that involve keys can be consolidated and separated from the comparison operations needed for the authentication processs.
The same goes for the client side, which makes that we have four separate places for computations.
That looks a bad idea at first sight, but it proves to be advantageous for security.
\par
We can divide the computations in two categories: simple comparisons and key operations.
\par
Maintaining and upholding good security needs expertise that does not come natural for many website owners and maintainers.
Even if security has full attention and focus, with a team of experts at the controls, it still is hard to keep things safe.
Simply because the site is almost directly connected to the wild wild west of the Internet; open for anyone, anytime.
\par
It is therefore not entirely unfortunate that we could separate all key operations away from the inherently vulnerable web site.
Instead, we outsource this to a back office, or preferrably remote, server, then we could move most keys along with it.
That server can be equipped with an HSM to store keys and compute values with it.
\Subsection{Login server}
The server that performs all key operations is called the login server in this authentication scheme.
It does not need to be a separate server at all; the web server may perform this task, if need be.
Having a dedicated server for this, however, is better for overall security because you can protect things better that way.
\Subsection{Password manager}
Although I like to use the phrase ``password'' as little as possible in this document for obvious reasons,
I fear that even when this protocol is adapted globally and ubiquitously,
any program that aides with logging in will be called a password manager,
despite its inaptness.
