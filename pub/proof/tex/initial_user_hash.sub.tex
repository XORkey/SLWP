\Subsection{Initial User hash}\label{initial_user_hash}
The user hash (denoted with \(H_u\)) serves as an improved replacement of the userid.
Instead of a guessable string
(an e-mail address or something derived from a person's name)
we send a 256-bit hash value.
\par
The ultimate purpose of this hash is to act as a search key into the accounts database of the website,
just as a userid would do in the traditional way of logging in.
The result of this search would be the site key that is used in the login computations.
\Subsubsection{Hash seed}
The user has a key ring, with keys of each website it can log in.
We could use a hash of the key
(e.g. at index 23 in the key ring)
of the website to the \(H_u\) identifier:
\[H_0 = \HSHA{Z[23]}\]
but that would be wrong for two reasons:
the key can possibly be recovered using brute force,
and we want to be able to refresh the keys every now and then, which would imply that \(H_u\) changes also.
\par
However, we could choose \(Z[0]\)
(i.e. the first key in the ring)
for this purpose.
We have to declare it `unchangeable' but that is no problem.
At the same time it can be the identifier of the key ring, and can be thought as bound the the individual that owns the key ring.
So we compute it like this:
\[H_0 = \HSHA{Z[0]}\]
\par
The presented means of computing \(H_u\) yields a value that is totally unguessable (with \(2^{256}\) possible values), but the same for each login.
Tracking users accross websites is easy when they use the same e-mail address as their userid.
When using a hash value like above, the same can be done, with no extra effort.
\par
To remedy this, we could add something specific for each website.
The host name part of the URL
(a Fully Qualified Domain Name)
is an perfect candidate for this:
\begin{equation}
H_0 = \HSHA{Z[0] + \mathtt{FQDN}}
\end{equation}
\begin{equation}
H_u = H_0
\end{equation}
We now have a unique \(H_u\) value per website, but it is still a single factor login.
Just the key ring is required to be able to login.
\Subsubsection{Adding more factors}
As the \(H_u\) value is a 256-bit hash, we can add more factors with great ease.
If each factor is presented as another 256-bit value, each of them can be added to the seed factor with the \XOR\ operator.
The number of factors is unlimited this way.
We will discuss this in \ref{full_user_hash}.
