\Subsection{User hash \(H_u\)}
The user hash (\(H_u\)) serves as an improved replacement of the userid.
Instead of a guessable string
(an e-mail address or something derived from a person's name)
we send a 256-bit hash value.
\Subsubsection{Initial development of \(H_u\)}
The user has a key ring, with keys of each website it can log in.
We could use a hash of the key
(e.g. at index 23 in the key ring)
of the website to the \(H_u\) identifier:
\[H_0 = \HSHA{Z[23]}\]
\[H_u = H_0\]
but that would be wrong for two reasons:
the key can possibly be recovered using brute force,
and we want to be able to refresh the keys every now and then, which would imply that \(H_u\) changes also.
\par
However, we could choose \(Z[0]\)
(i.e. the first key in the ring)
for this purpose.
We have to declare it `unchangeable' but that is no problem.
At the same time it can be the identifier of the key ring, and can be thought as bound the the individual that owns the key ring.
So we compute it like this:
\[H_0 = \HSHA{Z[0]}\]
\[H_u = H_0\]
\par
The presented means of computing \(H_u\) yields a value that is totally unguessable (with \(2^{256}\) possible values), but the same for each login.
Tracking users accross websites is easy when they use the same e-mail address as their userid.
When using a hash value like above, the same can be done, with no extra effort.
\par
To remedy this, we could add something specific for each website.
The host name part of the URL
(a Fully Qualified Domain Name)
is an perfect candidate for this:
\[H_0 = \HSHA{Z[0] + \mathtt{FQDN}}\]
\[H_u = H_0\]
We now have a unique \(H_u\) value per website, but it is still a single factor login.
Just the key ring is required to be able to login.
\Subsubsection{Adding a userid}
With the initial way of computing \(H_u\), it is easy to add a factor.
We could add a hash of something the user knows: a user identifier, as already used for traditional logins.
The value \(\HSHA{\mathtt{userid}}\) should do the trick.
\par
The userid may be the same for the login for each website;
it may also differ for each but that leaves the user with the task of remembering them.
The userid is NOT a password, because it is not stored on the server in any way, and it never changes.
I could use just my name `Timo Ruiter' or my initials `TMCR', or, to make it a bit harder (or more fun)
use my Japanese, Greek, or 17th century Dutch alter egos `Timoto Takahashi', `Martin Timodopoulos', or `Tinus Cornelisz'. 
\par
If we combine the two, using \XOR:
\[H_0 = \HSHA{Z[0] + \mathtt{FQDN}}\]
\[H_1 = \HSHA{\mathtt{userid}}\]
\[H_u = \xor{H_0}{H_1}\]
we now have something that is unique and multi-factor.
It has two components: something the user knows (the userid) and something the user has (the key ring).
\Subsubsection{Biometric data}
We can increase complexity of the \(H_u\) value,
e.g. by using biometric data instead of the userid.
\begin{dialogue}
\speak{Walter} Over the line, Smokey!  I'm sorry.  That's a foul.
\speak{Smokey} Bullshit.  Eight, Dude.
\speak{Walter} Excuse me!  Mark it zero.  Next frame.
\speak{Smokey} Bullshit, Walter!
\speak{Walter} This is not Nam.  This is bowling.  There are rules.
\speak{Dude} Come on Walter, it's just--it's Smokey.  So his toe slipped over a little, it's just a game.
\speak{Walter} This is a league game.  This determines who enters the next round-robin, am I wrong?
\speak{Smokey} Yeah, but--
\speak{Walter} Am I wrong!?
\speak{Smokey} Yeah, but I wasn't over.  Gimme the marker, Dude,  I'm marking it an eight.
\par\medskip\direct{Walter takes out a pistol.}
\speak{Walter} Smokey my friend, you're entering a world of pain.
\attrib{The Big Lebowski, 1998}
\end{dialogue}
We set \(H_1\) to zero to eliminate it (\(\xor{a}{0} = a\)),
\[H_1 = 0\]
and use the output of a fingerprint reader.
We then compute (in some way):
\[H_2 = \HSHA{\mathtt{fingerprint}}\]
as a third factor.
\par
We could add a voice recording:
\begin{dialogue}
\speak{Edna}	\direct{Into microphone} Edna Mode.
\par\medskip\direct{Laser gun comes out from the ceiling and points at Helen, who gasps.}
\speak{Edna} And guest.
\par\medskip\direct{The laser gun retracts}
\attrib{Incredibles 2, 2018}
\end{dialogue}
and compute (in some way):
\[H_3 = \HSHA{\text{``Edna\ Mode\ldots\ and\ guest''}}\]
or using a value supplied by a hardware token:
\[H_4 = \HSHA{\mathtt{token\ value}}\]
or still some other value:
\[H_5 = \HSHA{\cdots}\]
and combine them generally with:
\[H_u = \bigoplus_{i=0}^n H_i\]
(meaning \(H_u = \xor{\xor{\xor{\xor{H_0}{H_1}}{H_2}}{\cdots}}{H_n}\))
as a genuine multifactor value.
\par
Using a hardware token that generates time based keys leads to ever changing \(H_u\) values.
The way \(H_u\) is computed makes it easy to strip off the temporary value on the server side:
\[{H_4}^{*} = \HSHA{\mathtt{token\ value}}\]
\[{H_u}^{*} = \xor{H_u}{{H_4}^{*}}\]
where this time the token value is generated by the server side component of the hardware token software.
We can use \({H_u}^{*}\) as the key for account lookup.
\Subsubsection{Implications of a multi-factor \(H_u\)}
When using multiple factors, especially biometric ones, you need equipment that can provide them.
We login using smartphones, and most have a fingerprint reader of some kind.
With such a device it is fairly easy to use a fingerprint as a login factor.
\par
Switching to a desktop computer, and trying to login to the same site may prove difficult without an external fingerprint reader.
On this computer, a hardware token may be available instead, using a USB interface, for example.
Smartphones lack such an interface, for obvious reasons.
\par
It may be necessary to have more than one \(H_u\) value associated with a single account.
This would allow the user to indicate its account by more than one multi-factor way.
