\Section{A new way of logging in}
The \TIMO\ protocol has some main features that set it apart from other authentication protocols.
We start here with the seeding ideas and deal with improving it later.
%%%
\Subsection{Basic ideas}
A compromised account can be devastating.
It is worth a considerable amount of money when you have a database full of usernames and their passwords, so hackers are always looking for ways to get them.
I was thinking: 'What if there were no passwords (or equivalents) to be found?'
That thought ultimately lead to this authentication protocol, which reiies more on computation than on human's memorizing capabilities.
\par
Another thing bothering me was the use of a string of very few characters to identify a single person out of millions.
A traditional `userid' is typically either a short string of lowercase letters and numbers
(e.g. jessica77)
or an email address
(jessica.chastain@hotmail.com).
But this does not uniquely identify anyone, and anyone can misuse this for illicit purposes.
\par
As one thing leads to another, the following ideas emerged simultaneously as a complementary set.
%%%
\Subsubsection{Eliminating passwords}
If you do away with passwords altogether, what are you left with?
You still need to be able to authenticate yourself, otherwise you cannot be granted access to your personal account.
That implies that the website needs to store something to facilitate this basic process.
However, it must not be something useable in and of itself; it has to have a counterpart that is in the posession of the individual user.
This counterpart must be different but related, like the two parts of Annie's locket.
\begin{dialogue}
\speak{Annie}
No! No please don't make me take my locket off. I don't want a new one.
\direct{Fingers her locket.}
This locket, my Mom and Dad left me when... when they left me at the Orphanage.
\attrib{Annie, 1982}
\end{dialogue}
\par
The notion of having different keys for user and server is of paramount importance.
It changes the server from a treasure trove (millions of passwords) to a dry desert (millions of useless keys).
\par
The website has a unique key for each user, which do not change in principle.
The user has a key that is derived from the key the webserver has for that user.
The user keys are derived from the site keys by encrypting (it with another and altogether different key).
\par
If we imagine a random 128-bit value and its \AES\ encryption, we could store that random value as the key for a user in the website's database and give the user its encrypted counterpart.
With these keys we can design a protocol with which the user can prove it has the right key.
This way, what is stored on the website's disks is no longer usable for logging in, as the key to encrypt them is nowhere to be found.
As it turns out, the user can also verify the validity of the website (separate from any PKI scheme) with the same protocol.
\par
The protocol that initially emerged from my ponderings is discussed  in §\ref{rudimentary-login}.
%%%
\Subsubsection{The key ring}
Passwords you can remember; keys you can't\footnote{Well, that is true for most people.
Some are able to remember and recite well over 100.000 digits of \(\pi\) \cite{pi-world-ranking-list}, so what are a few 16 byte keys anyway?}.
The website uses a database to store its site keys;
the user needs something similar.
The user gets a key from every website it will login to, if it supports this protocol.
Before you know it, you have over 50 keys.
\par
In this scheme, all keys will be stored on a digital key ring, which is just a set of 128-bit random numbers.
Each key is associated with a website and automatically selected.
The key ring itself is identified by its first key (key zero).
Instead of calculating login values with it, it is used for the computation of the user hash.
\par
The key ring can be put on the user's harddisk as a blob of data.
It could be stored on a special USB device with fingerprint protection.
It could be stored in the Cloud.
\par
It is up to the user to select the right protection for their keys.

%%%
\Subsubsection{The user hash}
\begin{dialogue}
\speak{Centurion}	Where is Brian of Nazareth? I have an order for his release.
\speak{Mr. Cheeky}	Uh, I am Brian of Nasareth!
\speak{Brian}		What?
\speak{Mr. Cheeky}	Yeah, I - I - I'm Brian of Nazareth.
\speak{Centurion}	Take him down!
\speak{Brian}	I'm Brian of Nazareth!
\speak{Victim \#1}	Eh, I'm Brian!
\speak{Mr. Big Nose}	I'm Brian!
\speak{Victim \#2}	Look, I'm Brian!
\speak{Brian}		I'm Brian!
\speak{Victims}		I'm Brian!
\speak{Gregory}		I'm Brian, and so's my wife!
\attrib{Monty Python's ``Life of Brian'', 1979}
\end{dialogue}
There are several things that make traditional userid's cumbersome to use.
\begin{itemize}
\item Anyone can claim to be `Jessica' and try to login on her behalf.
If her password has been compromised or guessed,
the real Jessica may either have her privacy violated or her reputation destroyed.
If `she' has committed a crime, Jessica has a hard time proving it was not her.
\item They are in short supply (there are many Jessicas in the world, all (perhaps narcistically) preferring to use just 'jessica').
\item Each website has its own set of userids, which may make `jessica77' available on one site, but not on the other.
This leaves Jessica with many different userids, each of which she must remember, or write down.
\item Especially email userids can be used to link accounts between different websites.
Since email addresses are fairly unique, your browsing behaviour may be tracked through logins.
\end{itemize}
Since we are letting the computer join and aid the login process, why not compute some more and come up with a large, fairly random, value that is used to identify a person.
One that is unique; not per person (like mail accounts), but to an account per user per webserver.
\par
Instead of sending a legible string, why not send something that is not relatable to a human: a 256 bit hash, for example.
This hash could be made up of many things, and it has always been the task of a computer to match userids to accounts, so a hash value would do the trick just as well.
\par
The idea is that you cannot guess this value
(only Jessica knows how to generate one)
and that it may be compiled by using several sources,
so this authentication protocol becomes genuinally multi-factor.
\par
The creation of this hash is briefly touched in \ref{initial_user_hash} and described fully later in \ref{full_user_hash}.
%%%
\Subsubsection{Login server}
Having a protocol that eliminates passwords is great, but leaves you with a big challenge: instead of passwords you now need to secure your keys.
Your website stores site keys in its database, but they are of no real value by design.
This, howeer, is not the case for the key that turns site keys into user keys.
\par
This key can be put inside an Hardware Security Module (HSM) so nobody can really get to it.
However, if the HSM is accessible directly from the website's servers, you can probably use the key anyway.
\par
As it turnes out, all cryptographical operations of the protocol can be singled out and combined into what is called a Login server.
This server is separate from the web server and can be placed in the back office, or somewhere on the Internet.
Now, the protcol is performed by two servers instead of one, and the changes needed on the webserver to implement the protocol minimized.
\par
Being able to separate functions into a login server gives many benefits.
This will be discussed in \ref{login_server_benefits}.
\par

This also means that operating the Login server could be done by people other than those running the website traditionally
(which can keep their job and do their thing without change).
In fact, the whole business running Login servers can be put on the Internet as LaaS: Login as a Service.
\par
This service can be protected and secured by professionals that know nothing about websites,
but all the more from securing Internet traffic,
protecting keys in HSMs,
do key rollovers,
organize key ceremonies,
generate certificates,
and everything else you need to make the Login servers little bastions of security.

Designing and maintainig a website, especially large ones, is a specialist's job.
Interoperability with payment systems requires specialists too.
The same goes for the databases running the show.
\par
But securing keys and operating them is as much a specialist's job as any mentioned before, maybe more so.
This is not something you do on the side and hope everything will be all right.
Typically, none of all the specialists you have working on your website will have indepth knowledge of this field of expertise.

\Subsection{Initial User hash}\label{initial_user_hash}
The user hash (denoted with \(H_u\)) serves as an improved replacement of the userid.
Instead of a guessable string
(an e-mail address or something derived from a person's name)
we send a 256-bit hash value.
\par
The ultimate purpose of this hash is to act as a search key into the accounts database of the website,
just as a userid would do in the traditional way of logging in.
The result of this search would be the site key that is used in the login computations.
\Subsubsection{Hash seed}
The user has a key ring, with keys of each website it can log in.
We could use a hash of the key
(e.g. at index 23 in the key ring)
of the website to the \(H_u\) identifier:
\[H_0 = \HSHA{Z[23]}\]
but that would be wrong for two reasons:
the key can possibly be recovered using brute force,
and we want to be able to refresh the keys every now and then, which would imply that \(H_u\) changes also.
\par
However, we could choose \(Z[0]\)
(i.e. the first key in the ring)
for this purpose.
We have to declare it `unchangeable' but that is no problem.
At the same time it can be the identifier of the key ring, and can be thought as bound the the individual that owns the key ring.
So we compute it like this:
\[H_0 = \HSHA{Z[0]}\]
\par
The presented means of computing \(H_u\) yields a value that is totally unguessable (with \(2^{256}\) possible values), but the same for each login.
Tracking users accross websites is easy when they use the same e-mail address as their userid.
When using a hash value like above, the same can be done, with no extra effort.
\par
To remedy this, we could add something specific for each website.
The host name part of the URL
(a Fully Qualified Domain Name)
is an perfect candidate for this:
\begin{equation}
H_0 = \HSHA{Z[0] + \mathtt{FQDN}}
\end{equation}
\begin{equation}
H_u = H_0
\end{equation}
We now have a unique \(H_u\) value per website, but it is still a single factor login.
Just the key ring is required to be able to login.
\Subsubsection{Adding more factors}
As the \(H_u\) value is a 256-bit hash, we can add more factors with great ease.
If each factor is presented as another 256-bit value, each of them can be added to the seed factor with the \XOR\ operator.
The number of factors is unlimited this way.
We will discuss this in \ref{full_user_hash}.


%%%%%%%%%%%%%%%%%%%%%%
\Subsection{A rudimentary login algorithm}\label{rudimentary-login}
The core idea of the algorithm is that the user must prove it is in the posession of a key that the server has issued.
To do so, there is an almost limitless number of approaches to pursue, but not all will be fruitful.
\begin{dialogue}
\speak{Dr. Strange}	I went forward in time, to view alternate futures. To see all the possible outcomes of the coming conflict.
\speak{Star-Lord}   How many did you see?
\speak{Dr. Strange}	14,000,605.
\speak{Iron Man}	How many did we win?
\speak{Dr. Strange}	One.
\attrib{Avengers: Infinity War, 2018}
\end{dialogue}
Without the Time Stone at my disposal,
it took quite some time to traverse all possible paths a protocol may take to come up with something that works.
\par
In the solution we are disussing now, the server encrypts a random number (a nonce) with the key the user has.
The user can prove it has that key by decrypting the nonce and returning a value derived from it, like a hash for instance.
\par
But, we do not want the web server to have all the user keys, or any other directly usable value, for that matter.
This is the main inherent flaw of authentication using passwords.
Instead, we make the user key a direct derivative of the site key.
\par
There are several ways to achieve this; we do it with a calculation similar to
\[K_u = \EAES{K_s}{S}\]
The key the user has (\(K_u\)) is the \AES\ encrypted value (with some key \(S\)) of the key the website has (\(K_s\)).
More of this later.
\par
For symmetry, we ask the webserver to authenticate itself also, by similar means: decrypt the client's random value and send proof you did that with the right key.
\par
What follows is a step by step unrolling of the protocol, where each calculation is described.
%%%
\Subsubsection{Key length \(n\) indicated by the server}
At the start of the login protocol the webserver indicates the key lenght \(n\) to the user.
\[n\quad\Bigl\lvert\thickspace\text{with}\thickspace 1\le{n}\le128\]
This is the length of the keys we are going to use.
Not always the maximum is chosen; a value of 32 bit is a valid and convenient choice also.
%%%
\Subsubsection{Computing challenge \(A_u\) by the user}
We need one of our keys, stored on the key ring, to help us with the calculations.
A lookup is done to get the key for the website.
Suppose we find it at index 47 and compute:
\begin{equation}\label{rud:uku}
K_u = \mtpn{Z[47]}
\end{equation}
where the leftmost (most significant) $128 - n$ bits are stripped with the modulo operation.
This yields a key of precisely $n$ bits, as instigated by the website.
\par
Then, the user generates a random number, also of \(n\) bits, and encrypts it with the key it has for that website:
\begin{equation}\label{rud:uru}
R_u = random(n)
\end{equation}
\begin{equation}\label{rud:au}
A_u = \xor{R_u}{K_u}
\end{equation}
\par
Which makes this an \(n\)-bit value.
We use \(A_u\) as a challenge for the website: please decrypt it and send proof you have found our \(R_u\).
%%%
\Subsubsection{Computing response \(B_u\) by the user}
As proof of having found \(R_u\), we expect a value in return of our \(A_u\), which proves that we are dealing with the right website.
Just sending a 256-bit hash value
\(B_u = \HSHA{R_u}\)
would expose \(K_u\).
With some effort, all n-bit random numbers could be checked and \(K_u\) computed.
\par
By taking just the least \(n\) significant bits of the hash:
\(B_u = \mtpn{\HSHA{R_u}}\)
our random number is now hidden in a large set of \(2^{256-n}\) possible values.
That, in itself, does not hide it completely.
Especially with a large \(n\), there may be few (if any) collisions, so a brute force attack may still be fruitful.
Taking it \(\bmod\ 2^n\) does give us the opportunity to encrypt it with \(K_u\):
\begin{equation}\label{rud:bu}
B_u = \xor{\mtpn{\HSHA{R_u}}}{K_u}
\end{equation}
\par
Possibly many random numbers yield the same hash (taken \(\bmod\ 2^n\)).
This is no problem, because we use this random value only once, ever.
Producing this hash value is ample proof that the website has the right key.
\par
The user keeps the \(B_u\) value in memory, until the server returns its computations.
%%%%
\Subsubsection{Computing \(B_s\), \(P_s\), and \(Q_s\) on the server}
The \(A_u\) value is sent to the server, along with the generated \(H_u\) value that was built in an earlier step.
Upon receiving \(A_u\) the server will try to decrypt it.
For that, it needs the user key \(K_u\), which it can obtain using the site key \(K_s\), which is stored in the accounts database under index \(H_u\):
\begin{equation}
K_s = account\_lookup(H_u)
\end{equation}
\par
So, the next cacluations are needed: first get hold of the user key \(K_u\) and then extract the user random \(R_u\).
\begin{equation}\label{rud:sku}
K_u = \EAES{K_s}{S}
\end{equation}
The key \(S\) is the secret key.
We will be discussing this at length later.
\begin{equation}\label{rud:sru}
R_u = \xor{A_u}{K_u}
\end{equation}
Having obtained the \(R_u\) value, we can now compute the value that the user expect us to return.
Note that the \(B_s\) value is computed exactly like \eqref{rud:bu}.
\begin{equation}\label{rud:bs}
B_s = \xor{\mtpn{\HSHA{R_u}}}{K_u}
\end{equation}
The value \(B_s\) is stored in memory for a short while.
\par
Next, we generate challenge \(P_s\) for the user:
\begin{equation}\label{rud:rs}
R_s = random(n)
\end{equation}
\begin{equation}\label{rud:ps}
P_s = \xor{R_s}{K_u}
\end{equation}
The \(Q_s\) value will be the expected return value, which is computed slightly different than \eqref{rud:bs}, now using \(R_u\).
This makes the protocol asymmetrical by design.
\begin{equation}\label{rud:qs}
Q_s = \xor{\mtpn{\HSHA{R_s}}}{R_u}
\end{equation}
\par
We keep \(Q_s\) for later use; the values \(B_s\) and \(P_s\) are sent to the user and can be erased from memory right thereafter.
%%%
\Subsubsection{Validating \(B_u\) and computing \(Q_u\) by the user}
The user should assert that value \(B_s\) is  matching \(B_u\);
if not, the login has failed.
\par
If they match, we then can compute \(Q_u\) as a response to \(P_s\):
\begin{equation}
R_s = \xor{P_s}{K_u}
\end{equation}
Having extracted \(R_s\) we compute \(Q_u\) just like \eqref{rud:qs}:
\begin{equation}
Q_u = \xor{\mtpn{\HSHA{R_s}}}{R_u}
\end{equation}
This value is returned to the website to complete the authentication process.
%%%
\Subsubsection{Finalizing authentication on the server}
When the user sends a \(Q_u\) that matches \(Q_s\), we have sucessfully completed the authentication process.
The user is now logged in.
