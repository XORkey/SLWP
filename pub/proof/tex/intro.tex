\section{Introduction}
This piece of text tries to proof that the \TIMO protocol is secure.
\subsection{Applicability of the protocol}
The applicabitily of this protocol cannot be inferred directly from wether this proof holds or not.
If it does,
it would be implemented in an environment with many other things that influence the overall security.
A safe protocol does not mean that its use prevents the user from any other harm.
\par
If the proof is flawed in some way,
or flawed in some case,
it may still be implemented.
Simply because it is an improvement over the current situation.
\subsection{Assumptions}
The field of application of this protocol can be defined by the extent the assumptions made here hold true.
\subsubsection{Encryption Standards}
The \TIMO protocol uses several standard encryption and hashing functions,%
\footnote{For the record,
we do not assume XOR to be among them.}
and only in ways they are designed for.
A weakness in any of them will cause great problems throughout the world,
given the ubiquity of their use.
We assume that it is quite irrelevant wether or not this weakness will impact this protocol,
since everybody would be panicking about bigger things.
\par
The upshot of this is that,
since in reality it is of course devastating if your foundations crumble,
when the dust settles down this protocol is fixed with the rest,
or as permanently flawed as the rest.
\subsubsection{Hardware Security Modules}
By design,
an HSM is something to operate on data in a secure way.
Typically,
it stores cryptographic keys and computes hashes and encrypts data with them,
all the while keeping those keys to itself.
In other words,
it is assumed that any key residing in an HSM will never leave it in plain text.
There is of course a method of duplicating keys over several HSMs,
but we assume
(again)
that these methods are unbreakable.
\subsubsection{Human factors}
The login servers are quite important,
as they hold the secret keys that make it all work securely.
In theory,%
\footnote{Well,
this is how I demoed it.}
the whole service they offer can be replaced by a single bash script that calls some openssl commands,
with all keys in a flat file on disk.
We assume,
however,
that such a login service would not pass any audit that we also assume will be imposed on such key%
\footnote{Pun intended.}
element of the infrastructure.
