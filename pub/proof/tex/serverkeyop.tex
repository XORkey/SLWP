\Section{Server key operations}
The actions of the server in the presented rudimentary login scheme can be thought of as two separate things: computation and comparison.
Granting a login is an authentication process that is a comparison of two values, basically.
Before this can take place, some computations must be done with keys and random values.
\par
Maintaining and upholding good security needs expertise that does not come natural for many website owners and maintainers.
Even if security has full attention and focus, with a team of experts at the controls, it still is hard to keep things safe.
Simply because the site is almost directly connected to the wild wild west of the Internet; open for anyone, anytime.
\par
It is therefore not entirely unfortunate that we can separate all key operations away from the inherently vulnerable web server.
Instead, we outsource this to a back office, or preferrably remote, server; we then could move most keys along with it.
That looks a bad idea at first sight---increasing complexity, but it proves to be greatly advantageous for security.

\Subsection{Introduction of the login server}
All server side operations that involve keys can be consolidated and separated from the comparison operations needed for the authentication processs.
Instead of implenting this protocol on a single server
(which is still possible if you want to)
we move almost all computations to a separate server.
\par
The function of the web server stays exactly what it was before (serving pages),
and the security measures taken to protect it needs little or no altering.
\par
The logic of granting logins is already in place for web servers that use passwords.
In the case of passwords, this too is a comparison of values that decides whether or not you are authenticated.
This logic can be expanded with a different comparison in case this protocol is used, which is not a great change.
\par
Henceforth, the server granting logins
is simply referred to as 'web server',
which has always been the target system for this protocol.
The server that does all the computations is called the 'login server' from now on, as it computes login values.
\Subsubsection{Securing the login server}
Having a separate login server has many advantages, mainly security-wise.
The calculations with, and (most importantly) the storage of secret keys can be done with a Hardware Security Module (HSM).
The login server can be built around it, tightly hardened, and put somewhere away from open access from the Internet.
\par
In most implementations, the login server will be an unpenetrable bastion on the Internet.
It would be a 'dark' node that hardly iteracts with anything:
it has IP addresses that are not listed in the DNS and drops almost any packet you throw at it.
It could be considered a WIMP: a weakly interacting massively protected server.
Only hardcoded IPsec tunnels exist with its protective external firewall, allowing traffic to and from contracted web servers only.
In dark mode, it emanates and absorbes zero Internet traffic.
Only in maintenance mode the node comes to life, for updates, for instance.
No IPsec tunnels are active during that time, so no service is provided until the system goes 'dark' again.
\Subsubsection{Login as a Service}
A login server could be placed in the back office, like database servers are.
But it could also be positioned on highly accessible spots on the Internet, like in datacentres less than a millisecond away from an Internet Exchange.
A multitude of login servers on such places could offer a redundant login-as-a-service (LAAS) for the smallest, up to the largest sites on the Internet.
\par
The communication from web server to login server is flowing over the Internet,
through IPsec tunnels between the web server's outer firewall and the firewalls that secure each of the selected login servers.
Depending on the size of the web site, this number of login servers can range from just one (for testing) or two (redundancy only) to dozens.
Having many would provide for redundancy, geographic reachability, and the ability to handling the shear load of logins per second.
\par
If the service is a true LAAS, then it could be provided by somebody other than the web site owner.
Then, any of the login servers can be put to use for more than one web server.
The firewall for an individual login server will terminate several IPsec tunnels, one for each web site.
Each login server could be operated on high capacity, and revenues are generated for each serviced login request.
\par
\Subsubsection{Maintaining the login server}
As is true for securing a web server; securing a login server takes effort, expertise, and experience.
However, apart from 'effort', the expertise and experience are quite different.
\par
True, you need to run updates, and check your firewall logs, but there are a miriad of things you do not have to worry about when you run a website.
Most of them boil down to what lengths you go protecting your keys.
\par
Maintaining a login server is not for the faint at heart.
You need full transparency for all your processes and procedures regarding the handling of your keys.
For if they are compromised, you are dead instantly.
Scritinised regularly by authors, where they must find not a speck of dust on any of your cupboards; the house freshly ventilated, an apple pie in the oven.
\begin{dialogue}
\speak{Punk} You wanna buy some death sticks?
\speak{Obi-Wan Kenobi}	\direct{Waves his hands.} You don't want to sell me death sticks.
\speak{Punk}	I don't want to sell you death sticks.
\speak{Obi-Wan Kenobi} You want to go home and rethink your life.
\speak{Punk}	I want to go home and rethink my life.
\attrib{Star Wars: Episode II - Attack of the Clones}
\end{dialogue}
The only thing you want to achieve regarding this is 'absolute trust'.
But trust comes on foot and goes on the Millennium Falcon.
\par
Establishing and maintaining trust at this level is a challenge that you cannot \XOR\ yourself out of.
