\section{Scalability}
There are several degrees of scalability with this login scheme.

\subsection{Keyring}
Instead of a nice and round 100,
you can put any number of keys in a keyring file.
The value of 100 is just convenient;
all keys are numbered with a two-digit number,
which can easily be remembered.
\par
Any number will do,
from zero
(which leaves only the keyring identifier,
and allows you to login to zero websites)
up to any amount of keys you bother to carry around with you.
Having more keys than sites you want to login to is just a way to make things a bit harder for those that do not own the keyring to choose keys.
If you are not happy with this,
you can just add new keys as you acquire them,
just as with real keys.
(Nothing stops you from adding bogus brass and steel keys to the keyring that holds your carkeys.
But I don't believe it is common practice to add BMW and Ford keys to a keyring of a Toyota owner.)

\subsection{Key lengths}
\label{sec:key_length}
The length of keys is generally considered as an important aspect.
The more bits, the better the key.
\par
That is true if such a key is used to encrypt data that is in any way predictable,
like, for example, a piece of text.
If the encrypted data has patterns of any kind,
you can directly work with intermediate decryption attempts.
Statistical analysis of resulting bit patterns can reveil if a certain key or method is getting close or closer.
\par
But all values encrypted with our keys are comprised of random bits only.
This implies that any result of decryption has to be tried,
to establish if the decryption was in any way successful.
That would mean many login attempts
(millions, billions, or more)
which is infeasable.
And that is just to crack a user key,
which only gives you one login.

\subsubsection{Minimum key length}
With a Secret key fixed on 256 bits to accommodate the \AES\ encryption,
all other keys can be a lot smaller than that.
Theoretically,
a 1-bit key could suffice,
but this obviously is not strong enough,
as it would take only two attempts to test all possible values.
\par
To rule out any feasable brute force attempts
(supposing that a site does not stop countless consecutive failed attempts)
a key length of 24 bits should be enough.
Assuming that a single login attempt,
exhausting all Secret keys each time,
could be done in a second
(taking into account the network traffic to download and upload webpages and values),
a 16 bit key would be cracked in at most 18 hours.
365 days have 8760 hours,
and if Secret keys are replaced every half year,
then,
if an attempt takes more than 4380 hours,
it must be considered futile.
Doubling the time with each bit,
a 24 bit key would take 4608 hours of continuous effort,
which is well over half a year.

\subsubsection{Maximum key length}
There is no limit to the number of bits in a key,
but using more and more bits for keys has its trade-offs.
Keys longer than 64 bits require more processing,
as they do not fit in CPU registers common today.
But even keys longer than 32 bits may be subobtimal in some programming languages that are used to make client side login pages,
or server side login programs.
\par
Part of the exchanged values are least significant parts of \SHA\ hashes.
These give ample proof of having the right key,
but reveal nothing of the hashed value---even if the hash function should be reversible.
Therefore,
the number of bits of the user and site keys must not equal the number of bits of the hash result.
\par
There are of course hash functions that produce longer hashes,
like \E{AES}{512},
but putting more than,
say,
128 bits into a key brings no more security.
\par
If this scheme is somehow flawed,
it most likely will not because of insufficient key lengths.
As such,
128-bit keys should be considered the maximum practical key length.

\subsection{Old Secret keys}
At least one old Secret key is required,
if Secret keys are to be replaced every now and then.
An HSM can store many keys,
so there is no real practical limit there.
More likely,
the amount of effort needed to safely work with keys will cap the number to a small integer.

\subsection{Encryption algorithms}
In this article,
the basis for generating the keypairs for the site and the user is \AES.
All site keys and user keys are related using this algorithm,
so it has to be secure.
The encryption algorithm has to be a block cypher,
using a secret key.
A hashing function does not do the trick,
as this would directly reveil user keys once you have the site keys.
But any other block cypher may be employed for this task.
\par
Hashes are computed using \SHA\ in this article.
The hash values that are exchanged between website and user are incomplete by design,
and just contain some of the least significant bits of the hash.
The key length of the site and user keys
should not exceed the lenght of the hash value,
nor have equal length,
as that would weaken the login algorithm.
Using 128-bit keys, a hash function like \MDV\ is not recommended, as this produces a 128-bit hash.

\subsection{Server roles}
There is no technical restriction whatsoever combining all server roles
(webserver, account server, login server)
into a single server.
For security reasons it is advisable to split those roles, however.
\par
The webserver should be separate from the accounts server, if at all possible.
The login server function could be combined with the accounts server;
both somewhere in the back office.
This combination may leave system administrators with two different sets
of security constraints for the two roles.
A separate login server would be preferrable in those cases.
\par
As a final separation option,
the login server may be located outside the domain of the webserver's company.
A stand-alone login server may exist
(well, several of them, in fact)
on the Internet.
These login servers may offer websites their computing capabilities for logins,
and also reduce the burden of dealing with Secret keys.
That would leave a company using this login scheme with minimal changes to its login scheme.
