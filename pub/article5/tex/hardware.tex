\section{Required hardware}
Apart from a server to host the website itself,
you need a Secure Device Provisioning Server (SDPS)
that contains a Hardware Security Module (HSM),
which stores an array of Secret Keys and performs cryptographic functions.
Yet another server should host the accounts database, linking at least userids with site keys.

\subsection{Firewall}
There should be a firewall between the webserver and the SDPS,
to protect the SDPS from attack should the website be compromised.
The SDPS should be placed in a DMZ.
The other link from the firewall will be to the server that hosts the accounts database.

\subsection{Accounts database}
There should be a server for the database with account information.
This database will probably be used for many other things that the website needs as well.

\subsection{Login server}
\label{sec:login_server}
The login server uses a Hardware Security Module (HSM)
for storing cryptographic keys, performing cryptographic functions, and generating random numbers.

\subsection{HSM}
The Hardware Security Module is a piece of tamper proof hardware.
It stores cryptographic keys, certificates, and other data in a secure manner.
It also has a whealth of cryptographic functions and can generate good quality random numbers.
The HSM should be programmed in such a manner that even the SDPS, which hosts the HSM, is only able to use a small set of functions.
\par
Ideally, the set of functions the SDPS should provide should be programmed directly into the HSM.
This way, the Secret keys can be kept secret, and the SDPS cannot be misused by a hacker to regenerate login data.
