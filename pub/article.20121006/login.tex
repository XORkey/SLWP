\section{Logging in}
The user, wanting to login to a site, has to lookup or remember the keyring's PIN (if any), the key number, and the userid.
Having unencrypted its keyring, selected a key number ($k$), and entered the userid, the login process can commence.
For logging in, the user needs a little program to do some calculations and to operate on the keyring.

%%%%%%%%%%%%%%%%%%%%%%%%%%%%%%%%%%%%%%%%%%%%%%%%%%%%%%%%%%%%%%%%%%%
\subsection{Step 1: knock, knock\ldots}
\label{sec:login_step1}
The login program has to compute the folowing:
\begin{enumerate}
\item A random number ($R_u$) the website will need to appreciate by returning its hash later on:
\[R_u = \mathtt{Random}()\]
\item The user key ($K_u$) is taken from the keyring at index $k$ and the random number is encrypted with it to get the value $A_u$:
\[A_u = R_u \oplus K_u\]
\item The website will return a hash value of the user's random value.
We need to be able to verify this hash, so we compute our own hash ($B_u$) to verify it with:
\[B_u = \E{SHA}(R_u)\]
\item Finally, we need the hash of the userid ($U$) to present to the website:
\[U=\E{SHA}(userid)\]
\end{enumerate}
The hashed userid $U$ and the encrypted random number $A_u$ are sent to the website.

%%%%%%%%%%%%%%%%%%%%%%%%%%%%%%%%%%%%%%%%%%%%%%%%%%%%%%%%%%%%%%%%%%%
\subsection{Step 2: site key lookup}
\label{sec:login_step2}
After receiving $U$ from the user, the site key $K_s$ needs to be looked up.
A query with $U$ is sent to the database with account information.
If a match is found, $K_s$ is returned;
otherwise, $K_s$ is set to zero, indicating that no such record exists.
\par
In the latter case, the values $B_s$ and $P_s$ are both set to zero as well, and returned to the user.
The user needs to rethink its actions and start over.

%%%%%%%%%%%%%%%%%%%%%%%%%%%%%%%%%%%%%%%%%%%%%%%%%%%%%%%%%%%%%%%%%%%
\subsection{Step 3: SDPS actions}
\label{sec:login_step3}
Now an iterative process starts, trying to find the right Secret key to log the user in (see also section \ref{sec:secret_keys}).
The loop runs at most $min(m,\mathtt{MAX\_KEYS})$ times.
The website will send three values to the SDPS (see section \ref{sec:SDPS}): $A_u$, $K_s$, and a Secret key index $i$, starting at 0.
\par
In case $i$ is less than $m$
(the number of stored Secret keys, see section \ref{sec:secret_keys})
the HSM calculates the following:
\begin{enumerate}
\item Temporarily, regenerate a user key, using the same algorithm as when the key was originally generated:
\[K_u=\E{AES}(K_s,S[i])\]
with $S[i]$ the $i$-th Secret key stored in the HSM.
\item With the user key $K_u$, decrypt the random the user has sent:
\[R_u=A_u \oplus K_u\]
\item Calculate a hash with which the user can verify we own the site key $K_s$ and a corresponding Secret key $S[i]$:
\[B_s=\E{SHA}(R_u)\]
\item To be able to verify that the user ows and knows its user key $K_u$,
is not sending a replay of some earlier succesfull login sequence,
and will be unable to ly about the correctness of the hash of the random the site will send,
we calculate a challenge for the user.
\par
If the index $i$ equals zero, generate a random value
\[R_s=\mathtt{Random}()\]
otherwise, compute
\[R_s=\E{AES}(K_s,S[0])\]
which will be the new key for the user.
\par
We then encrypt $R_s$ to a value $P_s$ the user can decrypt using its key:
\[P_s=R_s \oplus K_u\]
This value will be different for each time the loop is executed,
even if the same new key is transmitted,
since $K_u$ will change each time.
\item Calculate the expected response also:
\[Q_s=\E{SHA}(R_s)\]
\end{enumerate}
The HSM will produce the values $B_s$, $P_s$, and $Q_s$ as a result of the calculations and the SPDS will send them back to the site,
as an answer to the three values it was given ($A_u$, $K_s$, and $i$).
The HSM will delete all temporary values from memory directly afterwards, and remember only its Secret keys.
\par
Should index $i$ equal $m$, then the array of Secret keys is exhausted.
If we reach this situation, we cannot log the user in since all possible attempts have failed.
Return zero values for $B_s$, $P_s$, and $Q_s$ to indicate this.

%%%%%%%%%%%%%%%%%%%%%%%%%%%%%%%%%%%%%%%%%%%%%%%%%%%%%%%%%%%%%%%%%%%
\subsection{Step 4: user verifies site key}
\label{sec:login_step4}
The site sends $B_s$ and $P_s$ to the user.
If both are zero, the login has failed and the user should return to step 1.
It is advisable for the user to choose different values for the next try.
\par
If $B_s$ contains a nonzero value, the user verifies if $B_s$ equals $B_u$.
\par
If the two values do not match, the user either has selected the wrong key or uses an old key.
A key can be old if it comes from a keyring that was not used recently.
Another possible reason for a key to be old is when the Secret key of the site has changed recently.
\par
Therefore, only after at least two unsuccesful attempts
the user may be presented a question whether it likes to select a different key from the ring, or continue trying with this key.
If the user opts to select a different key, the login procedure has to be aborted.
This may be done by setting
\[Q_u=\mathtt{0xffffffffffffffffffffffffffffffff}\]
(128 ones) and send this to the website.
The website may present the user a new login screen, to start over, or accept this silently.
At least, the Secret key index $i$ must be reset to \texttt{0}.
We return to step 1.
\par
If further key matching should be attempted, indicate this by
\[Q_u=\mathtt{0x0}\]
(128 zeroes) and send this value to the website.
This will need to be done at least once, when values don't match the first time.
The SDPS will need to start over, now with values $A_u$, $K_s$, and $i+1$.
So, back to step 3.

%%%%%%%%%%%%%%%%%%%%%%%%%%%%%%%%%%%%%%%%%%%%%%%%%%%%%%%%%%%%%%%%%%%
\subsection{Step 5: site verifies user key}
\label{sec:login_step5}
If $B_s$ matches $B_u$, then the website has found a Secret key for the user.
The user can now prove it has control over its user key by computing
\[R_s=P_s \oplus K_u\]
and
\[Q_u=\E{SHA}(R_s)\]
which is sent to the website as proof.
\par
If the website accepts $Q_u$ as correct it will log the user in.
\par
If the website receives a value $Q_u$ that does \emph{not} match,
something fishy is going on.
Further attempts for this userid, or from that IP address should be scrutinized.

%%%%%%%%%%%%%%%%%%%%%%%%%%%%%%%%%%%%%%%%%%%%%%%%%%%%%%%%%%%%%%%%%%%
\subsection{Step 6: key replacement}
\label{sec:login_step6}
If this attempt to login was not the first with this key,
the website has sent the user a new key in $R_s$ (instead of a random value).
The user should store this new key in the keyring, overwriting the old key the user has just used.

%%%%%%%%%%%%%%%%%%%%%%%%%%%%%%%%%%%%%%%%%%%%%%%%%%%%%%%%%%%%%%%%%%%
\subsection{Cryptographic considerations}
\label{sec:crypto}
The exchanged values used above are 128 bits long,
but could be taken much smaller;
say,
16 bits each,
without doing great injustice to the level of security.
The use of 128 bits is only a convenient size,
as \E{AES} and \E{SHA} work with these lengths naturally.
This way,
we take full advantage of their cryptographic power.
\par
The random values that are exchanged are encrypted using XOR.
Since the key is also a random value,
there is no way either values can be obtained using cryptoanalysis,
guessing,
or brute force attacks.
The returned \E{SHA} hash,
however,
may be of more cryptographic concern.
Although considered 'safe' at the moment of writing,
it cannot be guaranteed that this will stay that way.
Possibly, exchanging the full hash of a random value may be giving too much information.
The random value might be calculated from the whole hash value, an thus $K_u$ or $K_s$.
\par
Instead of using the whole value of the \E{SHA} function,
we could use only the least significant 64 bits of it.
For example, calculating $B_u$ now changes to
\[B_u = \E{SHA}(R_u)\  \mathbf{mod}\  \mathtt{0xffffffffffffffff}\]
The same modulus should be applied to values $B_s$, $Q_u$, and $Q_s$.
All four values are reduced to 64 bits.
\par
By only using the lower half of the hash, the random cannot be computed
(it lies in a field of $2^{64}$ possible values)
but this half value is ample proof that the right random value was used to create it.
%%%
