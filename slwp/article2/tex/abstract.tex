\begin{abstract}
These are the ingredients for secure logins:
\begin{itemize}
\item The website maintains a small array with Secret keys in an HSM, which will be renewed regularly
(oldest key removed, all keys shifted one position, new key added).
\item All cryptographic operations at the website take place in the HSM;
no values are remembered, except Secret keys.
\item For each user, a 128 bit random number acts as the site key for that user.
The user gets a different 128 bit value, which is the \AES\ encryption of the site key for that user with the newest Secret key.
\item Logging in is a process of proving that both the website and the user have the right key by sending hashes of
random 128 bit numbers encrypted with these keys.
Neither the site key nor the user key are ever sent over the line.
\item When Secret keys are changed,
the user automatically gets a new key.
This way, user keys are changed regularly as well.
\item All user keys are put on a keyring,
which is a set of at least 99 keys, each 128 bit long, most of them dummies.
No other information whatsoever is stored in a keyring, which may be stored in an ordinary file.
The keyring may be encrypted with its own keys.
\item The user selects which key is used for what, which it needs to write down or remember.
Although keys themselves are changed regularly,
the key number and its purpose will never change during the lifetime of the keyring, which might be forever.
This makes remembering key numbers, at least for those keys that are used regularly, doable for most people.
\item The keyring itself is something you must have; the key number something you must know,
so logging in using a keyring is a basic form of two-factor authentication.
\end{itemize}
\end{abstract}
