\section{Conclusion}
The security of the login process for websites can be greatly improved by using a keyring at the user side and an HSM at the website's side.
Instead of sending relatively short, easy to guess strings (passwords) over the line,
the use of encrypted random values,
of 128 bits each,
is a big improvement.
No keys are sent,
just random values,
which will be different each time a user logs in.
\par
For hackers,
getting login data in huge numbers will be very difficult,
since this data is no longer stored centrally,
but split between website and user.
Each part alone has no value,
and all values stored at the website render no valid login data without Secret keys.
These keys are kept in very secure hardware: an HSM.
Each and every user key is encrypted with different other keys on keyrings.
Sniffing network traffic
(even on links without SSL/TLS encryption),
or collecting keystrokes with Trojans will not help.
Keyrings are encrypted,
so to no direct use to hackers as well;
they may be stored on the web for easy access and backup.
\par
Several important security measures are automatically implemented:
keys are changed frequently
(as frequent as Secret keys are changed),
they differ for each website,
and keys on a keyring and the knowledge which key is used for what site constitutes two-factor authentication.
\par
For the user,
the way to logon to websites will change,
but it will be an improvement over the burden of keeping track of all passwords.
A single key number for a website is all you have to remember.
\par
The software to make it all happen is easy to write,
since the algorithm for logging in,
and to handle keyrings is not difficult.
No hardware on the user's side is necessary.
The hardware to implement this on the website's side is relatively cheap,
compared to all other security measures a website (still) needs to have
(Firewalls, Intrusion Prevention Systems, etcetera).
