\section{Logging in}
The user, wanting to login to a site, has to lookup or remember the keyring's PIN (if any), the key number, and the userid.
Having unencrypted its keyring, selected a key number ($k$), and entered the userid, the login process can commence.
For logging in, the user needs a little program to do some calculations and to operate on the keyring.

%%%%%%%%%%%%%%%%%%%%%%%%%%%%%%%%%%%%%%%%%%%%%%%%%%%%%%%%%%%%%%%%%%%
\subsection{Applying for a userid}
\label{sec:applying}
When a new user applies for an account, it should choose a userid%
---one that the website will accept as unique.
Furthermore, it should select the index of a hitherto unused key
(therefore, a dummy).
The user will send its $\SHA$ hash of its userid,
along with its dummy key $K_d$ to the website.
\par
The website will send the dummy key $K_d$ to an SDPS hosting an HSM (see section \ref{sec:SDPS}),
which will do the following:
\begin{enumerate}
\item Generate a random
\[K_s=\F{Random}{}\]
which will be the new site key for this user.
\item Then it will encrypt the $K_s$ with the current Secret key $S[0]$ to yield the new user key $K_u$
\[K_u=\F{\AES}{K_s, S[0]}\]
\item Finally, the new user key is encrypted with the user's dummy key $K_d$
with an XOR operation:
\[K_x=\XOR{K_u}{K_d}\]
\end{enumerate}
The SDPS will then send the new key $K_s$ and the $\SHA$ hash of the userid to the server hosting the accounts database.
It will send $K_x$ in a mail to the new user (optionally with the index the user supplied).
\par
The user's keyring program can import the new key $K_x$ into the keyring by calculating
\[K_u=\XOR{K_x}{K_d}\]
using the index it has originally selected for the slot in the keyring.
The dummy key $K_d$ is overwritten with the new key $K_u$.


\subsection{Basic login scheme}
The procedure described below to login is the basic login scheme.

%%%%%%%%%%%%%%%%%%%%%%%%%%%%%%%%%%%%%%%%%%%%%%%%%%%%%%%%%%%%%%%%%%%
\subsubsection{Step 1: knock, knock\ldots}
\label{sec:login_step1}
The user's login program has to compute the folowing:
\begin{enumerate}
\item A random number ($R_u$) the website will need to appreciate by returning its hash later on:
\[R_u=\F{Random}{}\]
\item The user key ($K_u$) is taken from the keyring at index $k$ and the random number is encrypted with it to get the value $A_u$:
\[A_u=\XOR{R_u}{K_u}\]
\item The website will return a hash value of the user's random value.
We need to be able to verify this hash, so we compute our own hash ($B_u$) to verify it with:
\[B_u=\sha{R_u}\]
\item Finally, we need the hash of the userid ($U$) to present to the website:
\[U=\F{\SHA}{userid}\]
\end{enumerate}
The hashed userid $U$ and the encrypted random number $A_u$ are sent to the website.

%%%%%%%%%%%%%%%%%%%%%%%%%%%%%%%%%%%%%%%%%%%%%%%%%%%%%%%%%%%%%%%%%%%
\subsubsection{Step 2: site key lookup}
\label{sec:login_step2}
After receiving $U$ from the user, the site key $K_s$ needs to be looked up.
A query with $U$ is sent to the database with account information.
If a match is found, $K_s$ is returned;
otherwise, $K_s$ is set to zero, indicating that no such record exists.
\par
In the latter case, the values $B_s$ and $P_s$ are both set to zero as well, and returned to the user.
The user needs to rethink its actions and start over.

%%%%%%%%%%%%%%%%%%%%%%%%%%%%%%%%%%%%%%%%%%%%%%%%%%%%%%%%%%%%%%%%%%%
\subsubsection{Step 3: SDPS actions}
\label{sec:login_step3}
Now an iterative process starts, trying to find the right Secret key to log the user in (see also section \ref{sec:secret_keys}).
The loop runs at most $min(m,\mathtt{MAX\_ACTIVE\_KEYS})$ times.
The website will send three values to the SDPS (see section \ref{sec:SDPS}): $A_u$, $K_s$, and a Secret key index $i$, starting at 0.
\par
In case $i$ is less than $m$
(the number of stored Secret keys, see section \ref{sec:secret_keys})
the HSM calculates the following:
\begin{enumerate}
\item Temporarily, regenerate a user key, using the same algorithm as when the key was originally generated:
\[K_u=\F{\AES}{K_s,S[i]}\]
with $S[i]$ the $i$-th Secret key stored in the HSM.
\item With the user key $K_u$, decrypt the random the user has sent:
\[R_u=\XOR{A_u}{K_u}\]
\item Calculate a hash with which the user can verify we own the site key $K_s$ and a corresponding Secret key $S[i]$:
\[B_s=\sha{R_u}\]
\item To be able to verify that the user ows and knows its user key $K_u$,
is not sending a replay of some earlier succesfull login sequence,
and will be unable to ly about the correctness of the hash of the random the site will send,
we calculate a challenge for the user.
\par
If the index $i$ equals zero, generate a random value
\[R_s=\F{Random}{}\]
otherwise, compute
\[R_s=\F{\AES}{K_s,S[0]}\]
which will be the new key for the user.
\par
We then encrypt $R_s$ to a value $P_s$ the user can decrypt using its key:
\[P_s=\XOR{R_s}{K_u}\]
This value will be different for each time the loop is executed,
even if the same new key is transmitted,
since $K_u$ will change each time.
\item Calculate the expected response also:
\[Q_s=\sha{R_s}\]
\end{enumerate}
The HSM will produce the values $B_s$, $P_s$, and $Q_s$ as a result of the calculations and the SPDS will send them back to the site,
as an answer to the three values it was given ($A_u$, $K_s$, and $i$).
The HSM will delete all temporary values from memory directly afterwards, and remember only its Secret keys.
\par
Should index $i$ equal $m$, then the array of Secret keys is exhausted.
If we reach this situation, we cannot log the user in since all possible attempts have failed.
Return zero values for $B_s$, $P_s$, and $Q_s$ to indicate this.

%------------------------------------------------------------------
\subsubsection{Step 4: user verifies site key}
\label{sec:login_step4}
The site sends $B_s$ and $P_s$ to the user.
If both are zero, the login has failed and the user should return to step 1.
It is advisable for the user to choose different values for the next try.
\par
If $B_s$ contains a nonzero value, the user verifies if $B_s$ equals $B_u$.
\par
If the two values do not match, the user either has selected the wrong key or uses an old key.
A key can be old if it comes from a keyring that was not used recently.
Another possible reason for a key to be old is when the Secret key of the site has changed recently.
\par
To indicate that no match has been found, send
\[Q_u=\mathtt{0x0}\]
(128 zeroes) to the website.
The SDPS will need to start over, now with values $A_u$, $K_s$, and $i+1$.
So, back to step 3.

%------------------------------------------------------------------
\subsubsection{Step 5: site verifies user key}
\label{sec:login_step5}
If $B_s$ matches $B_u$, then the website has found a Secret key for the user.
The user can now prove it has control over its user key by computing
\[R_s=\XOR{P_s}{K_u}\]
and
\[Q_u=\sha{R_s}\]
which is sent to the website as proof.
If the website accepts $Q_u$ as correct it will log the user in.
\par
If the website receives a value $Q_u$ that does \emph{not} match,
something fishy is going on.
Further attempts for this userid, or from that IP address should be scrutinized.

%------------------------------------------------------------------
\subsubsection{Step 6: key replacement}
\label{sec:login_step6}
If this attempt to login was not the first with this key,
the website has sent the user a new key in $R_s$ (instead of a random value).
The user should store this new key in the keyring, overwriting the old key the user has just used.

%%%%%%%%%%%%%%%%%%%%%%%%%%%%%%%%%%%%%%%%%%%%%%%%%%%%%%%%%%%%%%%%%%%
\subsection{Optimizations and alternatives}
With the basic login scheme several optimizations and alternatives are available.
Here are some obvious ones.
%------------------------------------------------------------------
\subsubsection{Aborting the login procedure}
Logging in with an old key every now and then is inherent in this scheme,
as most Secret keys will be changed at regular intervals.
Therefore,
it is likely to sometimes receive a hash value $B_s$ that does not match $B_u$.
An unsuccessful second attempt may be the result of using a seldomly used key or a wrong key.
So, after at least two unsuccesful attempts
the user may be presented a question whether it likes to select a different key from the ring,
or continue trying with this key.
If the user opts to select a different key, the login procedure has to be aborted.
This may be done by setting
\[Q_u=\mathtt{0xffffffffffffffffffffffffffffffff}\]
(128 ones) and send this to the website
(instead of $0x0$).
We return to step 1.

%------------------------------------------------------------------
\subsubsection{Detection of false logins}
If someone has stolen a keyring it should be useless.
Sequentially using all keys in the keyring to try to find the right key should not work.
After each unsuccessful attempt the account should be blocked for exponentially increasing amounts of time.
Starting from, say, a second, but after a second attempt a minute, then five minutes, an hour, a day, a week.

%------------------------------------------------------------------
\subsubsection{Using last login date}
Instead of always starting with Secret key $S[0]$,
the date of the last login of the user can be taken into account.
If the query with $U$ as key that is sent to the accounts database would also return the last login date a starting key could be selected.
It is no use trying new keys when you know beforehand that these keys were added after the last successful login.

%------------------------------------------------------------------
\subsubsection{Conditional key replacement}
For most websites, restoring user keys that refer to the most recent Secret key $S[0]$ will be done without hesitation.
For some,
refreshing keys will be done only when certain conditions are met,
like payment of monthly fees,
a certain number of reviews written,
or some amount of data uploaded.
Until then,
logging in with a valid
(but in this context a typically `old')
key is granted.
But if, for instance, payment is overdue, the key will expire and logging in is no longer possible.
\par
To indicate that no new key is sent,
use
\[P_s = A_u\]
and return this value to the user.
In this case,
no keys should be decrypted or overwritten.

%%%%%%%%%%%%%%%%%%%%%%%%%%%%%%%%%%%%%%%%%%%%%%%%%%%%%%%%%%%%%%%%%%%
\subsection{Changing Secret keys}
At regular intervals a new Secret key is introduced and,
at the same time,
an old Secret key may be sent to the Eternal Hunting Grounds.

%------------------------------------------------------------------
\subsubsection{Terminated accounts}
When deleting the oldest Secret key from memory,
all accounts with a `last login' date older than the installation
date of the second oldest Secret key should be marked `terminated'.
Removal of the oldest key renders all those keys unusable.
Those accounts can be purged, as they cannot be used again.

%------------------------------------------------------------------
\subsubsection{Expired accounts}
In some situations user access must be barred until some condition is met.
Accounts can be made temporarily inaccessible for this purpose by letting keys expire.
Expiration can happen automatically when users do not login in time to get their keys replaced,
or intently by denying key updates.
Expired user keys are associated with Secret keys with an index in the range $[\texttt{MAX\_ACTIVE\_KEYS},\texttt{MAX\_KEYS})$.
\par
Expired accounts can easily be made active again by using inactive Secret keys for the login process (and then renew the key).
\par
Expired accounts can be reinstated,
but only before the associated Secret key is erased from memory.
After that, there is no way to regenerate the user key for logging in.
The user should apply for a new account if it wants to regain access.
