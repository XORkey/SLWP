%
% $Header: /usr/local/src/slwp/article/tex/keys.tex,v 1.8 2014/02/01 13:15:42 timo Exp $
%
% AUTHOR:	ing. T.M.C Ruiter
%
\section{Keys}
The keys used in this login scheme do not all have the same length.
The Secret keys have 256 bits, for instance.
The $\mathtt{RSA}$ operations use keys of 2048 bits.
Keys stored in a keyring have 128 bits,
but fewer bits may actually be used.
The number of bits in such a key may differ among websites,
and is denoted in this article as value $n$.
See \vref{sec:key_length} for a discussion of key lengths.
\par
Three sort of keys are used:
user keys
(on a keyring),
site keys
(in a database),
and Secret keys
(in an array in an HSM).
A user key is the result of encrypting the site key for that user with a Secret key.

\subsection{Secret keys}
\label{sec:secret_keys}
The keys used to encrypt other keys%
---called Secret keys in this article---%
are used to perform \AES\ encryptions of site keys to get user keys,
and are 256 bits long.
They reside in the HSM of a login server.
\subsubsection{Array of Secret keys}
For enhanced security,
Secret keys need to be replaced at regular intervals.
This change of keys is initiated by the website,
without the possibility to make individual arrangements with any of its users.
\par
If there were only one Secret key,
changing it would render all user keys permanently useless.
Therefore,
an array of Secret keys needs to be kept
(with room for at least one old Secret key),
allowing users to login using an old(er) key.
Several Secret keys are used this way and
are considered to be stored in array $S$, in this article.
\begin{table}[h]
\begin{tabular}{|c|c|c|c|c|c|c|c|}
\hline
1 & 2 & \ldots & $\mathtt{MAX\_ACTIVE\_KEYS}$ & \ldots & $\mathtt{MAX\_KEYS}$ \\
\hline
\end{tabular}
\end{table}
\par
With this array, three values are defined.
\begin{enumerate}
\item	The number of stored Secret keys is represented by $m$.
\item	The constant%
\footnote{Although declared a constant, the value of \texttt{MAX\_KEYS} may change over time.
When the login policy regarding the use of old keys is changed, more or fewer old keys will be stored.
For this, \texttt{MAX\_KEYS} needs to be changed.}
\texttt{MAX\_KEYS}, which is the maximum number of keys that will be kept.
The value of $m$ starts at 0 and will reach \texttt{MAX\_KEYS} eventually and grow no more.
The value \texttt{MAX\_KEYS} has a minimum of 2 (a new key and at least 1 old key).
All values in the array $S$ are shifted up one position when a new key is stored, which will always be inserted at $S[0]$.
This way, with at least one key loaded ($m>0$), it always holds that $S[0]$ is the newest key,
and $S[m-1]$ is the oldest key.
With $m>1$, $S[1]$ is the youngest old key.
\item	The constant \texttt{MAX\_ACTIVE\_KEYS},
which is the maximum number of keys that will be used for logging users in.
This value lies in the range $[1, \mathtt{MAX\_KEYS}]$.
\par
Subtracting \texttt{MAX\_ACTIVE\_KEYS} from \texttt{MAX\_KEYS}
gives the number of inactive keys which can be used to reactivate expired user keys.
Users using a user key that is older than the oldest active Secret key cannot login,
but their key can be restored with an inactive Secret key.
If a key is used that is older than the oldest inactive key,
the key cannot be restored and is lost forever.
\end{enumerate}
\par
Upon successful login with an old key,
the user will be provided with a new user key automatically,
with which it can login using the newest Secret key,
from that moment on.
Changing of user keys is seamless
(see section~\vref{sec:userkeys})
and the user might,
just as well,
be kept unaware of this.
\par
The website's security policy should prescribe with what frequency Secret keys are to be changed,
and how many old keys should be kept.
If,
for instance,
the Secret key is changed every month,
and 11 old keys are kept,
users can still login using a key that is a year old.
If the user key is older than that,
the site is not able to verify the user's key any longer.

%------------------------------------------------------------------
\subsubsection{Terminated accounts}
When deleting the oldest Secret key from memory,
all accounts with a `last login' date older than the installation
date of the second oldest Secret key should be marked `terminated'.
Removal of the oldest key renders all those keys unusable.
Those accounts can be purged, as they cannot be used again.

%------------------------------------------------------------------
\subsubsection{Expired accounts}
In some situations user access may be barred until some condition is met
(see also \vref{sec:conditional_key_replacement}).
Accounts can be made temporarily inaccessible for this purpose by letting keys expire.
Expiration can happen automatically when users do not login in time to get their keys replaced,
or intently by denying key updates.
Expired user keys are associated with Secret keys with an index in the range $[\texttt{MAX\_ACTIVE\_KEYS},\texttt{MAX\_KEYS})$.
\par
Expired accounts can easily be made active again by using inactive Secret keys for the login process (and then renew the key).
\par
Expired accounts can be reinstated,
but only before the associated Secret key is erased from memory.
After that, there is no way to regenerate the user key for logging in.
The user should apply for a new account if it wants to regain access.
%%%%%%%%%%%%%%%%%%%%%%%%%%%%%%%%%%%%%%%%%%%%%%%%%%%%%%%%%%%%%%%%%%%
\subsection{Site keys}
For each user,
a site key is generated once by a pseudo-random generator of the HSM of the login server
(see section~\vref{sec:apply_step3}),
and need never be replaced.
\par
The site keys are stored in a table of a database of the accounts server,
where a hash value will act as the primary key for that table.
Each time a user tries to login,
the database is queried with the hash value the user supplies,
and the site key for that user is returned for further processing.

%%%%%%%%%%%%%%%%%%%%%%%%%%%%%%%%%%%%%%%%%%%%%%%%%%%%%%%%%%%%%%%%%%%
\subsection{User keys and keyrings}
\label{sec:userkeys}
Together with the site key,
a user key is computed by encrypting the site key for the user with the youngest
(or current)
Secret key,
using the \AES\ algorithm
(also see section~\vref{sec:apply_step3}).
\par
The keys users get from the website are stored on a keyring (see section~\vref{sec:keyring}).
The keyring will be considered an array $Z$ in the rest of the article.
A key in a keyring is automatically replaced with a new key by the webserver
whenever the Secret key of the login server is changed
(see section~\vref{sec:login_step6}).

\subsubsection{Storing user keys in a keyring}
\label{sec:keyring}
A keyring is created by using a random generator of sufficient quality,
generating a set of random numbers of 128 bits,
stored in a file.
Key number zero is used to identify the keyring and will never change after its creation.
The other keys are dummies
(random bits with no meaning whatsoever),
some of which will be overwritten with real keys over time.
Real keys in the keyring should optimally be put randomly between the dummy keys.
\par
Typically,
100 random numbers are generated this way,
so all keys are conveniently numbered with a one or two digit number
(1 through 99),
which can easily be remembered.
However,
you can put any number of keys in a keyring file.
From zero keys
(which leaves only the keyring identifier $Z[0]$,
and allows you to login to zero websites)
up to any amount of keys you bother to carry around with you.
\par
Having more keys than sites you want to login to is just a way to make things
a bit harder for those that do not own the keyring to choose keys.
If you are not happy with this,
you can just add new keys as you acquire them,
just as with real keys.

\subsubsection{Copying keyrings}
A keyring can be freely copied to other devices, so all keys are available there as well.
\par
Encrypted keyrings can be stored in a public place, for easy access, and act as a backup.
A dedicated webserver can be employed for storing encrypted keyrings,
which can be used to restore lost keyrings.
They can also be used to login when away from home with no access to your own keyring.
These temporary keyrings should be discarded when logging out.

\subsubsection{Irretrievably lost keys and keyrings}
When a key number is forgotten,
or a key inadvertently overwritten,
and no record or backup of it can be found,
the key must be considered lost.
\par
As the user has supplied websites with personal information,
it may be easy to regain login capabilities by just asking for a new key,
provided a username and maybe some other information can be given.
Selecting a free keynumber on the keyring and replacing that key with the new one should restore the ability to login.
\par
Keyrings can become unusable or lost in two situations:
the device holding it is defective
(like a harddisk crash)
and no backup has been made,
or the encryption cannot be reversed
(PIN forgotten).
In both cases,
the user has to start over and create a new keyring,
containing only dummy keys.

\subsection{Certificates and key pairs}
As part of the security of the login process,
keys need to be exchanged in encrypted form to and from the login server.
Furthermore,
the website may want to signify to its users that it is using a trusted login server.
\par
To accomodate for this requirement, the login server needs a certificate that holds a public encryption key.
The private key that belongs to it is stored in the HSM.
With this certificate the users of the login system can validate the trustworthyness of the used login server
by validating the chain of trust.
The certificate is signed by a Certificate Authority (CA).
This CA's certificate is also signed, etcetera, up to a Root CA.
A list of all trusted Root CA's is stored in the browser of the user
and is already used to validate the certificates of websites.
\subsubsection{Login server keys}
The login server uses two types of \RSA\ keys:
an encryption key and a signing key; both private parts tucked away in its HSM.
The public counterparts of these keys are presented to the user with a certificate,
signed by a Certificate Authority so the user can validate them.
\par
The encryption keys of the login server are called $K_{le}$ for the public key
(`l' for `login'; `e' for `encryption'),
and $K_{LE}$ for the private key.
The signing keys of the login server are called $K_{ls}$ for the public key
(`s' for `signing'),
and $K_{LS}$ for the private key.
\subsubsection{Accounts server keys}
Like the login server, the accounts server needs its own set of keys for encryption of other keys.
No signing is neccesary,
so we have $K_{ae}$ as the public key
(`a' for `account')
and $K_{AE}$ as the private key.

\subsection{Key lengths}
\label{sec:key_length}
The length of keys is generally considered as an important aspect.
The more bits, the better the key.
\par
That is true if such a key is used to encrypt data that is in any way predictable,
like, for example, a piece of text.
If the encrypted data has patterns of any kind,
you can directly work with intermediate decryption attempts.
Statistical analysis of resulting bit patterns can reveil if a certain key or method is getting close or closer.
\par
But all values encrypted with our keys are comprised of random bits only.
This implies that any result of decryption has to be tried,
to establish if the decryption was in any way successful.
That would mean many login attempts
(millions, billions, or more)
which is infeasable.
And that is just to crack a user key,
which only gives you one login.

\subsubsection{Minimum key length}
With a Secret key fixed on 256 bits to accommodate the \AES\ encryption,
all other keys can be a lot smaller than that.
Theoretically,
a 1-bit key could suffice,
but this obviously is not strong enough,
as it would take only two attempts to test all possible values.
\par
To rule out any feasable brute force attempts
(supposing that a site does not stop countless consecutive failed attempts)
a key length of 32 bits should be more than adequate.
\par
In a setup of 12 Secret keys, changed monthly,
you cannot login after a year's worth of trying,
since your key has expired by then.
To crack a key you will need to try each one individually.
Statistically, the average time to find it is half the time to test them all,
so you must be kept busy for at least two years to regard a key safe enough.
\par
Assuming that a single login attempt,
exhausting all Secret keys each time,
could be done in a second
(taking into account the network traffic to download and upload webpages and values),
you can do 3600 test in one hour.
Two years have 17520 hours in total,
so $17520\times 3600=63,072,000$ attempts could be done within that time.
A 26-bit key would require $2^{26}=67,108,864$ attempts.
\par
But having 12 Secret keys would also give you 12 posibile hits with each attempt,
dividing the time to find it by twelve.
Adding another 4 bits to the key
(30 bits)
would give room for 16 Secret keys.
A 32-bit key would enable you to have 64 Secret keys and still need more than 2 years of continuous effort to test all of them.
\par
Keys with fewer than 32 bits are still viable,
as long as other measures are taken against brute force attemtps.
If that is the case,
keys can be a lot shorter,
without making a brute force attempt an attractive option.

\subsubsection{Maximum key length}
There is no limit to the number of bits in a key,
but using more and more bits for keys has its trade-offs.
Keys longer than 64 bits require more processing,
as they do not fit in CPU registers common today.
But even keys longer than 32 bits may be subobtimal in some programming languages that are used to make client side login pages,
or server side login programs.
\par
Part of the exchanged values are least significant parts of \SHA\ hashes.
These give ample proof of having the right key,
but reveal nothing of the hashed value---even if the hash function should be reversible.
Therefore,
the number of bits of the user and site keys must not equal the number of bits of the hash result.
\par
There are of course hash functions that produce longer hashes,
like \E{AES}{512},
but putting more than,
say,
128 bits into a key brings no more security.
\par
If this scheme is somehow flawed,
it most likely will not because of insufficient key lengths.
As such,
128-bit keys should be considered the maximum practical key length.
