\section{Login for dummies}
Not that you are one if you decide to read this section.
It serves as a general description of the login process,
without going into technical details too much.

\subsection{A new login scheme}
Userids tend to be in short supply for people with first names like John, Peter, and Chris,
or surnames like Jones or Smith.
This is true for people in almost any country.
And the larger the site, the rarer free userids.
\par
Passwords tend to be weak,
as people choose short, real-life words, like things or names.
Not that people are not willing, but remembering too long, too difficult passwords is not easy.
It is very frustrating when you cannot login because you have forgotten your password,
or,
just when you are starting to remember it,
you need to change it again.
\par
As a radical measure, we do away with all this!
\par
Instead of userids, the website holds hashes based on keyring identifiers.
Instead of a password for an account, the website holds one part of a key, of which you have the other part,
like the broken locket Annie clings to in the musical that bears her name.
Your key is different from the key the website has, but they are related to each other.
It is not possible to login with the key that is stored on the website;
you can only login with your personal key, which is verified with the key from the website.
\par
We will let the computers
(yours, the one running the website, and another one dedicated for logins)
do what they do best: compute.
Give them some random bits to chew on, and they are in heaven.
We will let humans (you, your neighbor and your fellow earthlings) do what they do best:
remembering small ordinal numbers.

\subsection{Storing your keys}
Keys are just long strings of random bits with no information whatsoever.
Your keys are personal and are stored locally.
\par
Imagine a keyring,
	not unlike your own keyring on which you have keys for your car,
		your house,
			shed,
				or locker.
This keyring has a label and 99 keys on it,
	numbered 1 through 99.
Instead of brass or steel,
	these keys are made of 128 random bits each.
The label is another 128 bits long,
	and as random as possible,
		like the keys.
Instead of being on a steel ring,
	these keys,
		with the keyring label,
			are written to a file on disk;
				a blob of 100 strings of 128 bits.
\par
The use of keys on this keyring is not entirely different from using real physical keys.
The bits in a key are comparable with the tooths or holes of a physical key you use to unlock your home.
You do not need to remember exactly how far the tooths need to protrude or exactly where and how deep the holes in your key need to be,
	to be able to unlock the door;
		you just select the right key
			(the whole physical thing at once, with all the right tooths or holes)
				by recognising its form or its label.

\subsection{Computers and links}
This proposed way of logging in concentrates on computing values and communication of the results.
It is therefore appropriate at this point to elaborate a bit on who is communicating with whom and why.
\subsubsection{Computer roles}
Four computer roles can be distinguished when logging in.
\begin{description}
\item[Webserver]
	This is the front-end server to which the user is communicating.
	From our point of view,
	it is the central system,
	acting as a hub.
	It communicates with three other servers:
	the accounts server, the login server, and of course the client computer.
	We consider it the most vulnerable to `visits' with malicious intent,
	so it has been appointed the most simplests of tasks.
	It does not compute anything remotely valuable,
	it just compares values provided by others.
\item[Accounts server]
	This server is typically located in a network only accessible by the webserver.
	It stores all kind of user related account data, but no userid or passwords;
	it stores hashes and site keys instead.
	Like the webserver it does not compute anything either.
\item[Login server]
	This server uses an HSM which stores an array of secret keys per website it services.
	This server computes nothing on its own,
	it lets the HSM do all the work secretly,
	using some of its finest cryptographic functions.
	It receives login requests from the webserver and returns values that the webserver can compare or relay.
\item[Client computer]
	With this device the user communicates with the webserver.
	It generates random numbers and uses \XOR\ and \SHA\ to compute the values exchanged with the webserver.
\end{description}
These roles can be played by as many computers as there are roles,
but that need not always be the case.
In fact,
each and every role may be combined;
a single computer may perform them all,
for that matter.
\par
The accounts server and the login server may be combined in a single server in the back-office.
The webserver and the accounts server may be combined,
with the login server somewhere on the Internet.
The combination of webserver, accounts server, and login server is also valid;
although this concentration of roles is not highly preferred.
\subsubsection{Secure links}
Traditionally,
the communication channel between the user and the webserver is secured by means of TLS.
The user sees the URL that starts with ``https://'',
which means that the webserver has authenticated itself using a digital certificate.
The browser will validate the supplied certificate using other trusted certificates.
This security measure is sufficient but required.
\par
The communication between the webserver and the accounts server should be conducted over a secure channel as well.
\par
Finally,
the communication between the webserver and the login server needs to be encrypted.
For this,
TLS is a sufficient security measure,
although in this case mutual authentication is a must.
The webserver needs to know it is talking to a legitimate and trusted login server,
and the login server needs to know it is receiving valid requests.
The certificate of the websever is used to select which array of secret keys to use,
as the login server may service more than one webserver.
\subsection{Logging in}
\subsubsection{Part I}
Before starting the login process you select a key from your keyring;
the one you know to belong to the website you are trying to login to.
\par
To login you don't send your key to the server but generate a secret random value,
which you encrypt with your key using a simple \XOR\ operation.
Your key is totally random and the secret random value you encrypt with it is also, well\ldots totally random.
Mixing these random bits gives another totally random value.
The website will receive this value and will try to decrypt this with the key that belongs to your account
(we will keep you in the dark for now, about how this works).
When it has decrypted the random login value,
it will know which secret random value you generated.
Instead of telling you the secret random number,
the website sends a hash value of it,
because otherwise someone able to see the network traffic will instantly know your key.
\par
If the website returns a hash value matching your secret random number, you know two things.
First, you know that the website you are talking to can successfully decrypt your random value.
It can only do this if it has a matching key.
Second, you know you are talking to the same site that sent you your key when you applied for an account.
This implies that if you trusted that site then, you can trust this site now, as it can only be the same site.

\subsubsection{Part II}
This is all great and very sophisticated, but the website wouldn't dare share your wonder about this,
as you may well fake your enthusiasm, and flatly lie about the correctness of the hash it has sent you.
To see if your claims hold,
the website will do the same as you and send you an encrypted secret random value.
This cryptogram can only be decrypted by someone owning the right key for the account.
Using \XOR\ with the key you have originally selected from your keyring,
you decrypt it.
Then,
using \XOR\ again,
you combine the two random values you now have,
and return this value to the website.
\par
When the website receives a value matching its secret random number, it knows two things.
First, the user on the other side can successfully decrypt the secret random number.
It can only do this if it has a matching key.
Second, the website knows that this key is from the same keyring that was used when applying for an account.
This implies that if you trusted this user then, you can trust this user now, as it can only be the same user.

\subsection{Keys}
\subsubsection{On site keys and user keys}
Beware, the icky parts start here (a bit).
\par
When applying for an account, a site key is created by a random generator.
The user key is then computed as the encryption of the site key with a Secret Key, with a block cypher like \AES.
\par
When the user encrypts its secret random value with its key,
it uses \XOR.
This is a very simple and reversable encryption method.
But as both key and value are utterly random, no information is stored in the encrypted value.
The website needs to decrypt the value, and this can only be done with the user key.
But the website does not store user keys, only site keys\ldots
\par
The solution to this may be a bit of a dissappointment and a cheat:
the user key is temporarily regenerated by encrypting it again with the Secret Key.
Once the user key is available, the random value can easily be decrypted by \XOR-ing it.
Also, the cryptogram that is sent to the user is a random value \XOR-red with this regenereated user key.
\par
The clever part,
however,
is that all the crpytographic functions the website is required to perform take place inside a Hardware Security Module,
or HSM for short.
Secret Keys can be put in the HSM and made to good use there,
but can never be retrieved from it.
The HSM computes all values needed for the login process,
without ever revealing the keys that are used,
not even to a hacker with full control of the HSM.
\par
In the end,
only random bits enter the HSM,
and only random bits leave it.

\subsubsection{Key lifetime}
Every key has its lifetime, so Secret Keys need changing every so often, as a good security measure.
The same goes for keys on the user's keyring.
\par
The login protocol is capable of changing keys on the website and keys on the keyring,
without any effort on the user side.
Well\ldots, a little program will do it for you, without you noticing.
\par
Logins can be granted,
but new keys may not be,
which results in the expiration of an account.
A website may deny a user new keys when payment is due,
giving users limited login rights.
At any time new keys can be provided again.
\par\vspace{10mm}\noindent
To learn how this all works,
how keys are used,
what a website can do with logins,
and more,
can all be read in the next chapters.
