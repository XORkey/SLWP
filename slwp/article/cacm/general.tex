\section{Login for dummies}
Not that you are one if you decide to read this section.
It serves as a general description of the login process,
without going into technical details too much.

\subsection{What are we talking about?}
Instead of choosing a userid and a password we let the computer work things out for us instead.
Userids tend to be in short supply for people with surnames like John, Peter, and Chris,
or last names like Jones or Smith.
But the larger the site, the rarer free userids.
\par
Passwords tend to be weak,
as people choose short, real-life words, like things or names.
Not that people are not willing, but remembering too long, too difficult passwords is not easy.
It is very frustrating when you cannot login because you have forgotten your password.
\par
As a radical measure, we do away with all this!
We will let the computers
(yours and the one running the website)
do what they do best: compute.
Give them some random bits to chew on and they are in heaven.
\subsection{Random values}
The website holds no userids, but hashes of keyring identifiers.
\par
The website holds no passwords, but one part of a key, of which you have the other part.
Your key is different from the key the website has, but they are related to each ohter.
It is not possible to login with keys that are stored on the website;
you can only login with your key, which is verified with the key from the website.
And oh..., keys are created using a random generator.
Keys are just random bits with no information whatsoever.
\par
Keys are kept on a keyring and have ordinal key numbers from 1 to 99 for easy recollection.
\subsection{Loging in part I}
Before starting the login process you select a key from your keyring.
To login you don't send your key to the server but a secret random value,
which you encrypt with your key using a simple XOR operation.
Your key is totally random and the secret random value you encrypt with it is also, well..., totally random.
Mixing these random bits gives another totally random value.
The website will receive this value and will try to decrypt this with the key that belongs to your account
(we will keep you in the dark for now, about how this works).
When it has decrypted the random login value, it will know which secret random value you generated and XOR-red with your key.
Instead of telling you the secret random number it sends a hash value of it,
because otherwise someone able to see the network traffic will instantly know your key.
\par
If the website returns a hash value matching your secret random number, you know two things.
First, you know that the website you are talking to can successfully decrypt your random value.
It can only do this if it has a matching key.
Second, you know you are talking to the same site that sent you your key when you applied for an account.
This implies that if you trusted that site, you can trust this one also, as it can only be the same site.
\par
This is all great and very sophisticated, but the website cannot share your wonder about this,
as you may fake your enthusiasm, and lie about the correctness of the hash it has sent you.
\subsection{Loging in part II}
To put your money where your mouth is,
the website will do the same as you and send you an encrypted secret random value.
This cryptogram can only be decrypted by someone owning the right key for the account.
Using XOR with the key you have originally selected from your keyring,
you decrypt it and return a hash of this value to the website.
\par
When the website receives a hash value matching its secret random number, it knows two things.
First, the user on the other side can successfully decrypt the secret random number.
It can only do this if it has a matching key.
Second, you know that this key is from the same keyring that was used when applying for an account.
This implies that if you trusted this user then, you can trust this user now, as it only can be the same user.
\subsection{On site keys and user keys}
Beware, the icky parts start here (a bit).
\par
When applying for an account, a site key is created by a random generator.
The user key is then computed as the AES encryption of the site key with a Secret Key.
\par
When the user encrypts its secret random value with its key it uses XOR.
This is a very simple and reversable encryption method.
But as both key and value are random, no information is stored in the encrypted value.
The websites needs to decrypt the value, and this can only be done with the user key.
The website does not store user keys, only site keys.
\par
The next part may be a bit of a dissappointment:
the website temporarily regenerates the user key by encrypting it again with the Secret Key.
Once it has the user key, it can decrypt the random value easily by XOR-ing it.
Also, the cryptogram that is sent to the user is a random value XOR-red with this regenereated user key.
\par
The clever part, however, is that this all takes place inside a very safe hardware device called an HSM.
Secret Keys can be put in the HSM and made to good use there, but can never be retrieved from it.
The HSM computes all values needed for the login process,
without ever revealing the keys that are used.
In the end,
only random bits enter the HSM,
and only random bits leave it.
\subsection{More about keys}
Every key has its lifetime, so Secret Keys need changing every so often.
The same goes for keys on the user's keyring.
The login protocol is capable of changing keys on the website and keys on the keyring,
without any effort on the user side.
\par
Login can be granted,
but new keys may not be,
which results in the expiration of an account.
A website may deny new keys when payment is due,
giving users limited login rights.
At any time new keys can be provided again.
\par
\par
To learn how this all works,
how keys are used,
what a website can do with logins,
and more,
can all be read in the next chapters.
