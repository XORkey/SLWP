%
% $Header: /usr/local/src/slwp/article/keys.tex,v 1.5 2013/03/09 15:32:54 timo Exp $
%
% AUTHOR:	ing. T.M.C Ruiter
%
\section{Keys}
%The exchanged values in the login process and the keys in a keyring are 128 bits long in this article,
%but could be taken much smaller;
%say,
%16 bits each,
%without doing great injustice to the level of security.
%The use of 128 bits is only a convenient size,
%as it is the \AES\ blocksize, and \SHA\ works with this length naturally.
%This way,
%we take full advantage of their cryptographic power.

\subsection{Secret keys}
\label{sec:secret_keys}
The keys used to encrypt other keys%
---called Secret keys in this article---%
are used to perform \AES\ encryptions of site keys,
and are 256 bits long.
\par
Several Secret keys are used and
are considered to be stored in array $S$, in this article.
With this array, three values are defined.
\begin{enumerate}
\item	The number of stored Secret keys is represented by $m$.
\item	The constant%
\footnote{Although declared a constant, the value of \texttt{MAX\_KEYS} may change over time.
When the login policy regarding the use of old keys is changed, more or fewer old keys will be stored.
For this, \texttt{MAX\_KEYS} needs to be changed.}
\texttt{MAX\_KEYS}, which is the maximum number of keys that will be kept.
The value of $m$ starts at 0 and will reach \texttt{MAX\_KEYS} eventually and grow no more.
The value \texttt{MAX\_KEYS} has a minimum of 2 (a new key and at least 1 old key).
All values in the array $S$ are shifted up one position when a new key is stored, which will always be inserted at $S[0]$.
This way, with at least one key loaded ($m>0$), it always holds that $S[0]$ is the newest key,
and $S[m-1]$ is the oldest key.
With $m>1$, $S[1]$ is the jongest old key.
\item	The constant \texttt{MAX\_ACTIVE\_KEYS},
which is the maximum number of keys that will be used for logging users in.
This value lies in the range $[2, \mathtt{MAX\_KEYS}]$.
\par
Subtracting \texttt{MAX\_ACTIVE\_KEYS} from \texttt{MAX\_KEYS}
gives the number of inactive keys which can be used to reactivate expired user keys.
Users using a user key that is older than the oldest active Secret key cannot login,
but their key can be restored with an inactive Secret key.
If a key is used that is older than the oldest inactive key,
the key cannot be restored and is lost forever.
\end{enumerate}

\subsection{Site keys}
For each user, there exists a site key and a derived user key (see section~\ref{sec:userkeys}).
The site keys are stored in a table of a database,
where the \SHA\ hash of the userid will act as the primary key for that table.
Since the user only sends the \SHA\ hash of its userid, no actual userids will be stored.
\par
Site keys are generated once by a pseudo-random generator, and need never be replaced.
\par
Each time a user tries to login,
the database is queried with the \SHA\ hash of the userid,
and the site key for that user is returned for further processing.

\subsection{User keys}
\label{sec:userkeys}
A user key is computed by encrypting the site key for the user with the joungest (or current) Secret key,
using the \AES\ algorithm.
\par
The keys users get from the website are stored on a keyring (see section \ref{sec:keyring}).
These keys are automatically changed for new keys by the website when the Secret key is changed.

\subsection{Key lifetime and old keys}
For enhanced security, Secret keys need to be replaced at regular intervals.
This change of keys is initiated by the website,
without the possibility to make individual arrangements with any of its users.
\par
After changing the Secret key, it is no longer possible to login with any user key.
Therefore, an array of Secret keys needs to be kept, allowing users to login using an old(er) Secret key.
At least one old Secret key needs to be stored; otherwise, the Secret key cannot be changed without rendering all user keys permanently useless.
\par
Upon successful login with an old key, the user will be provided with a new user key,
with which it can login using the newest Secret key, from that moment on.
Changing of user keys is seamless and the user might, just as well, be kept unaware of this.
\par
The website's security policy should prescribe with what frequency Secret keys are to be changed, and how many old keys should be kept.
If, for instance, the Secret key is changed every month, and 11 old keys are kept, users can still login using a key that is a year old.
If the user key is older than that, the site is not able to verify the user's key any longer.

\subsection{Storing user keys in a keyring}
\label{sec:keyring}
Keys for the user are collected and stored in a keyring.
The keyring is a block of
(at least)
100 random numbers with 128 bits, most of them dummy keys.
The keyring will be considered an array $Z$ in the rest of the article.
Since all bits are entirely random in a keyring%
---for dummies and real keys---%
a keyring stores no information whatsoever.
\par
Real keys in the keyring should be randomly put between the dummy keys.

\subsubsection{Creating a keyring}
A keyring is created by using a random generator of sufficient quality,
which generates 100 random numbers of 128 bits.
Key number zero is used to identify the keyring and will never change after its creation.
The other 99 keys are dummies
(random bits with no meaning whatsoever).
These random numbers are then encrypted and written to a file%
\footnote{A password protected PKCS\#12 file would do nicely.}
on a storage device.
Unencrypted values should never be written to a storage device and only be kept in volatile memory.
It is up to the user to remember how to decrypt its keyring.

\subsubsection{Copying keys and keyrings}
A keyring can be freely copied to other devices, so all keys are available there as well.
The keys of one keyring can be copied to other keyrings, in different locations.
It is up to the user to keep a registration of this.
\par
Encrypted keyrings can be stored in a public place, for easy access, and act as a backup.
A dedicated webserver can be employed for storing encrypted keyrings,
which can be used to restore lost keyrings.
They can also be used to login when away from home with no access to your own keyring.
These temporary keyrings should be discarded when logging out.

\subsubsection{Irretrievably lost keys and keyrings}
When a key number is forgotten and no record of it can be found,
the key must be considered lost.
\par
As the user has supplied websites with personal information,
it may be easy to regain login capabilities by just asking for a new key,
provided a username and maybe some other information can be given.
Selecting a free keynumber on the keyring and replacing that key with the new one should restore the ability to login.
\par
Keyrings can become unusable or lost in two situations:
the device holding it is defective (like a harddisk crash) and no backup has been made,
or the encryption cannot be reversed (PIN forgotten).
In both cases, the user has to start over and create a new keyring, containing only dummy keys.
