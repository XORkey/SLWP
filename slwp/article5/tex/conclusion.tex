\section{Conclusions}
The security of the login process for websites can be greatly improved by using a keyring at the user side and an HSM employed by the website.
Instead of sending relatively short, easy to guess strings (passwords) over the line,
the use of encrypted random values,
up to 128 bits each,
is a big improvement.
No keys are sent,
just random values,
which will be different each time a user logs in.
\par
For hackers,
getting login data in huge numbers will be very difficult,
since this data is no longer stored centrally,
but split between website and user.
Each part alone has no value,
and all values stored at the website render no valid login data without Secret keys.
These keys are kept in very secure hardware: an HSM.
Sniffing network traffic
or collecting keystrokes with Trojans will not help.
Keyrings have very high entropy and are encrypted,
so to no direct use to hackers as well;
they may be stored on the web for easy access and backup.
\par
Several important security measures are automatically implemented:
keys are changed frequently
(as frequent as Secret keys are changed),
they differ for each website,
and keys on a keyring and the knowledge which key is used for what site constitutes two-factor authentication.
\par
For the user,
the way to logon to websites will change,
but it will be an improvement over the burden of keeping track of all passwords.
One userid for all sites and a single key number for a website is all you have to remember.

\subsection{Advantages}
In the following cases a keyring is superior to the use of passwords.
\begin{itemize}
\item Websites have all login information stored centrally.
If an hacker can obtain this data and decrypt it, it has access to all---possibly millions---accounts at once \cite{wiki:linkedin}.
\par
\emph{Using the keyring system there is no usable login information whatsoever at the server hosting the website.
There is no way that the user information that \emph{is} present yield any usable login data.
Hacking a website to obtain logins is useless.
\linebreak
Other reasons to hack websites will remain, however, and using keyrings does not prevent hacking; it just eliminates one of the major attractions.}
\item Users tend to have the same password and the same login name for several websites.
A hack of an insufficiently protected site could yield valid usernames and passwords of perfectly protected sites.
(Hack of www.babydump.nl yields at least 500 valid logins for www.kpn.nl.)
\par
\emph{Even if all user keys for a website were obtained in a hack, these would be useless for any other website, since they differ by definition.}
\item Websites require the user to change passwords.
As more websites do this, more and more passwords a user has to remember, change.
Ideally, no password for a website should be the same as for another website, but that is impractical.
This would mean that each and every password needs to be written down, because the number of passwords is too much to remember for most.
This thwarts the principle that passwords need to be remembered and never written down.
The requirements to change passwords frequently and that they should differ from any other password is an inhuman task.
\par
\emph{Using the keyring system, keys will differ for each website by definition and change regularly and automatically.
Key numbers (the ordinal numbers in the keyring) don't change, so most of them can be remembered by the average human.}
\item Sometimes, getting unauthorized access to an account is as simple as just looking at the keyboard to see what the password is.
The userid is always displayed when logging in, so shoulder surfing is very effective.
\par
\emph{Using a keyring, shoulder surfing cannot be used directly to login.
Since a keyring is something you have to have, you cannot login using only the userid and the key number.
You need to have access to the (unencrypted) keyring as well.
Therefore, using a keyring is a basic form of 2-factor authentication.}
\item People tend to use weak passwords (unless a website specifically enforces the use of strong passwords) which can be guessed using specialized tools.
If that yields no success, brute force attacks can be launched; to just try all possible passwords with limited length.
\par
\emph{Password guessing nor brute force attacks are an option when trying to login, since no passwords of any kind are exchanged.
Even if the keys themselves would be used as an old-fashioned password, the search area would encompass $2^{128}$ or $3.4\cdot 10^{38}$ equally likely possibilities.
Trying 1 million possibilities per second it would still take $10^{32}$ seconds to try them all.}
\item The validity of the connection to websites is built on trust.
HTTPS connections are protected using certificates.
Sometimes trust only goes so far, and bogus but valid website certificates are used (Dorifel virus) or even the Root CA certificates are forged (see the DigiNotar hack).
In that case, the user's trust is betrayed and the user left helpless.
\par
\emph{With the keyring solution no standalone substitute websites can exist; login data must be redirected to the real website.
A substitute website does not have the right Secret keys.
A user will notice this by wrong answers from the website and the login will abort from the user side.}
\item Visitors of websites are lured to other, well built fraudulent websites, mimicking websites of banks and such (phishing).
Here, a simple mail can give a lot of trouble, redirecting users to an unsecure copy of a website, without the user suspecting anything.
\par
\emph{All communication to and from the malicious website can be passed on to the real website, to give the user the sense it is talking to the real site.
Obtaining valid login data this way (as a man in the middle) is useless, since no keys are sent over the line.
The data that is used to validate a user is meaningless for the next login.}
\item People are sometimes called by other people, claiming to be employees of banks.
In order to ``help'' solve a problem, users are asked to give their login credentials.
Some ignorent users are willing to oblige.
\par
\emph{With a keyring the only thing slightly useful to a hacker would be the userid.
The keynumber is of no use,
since the hacker has no access to the keyring itself;
telling him which key index is used to login has no value.}
\end{itemize}

\subsection{Acknowledgements}
Special thanks to Martijn Donders for his cryptographic support, and Rob Bloemer for reviewing the first draft of this article.
