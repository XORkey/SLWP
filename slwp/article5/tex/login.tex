%
% $Header: /usr/local/src/slwp/article/tex/login.tex,v 1.8 2013/10/26 17:51:28 timo Exp $
%
% AUTHOR:	ing. T.M.C. Ruiter
%
\section{Logging in}
\label{logging_in}
Having unencrypted its keyring, selected a key number ($k$), and entered the userid, the login process can commence.
Keys used in this scheme may vary in length per website.
We will use the value $n$ to indicate the length of the site and user keys.
See \vref{sec:key_length} for a discussion of key lengths.
\subsection{User IDs}
\label{sec:user_ids}
Instead of sending a traditional alphanumeric userid,
we send a hash.
This hash will be used by the webserver to identify the user,
and the webserver will only store hashes in its user database.
This hash is specially crafted,
so guessing will be hard.
\par
The user will still need a userid,
but this userid may be the same%
\footnote{Different userid's for different websites are still possible, but not necessary.
The chosen userid acts like a default userid, or as a password for the keyring, if you like.
The browser may remember the (default) userid for the user.}
for each and every webserver.
The userid is chosen once and can remain the same as long as the keyring exists.
The hash will be constructed using a combination of the userid and the keyring identifyer $Z[0]$.
The hash value that is sent is constructed as follows.
Compute the hash of the keyring identifier:
\[H_0=\F{\SHA}{Z[0]}\]
Then replace the most significant bits of $H_0$ with the bit representation of the userid
(a simple string like 'John Doe')
to get $H_0'$.
Then, compute the final 256-bit hash:
\[U_h=\F{\SHA}{H_0'}\]
\par
This hash value is a combination of something the user has
(the keyring)
and something the user knows
(its chosen userid).
This makes it hard to match a stolen set of hashes from a webserver with a set of keyrings from a keyring server.

%%%%%%%%%%%%%%%%%%%%%%%%%%%%%%%%%%%%%%%%%%%%%%%%%%%%%%%%%%%%%%%%%%%
\subsection{Applying for an account}
\label{sec:applying}
When a new user applies for an account,
it presents the webserver with its hash value $U_h$.
Furthermore,
the user should select the index ($d$) of a hitherto unused key
(therefore, a dummy).
\[K_d=\mod{Z[d]}{2^n}\]
The user will send $U_h$, along with its dummy key $K_d$ to the webserver.
\par
The webserver will request values for a new user from the login server,
by forwarding the dummy ($K_d$) and
specifying for which website ($W$) the user will be granted access.
The login server will do the following:
\begin{enumerate}
\item First,
	lookup a pre-shared key $K_w$,
	which belongs to the website $W$.
\item Generate a $n$-bit pseudo-random site key $K_s$ for the user:
\[K_s=\F{Random}{n}\]
\item Encrypt the $K_s$ with the current Secret key $S[0]$ to yield the new user key $K_u$,
which is then encrypted with the users dummy key $K_d$:
\[K_u=\F{\AES}{K_s, S[0]}\]
\[K_x=\xor{K_u}{K_d}\]
\item Finally, the two new values are encrypted with the pre-shared key $K_w$:
\[E_s=\F{\AES}{K_s, K_w}\]
\[E_x=\F{\AES}{K_x, K_w}\]
And both are returned to the webserver.
\end{enumerate}
The webserver will then relay the new keys $E_s$ and $E_u$, and $U_h$ to the accounts server.
The accounts server will decrypt those values with it's own pre-shared key $K_w$:
\[K_s=\F{\AES}{E_s, K_w}\]
\[K_x=\F{\AES}{E_x, K_w}\]
It will then store the combination of $K_s$ and $U_h$ with the other data it already has for the user,
and the encrypted user key $K_x$ is sent to the user through a separate channel,
like email.
\par
The user can import the new key $K_u$ into the keyring by calculating
\[K_u=\xor{K_x}{K_d}\]
Finally, with something like
\[Z[d]=\xor{(\F{Random}{128-n}\ \mathbf{<<}\ n)}{K_u}\]
the dummy key $K_d$ at $Z[d]$ is overwritten with 128 random bits,
of which the last $n$ bits contain the new key $K_u$.

\subsection{Basic login scheme}
\label{sec:basic_login}
The procedure described below to login is the basic login scheme.

%%%%%%%%%%%%%%%%%%%%%%%%%%%%%%%%%%%%%%%%%%%%%%%%%%%%%%%%%%%%%%%%%%%
\subsubsection{Step 1: Knock, knock\ldots}
\label{sec:login_step1}
The user's login program has to compute the folowing:
\begin{enumerate}
\item An $n$-bit pseudo-random number $R_u$ which hash will be returned by the webserver:
\[R_u=\F{Random}{n}\]
\item The user key $K_u$ is taken from the keyring $Z$ at index $k$.
As this is 128-bit value,
take the least significant $n$ bits of it.
The pseudo-random number is encrypted with it to get the value $A_u$:
\[K_u=\mod{Z[k]}{2^n}\]
\[A_u=\xor{R_u}{K_u}\]
\item The webserver will return a hash value of the user's pseudo-random value.
To be able to verify this hash,
we compute our own hash $B_u$ to verify it with:
\[B_u=\sha{R_u}\]
\end{enumerate}
The special userid hash $U_h$ and the encrypted pseudo-random number $A_u$ are sent to the webserver.

%%%%%%%%%%%%%%%%%%%%%%%%%%%%%%%%%%%%%%%%%%%%%%%%%%%%%%%%%%%%%%%%%%%
\subsubsection{Step 2: Site key lookup and key matching attempts}
\label{sec:login_step2}
The webserver runs a login procedure,
see Algorithm~\vref{alg:webserver_login}.
\par
After receiving $U_h$ from the user, the site key $K_s$ needs to be looked up.
A query with $U_h$ is sent by the webserver to the accounts server.
If a match is found, $K_s$ is returned;
otherwise, $K_s$ is set to zero, indicating that no such record exists.
In the latter case, the values $B_s$ and $P_s$ are both set to zero as well, and returned to the user.
The user needs to rethink its actions and start over.
\par
If a match is found,
an iterative process starts,
trying to find the right Secret key to log the user in.
%The loop runs at most $min(m,\mathtt{MAX\_ACTIVE\_KEYS})$ times.
The webserver will send three values to the login server:
$A_u$,
$K_s$,
and a Secret key index $i$,
starting at 0.
This will be repeated
(increasing index $i$),
until a match is found,
or no more keys can be tried.

%%%%%%%%%%%%%%%%%%%%%%%%%%%%%%%%%%%%%%%%%%%%%%%%%%%%%%%%%%%%%%%%%%%
\subsubsection{Step 3: Login server actions}
\label{sec:login_step3}
The login server is a processor of login values and calls a function of the HSM to compute them
(see Algorithm~\vref{alg:hsm}).
\par
In case $i$ is less than $m$
(the number of stored Secret keys, see section~\vref{sec:secret_keys})
the login server calculates the following, all within the HSM:
\begin{enumerate}
\item Temporarily, regenerate a user key, using the same algorithm as when the key was originally generated:
\[K_u=\F{\AES}{K_s,S[i]}\]
with $S[i]$ the $i$-th Secret key stored in the HSM.
\item With the user key $K_u$, decrypt the pseudo-random the user has sent:
\[R_u=\xor{A_u}{K_u}\]
\item Calculate a hash with which the user can verify we own the site key $K_s$ and a corresponding Secret key $S[i]$:
\[B_s=\sha{R_u}\]
\item To be able to verify that the user owns and knows its user key $K_u$,
is not sending a replay of some earlier succesfull login sequence,
and will be unable to ly about the correctness of the hash of the pseudo-random the webserver will send,
we calculate a challenge for the user.
\par
If the index $i$ equals zero, generate a pseudo-random value
\[R_s=\F{Random}{n}\]
otherwise, compute
\[R_s=\F{\AES}{K_s,S[0]}\]
which will be the new key for the user.
\par
We then encrypt $R_s$ to a value $P_s$ the user can decrypt using its key:
\[P_s=\xor{R_s}{K_u}\]
This value will be different for each time the loop is executed,
even if the same new key is transmitted,
since $K_u$ will change each time.
\item Calculate the expected response also:
\[Q_s=\sha{R_s}\]
\end{enumerate}
The HSM will produce the values $B_s$,
$P_s$,
and $Q_s$ as a result of the calculations and the login server will send them back to the webserver,
as an answer to the three values it was given ($A_u$, $K_s$, and $i$).
The HSM will delete all temporary values from memory directly afterwards, and remember only its Secret keys.
\par
Should index $i$ equal $m$, then the array of Secret keys is exhausted.
If we reach this situation, we cannot log the user in since all possible attempts have failed.
Return zero values for $B_s$, $P_s$, and $Q_s$ to indicate this.

%------------------------------------------------------------------
\subsubsection{Step 4: user verifies site key}
\label{sec:login_step4}
The webserver sends $B_s$ and $P_s$ to the user.
If both are zero, the login has failed and the user should return to step 1.
It is advisable for the user to choose different values for the next try.
\par
If $B_s$ contains a nonzero value, the user verifies whether this value equals $B_u$.
If the two values do not match,
i.e.:
\[B_u \neq B_s\]
the user either has selected the wrong key or uses an old key.
A key can be old if it comes from a keyring that was not used recently.
Another possible reason for a key to be old is when the Secret key of the site has changed recently.
\par
To indicate that no match has been found, we send
\[Q_u=\mathtt{0x0}\]
($n$ zeroes) to the webserver.
The webserver will need to start over, and send values $A_u$, $K_s$, and $i+1$ to the login server.
So, back to step 3.

%------------------------------------------------------------------
\subsubsection{Step 5: site verifies user key}
\label{sec:login_step5}
If $B_s$ matches $B_u$, then the webserver has found a Secret key for the user.
The user can now prove it has control over its user key by computing
\[R_s=\xor{P_s}{K_u}\]
and
\[Q_u=\sha{R_s}\]
which is sent to the webserver as proof.
If the webserver accepts $Q_u$ as correct it will log the user in.
\par
If the webserver receives a value $Q_u$ that does \emph{not} match,
something fishy is going on.
Further attempts for this account, or from that source should be scrutinized.

%------------------------------------------------------------------
\subsubsection{Step 6: user key replacement}
\label{sec:login_step6}
If this attempt to login was not the first with this key,
the webserver has sent the user a new key in $R_s$ (instead of a pseudo-random value).
The user should store this new key in the keyring, overwriting the old key the user has just used:
\[Z[k]=\xor{(\F{Random}{128-n}\ \mathbf{<<}\ n)}{R_s}\]
where all bits of $Z[k]$ are replaced.

%%%%%%%%%%%%%%%%%%%%%%%%%%%%%%%%%%%%%%%%%%%%%%%%%%%%%%%%%%%%%%%%%%%
\subsection{Changing Secret keys}
At regular intervals a new Secret key is introduced and,
at the same time,
an old Secret key may be sent to the Eternal Hunting Grounds.

%------------------------------------------------------------------
\subsubsection{Terminated accounts}
When deleting the oldest Secret key from memory,
all accounts with a `last login' date older than the installation
date of the second oldest Secret key should be marked `terminated'.
Removal of the oldest key renders all those keys unusable.
Those accounts can be purged, as they cannot be used again.

%------------------------------------------------------------------
\subsubsection{Expired accounts}
In some situations user access must be barred until some condition is met
(see also \vref{sec:conditional_key_replacement}).
Accounts can be made temporarily inaccessible for this purpose by letting keys expire.
Expiration can happen automatically when users do not login in time to get their keys replaced,
or intently by denying key updates.
Expired user keys are associated with Secret keys with an index in the range $[\texttt{MAX\_ACTIVE\_KEYS},\texttt{MAX\_KEYS})$.
\par
Expired accounts can easily be made active again by using inactive Secret keys for the login process (and then renew the key).
\par
Expired accounts can be reinstated,
but only before the associated Secret key is erased from memory.
After that, there is no way to regenerate the user key for logging in.
The user should apply for a new account if it wants to regain access.

%%%%%%%%%%%%%%%%%%%%%%%%%%%%%%%%%%%%%%%%%%%%%%%%%%%%%%%%%%%%%%%%%%%
\subsection{Changing site keys}
What is true for Secret keys is also true for site keys:
regular replacement improves security.
Where new Secret keys automatically yield new user keys, a new site key does not.
\par
The user may request a new key from the webserver,
but this is like applying for a new account.
All old keyrings will be useless for this account when the new key is put on it.
You will get a new key that is valid with the current Secret key $S[0]$.
Using an old user key will not match anything,
so you cannot use it any more.

%%%%%%%%%%%%%%%%%%%%%%%%%%%%%%%%%%%%%%%%%%%%%%%%%%%%%%%%%%%%%%%%%%%
\subsection{Optimizations and enhancements}
With the basic login scheme several optimizations and alternatives are possible.
Here are some obvious ones.
%------------------------------------------------------------------
\subsubsection{Aborting the login procedure}
Logging in with an old key every now and then is inherent in this scheme,
as most Secret keys will be changed at regular intervals.
Therefore,
it is likely to sometimes receive a hash value $B_s$ that does not match $B_u$.
An unsuccessful second attempt may be the result of using a seldomly used key or a wrong key.
So, after at least two unsuccesful attempts
the user may be presented a question whether it likes to select a different key from the ring,
or continue trying with this key.
If the user opts to select a different key, the login procedure has to be aborted.
This may be done by setting
\[Q_u=\mathtt{0x1}\]
and send this to the webserver
(instead of $\mathtt{0x0}$).
We return to step 1.

%------------------------------------------------------------------
\subsubsection{Using last login date}
Instead of always starting with Secret key $S[0]$,
the date of the last login of the user can be taken into account.
If the query with $U_h$ as key,
that is sent to the accounts database,
would also return the last login date,
a starting key index $i$ could be selected.
It is no use trying new keys when you know beforehand that these keys were added after the last successful login.

%------------------------------------------------------------------
\subsubsection{Conditional key replacement}
\label{sec:conditional_key_replacement}
For most websites, restoring user keys that refer to the most recent Secret key $S[0]$ will be done without hesitation.
For some,
refreshing keys will be done only when certain conditions are met,
like payment of monthly fees,
a certain number of reviews written,
or some amount of data uploaded.
Until then,
logging in with a valid
(but in this context a typically `old')
key is granted.
But if, for instance, payment is overdue, the key will expire and logging in is no longer possible.
\par
Instead of incrementing index $i$ in step 2,
the index is decremented.
The HSM will just be sending random values for $P_s$ in that case, for all values of index $i$.
To indicate to the user that no new key is sent,
the website will replace it with $A_u$
\[P_s = A_u\]
and return this value to the user.
In this case,
no keys should be decrypted or overwritten.

%------------------------------------------------------------------
\subsubsection{Certifying the login server}
Allowing the user to verify it recieves trustworthy responses from the webserver,
the webserver may present the user with a certificate,
stating that it is using a bona fide login server.
This certificate---signed by a regular Root CA---contains a public encryption/decryption key $K_D$.
The private key for the login server ($K_E$) will be stored in its HSM. 
\par
Before sending $B_s$ in step 3,
encrypt it with $K_E$:
\[B_s'=\F{\RSA}{B_s, K_E}\]
\par
In step 4,
before comparing $B_s$ with $B_u$,
this value has to be decrypted using the public key $K_D$ from the login server's certificate.
So, if $B_u$ equals
\(\F{\RSA}{B_s', K_D}\)
we proceed to step 5.

%------------------------------------------------------------------
\subsubsection{Encrypting $K_s$}
The site key $K_s$ is passed to the login server as-is.
Although dispatched over a secure channel, it is a key which could do with a bit of protection.
When we have a certificate of the login server,
which we present to the users,
the webserver also can make use of it by encrypting $K_s$ with the key $K_D$.
Only the HSM in the login server holds the other key of the pair
($K_E$)
with which this value can be decrypted.
\par
By computing
\[K_Y=\F{\RSA}{K_s, K_D}\]
and sending this value to the login server,
instead of plain $K_s$
we have secured the site key.
\par
The login server decrypts value $K_Y$ by computing
\[K_s=\F{\RSA}{K_Y, K_E}\]
and using $K_s$ in subsequent calculations.

%%%%%%%%%%%%%%%%%%%%%%%%%%%%%%%%%%%%%%%%%%%%%%%%%%%%%%%%%%%%%%%%%%%
\subsection{Simplifications}
The basic login scheme
(starting with section \vref{sec:basic_login})
is complete and offers the whole set of security measures you would want for a safe login scheme.
Several measures are taken to establish this level of security.
This section explores what happens when you leave them out.
\subsubsection{Using traditional userids}
Instead of using a value dependent of $Z[0]$,
you could stick to the old fashioned userid to indicate as which user you want to log in.
You will need a userid that is unique for the website.
What's more:
you will need a userid,
period.
\subsubsection{Getting rid of the HSM}
\begin{dialogue}
\speak{Jen}		\direct{Holding an invoice.}
				That's quite an expensive set of toys you have there, lads, your\ldots
				\direct{Reads.}
				H-S-Ms.
\speak{Moss}	Thank you for taking an interest!
				For a whole week I was wondering if you'd notice.
\speak{Jen}		Why on earth did you order two of those?
\speak{Moss}	One is none, I always say.
				\direct{Pauses.}
				I keep wondering why this sounds contradictory\ldots
\speak{Roy}		Could we not have three? We all would sleep better. Moss?
				\direct{\refer{Moss} nods.}
\speak{Jen}		Could we do without, instead?
				\direct{With a firm look,
				\refer{Jen} strides purposefuly towards the apparatus,
				with her her hand reaching for the power plug.}
\speak{Moss}	\direct{Looks up in horror.}
				Impossible!
\speak{Roy}		\direct{Lunges forward with a wild look to block \refer{Jen}'s path.}
				Over my dead body!
\speak{Jen}		\direct{Mildly disgusted.}
				So that's where you keep your porn, then?
\speak{Roy}		\direct{\refer{Moss} rolls his eyes.}
				Yeah, you've got us.
\attrib{The IT Crowd (not).}
\end{dialogue}
Yes, you can do without.
But you will probably be hacked.
And hacked again,
most likely;
just for sport.
Your secret keys on your harddrive are a prize,
even if publishing them does not directly harm your security.
You have a lot of explaining to do when this happens for the second time.
The third time\ldots, oh, don't bother.
\par
The HSM is used for generating good quality random numbers.
Using some other pseudo-random generator will probably yield numbers less random,
which may be an attack vector.
\par
An HSM gives you physical and logical security,
at the cost of personnel.
The hardware costs are minimal compared to the people involved in securely handling your sensitive Secret keys.
But do it right,
and you are safe,
even if you are hacked.
\subsubsection{Eliminating Secret keys}
Instead of having different keys for website and user,
you may opt to store the user key in the website's user database.
Doing so will yield usable login data when the accounts of the website will be stolen.
\par
This is not recommended,
but not necessarily wrong.
Still,
no replay of logins can occur,
as there is still the use of random numbers for this.
\par
A possible bonus for this large omission in security could be that you don't need to store any Secret keys anywhere,
except the keys belonging to the users.
See if your users will put up with this for very long.
\subsubsection{Skipping website authentication}
A website may be offering user authentication only.
When Secret keys have been dismissed as well,
this could be considered the most basic login scheme.
\par
In this case,
a user only sends its user id.
The website uses the corresponding user key to encrypt a random value.
All the user has to do is return the right hash value to log in.
\par
With the absence of the site verification,
the user loses the ability to verify that it is talking to the website that offered him its key originally.
It could be any other website,
possibly allowing him to login anyway and lure him to reveil sensitive or valuable information.
\par
Of course there are certificates that indicate the trustworthyness of a webserver,
but site authentication makes almost absolutely sure,
and you cannot annoyingly ``click this away'' like a warning for a bogus website certificate.
