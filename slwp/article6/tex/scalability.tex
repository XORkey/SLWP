\section{Scalability}
There are several degrees of scalability with this login scheme.

\subsection{Keyring}
Instead of a nice and round 100,
you can put any number of keys in a keyring file.
The value of 100 is just convenient;
all keys are numbered with a two-digit number,
which can easily be remembered.
\par
Any number will do,
from zero
(which leaves only the keyring identifier,
and allows you to login to zero websites)
up to any amount of keys you bother to carry around with you.
Having more keys than sites you want to login to is just a way to make things a bit harder for those that do not own the keyring to choose keys.
If you are not happy with this,
you can just add new keys as you acquire them,
just as with real keys.
(Nothing stops you from adding bogus brass and steel keys to the keyring that holds your carkeys.
But I don't believe it is common practice to add BMW and Ford keys to a keyring of a Toyota owner.)

\subsection{Old Secret keys}
At least one old Secret key is required,
if Secret keys are to be replaced every now and then.
An HSM can store many keys,
so there is no real practical limit there.
More likely,
the amount of effort needed to safely work with keys will cap the number to a small integer.

\subsection{Encryption algorithms}
In this article,
the basis for generating the keypairs for the site and the user is \AES.
All site keys and user keys are related using this algorithm,
so it has to be secure.
The encryption algorithm has to be a block cypher,
using a secret key.
A hashing function does not do the trick,
as this would directly reveil user keys once you have the site keys.
But any other block cypher may be employed for this task.
\par
Hashes are computed using \SHA\ in this article.
The hash values that are exchanged between website and user are incomplete by design,
and just contain some of the least significant bits of the hash.
The key length of the site and user keys
should not exceed the lenght of the hash value,
nor have equal length,
as that would weaken the login algorithm.
Using 128-bit keys, a hash function like \MDV\ is not recommended, as this produces a 128-bit hash.
