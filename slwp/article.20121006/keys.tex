\section{Keys}
The keys used in this scheme are all 128 bits in size and as random as possible.
Some keys are kept very secret; some are put on the user's keyring.

\subsection{Secret keys}
\label{sec:secret_keys}
The keys used to encrypt other keys%
---called Secret keys in this article---%
need to be secret, inaccessible and unobtainable.
Therefore, they are kept in a Hardware Security Module (HSM).
The Secret keys are used by the HSM to perform \E{AES} encryptions of site keys.
\par
There are multiple Secret keys stored in the HSM.
They are considered to be stored in array $S$, in this article.
With this array, two values are defined.
First $m$, being the number of stored Secret keys.
Second, the constant \texttt{MAX\_KEYS}, which is the maximum number of keys that will be kept.
The value of $m$ starts at 0 and will reach \texttt{MAX\_KEYS} eventually and grow no more.
The value \texttt{MAX\_KEYS} has a minimum of 2 (a new key and at least 1 old key).
All values in the array $S$ are shifted when a new key is stored, which will always be inserted at $S[0]$.
This way, with at least one key loaded ($m>0$), it always holds that $S[0]$ is the newest key,
and $S[min(m,\mathtt{MAX\_KEYS)} - 1]$ is the oldest key.
With $m>1$, $S[1]$ is the jongest old key.
\par
Although declared a constant, the value of \texttt{MAX\_KEYS} may change over time.
When the login policy regarding the use of old keys is changed, more or fewer old keys will be stored.
For this, \texttt{MAX\_KEYS} need to be changed.

\subsection{Site keys}
For each user, there exists a site key and a derived user key (see section~\ref{sec:userkeys}).
The site keys are stored in a table of a database,
where the \E{SHA} hash of the userid will act as the primary key for that table.
Since the user only sends the \E{SHA} hash of its userid, no actual userid's will be stored.
\par
Site keys are generated once by the random generator of the HSM, and need never be replaced.
\par
Each time a user tries to login,
the database is queried with the \E{SHA} hash of the userid,
and the site key for that user is returned for further processing.

\subsection{User keys}
\label{sec:userkeys}
A user key is computed by encrypting the site key for the user with the joungest (or current) Secret key,
using the \E{AES} algorithm.
\par
The keys users get from the website are stored on a keyring (see section \ref{sec:keyring}).
These keys are automatically changed for new keys by the website when the Secret key is changed.

\subsection{Key lifetime and old keys}
For enhanced security, Secret keys need to be replaced at regular intervals.
%\footnote{The process of changing keys in an HSM is beyond the scope of this article.
%It includes more than just entering the new value and should be done with great procedural care.
%Ideally, the website's organization is certified for this process under ISO27001.}
This change of keys is initiated by the website,
without the possibility to make individual arrangements with any of its users.
\par
After changing the Secret key, it is no longer possible to login with any user key.
Therefore, an array of Secret keys need to be kept, allowing users to login using an old(er) Secret key.
At least one old Secret key needs to be stored; otherwise, the Secret key cannot be changed without rendering all user keys permanently useless.
\par
Upon successful login with an old key, the user will be provided with a new user key,
with which it can login using the newest Secret key, from that moment on.
Changing of user keys is seamless and the user might, just as well, be kept unaware of this.
\par
When the same key is on multiple keyrings on different devices
(a perfectly normal situation),
some keyring may still have a key that was valid, but renewed on another device.
This is the reason that more than one old key need to be kept.
It is up to the website's administrator with what frequency Secret keys are changed, and how many old keys are kept.
If, for instance, the Secret key is changed every month, and 11 old keys are kept, users can still login using a key that is a year old.
If the user key is older than that, the site is not able to verify the user's key any longer.

\subsection{Compromised keys}
There are several scenarios where keys can become compromised.
\begin{enumerate}
\item A hacker could get full and unrestricted access to the database holding all site keys and userid hashes.
\par
The hacker can then start cracking the \E{AES} encryption by capturing all
login traffic to the site for a long period of time.
Since there is just a single block of 128 bits used for each login,
which consists purely of random data,
there is not much to go on.
\item A hacker could obtain the keyring from a user.
\par
The hacker would also need to know with which key(s) the keyring is encrypted.
Should the hacker know that too, it would still need to know which key is used for what.
If the hacker should know a key, then he or she can login to that site.
He or she has obtained a single login.
\par
The site name is often entered using the keyboard.
If possible,
selecting a key from the keyring should be done using a graphical interface,
so a simple key-logging Trojan would not be usable to gain knowledge about which key is used for which site.
\item A hacker could steal the backup copies of the Secret keys from a vault.
\par
With just the Secret keys compromised, no harm is done.
Only when the database is also available, the hacker has full control.
This full control is just for that site, and no other site is in any way compromised as well.
\item A hacker could break the SSL/TLS encryption of the HTTPS connection and
view all traffic between user and website in plain text.
\par
The hacker would learn nothing, except which $\E{SHA}$ hash code the user uses for that site.
Using the hash to login requires a special replacement login program, to insert the hash directly in the data stream.
It is hard to find a login that produces the same hash.
\par
All other login data is random and encrypted at that.
\end{enumerate}
