\Section{Introduction}
Passwords should be kept in memory%
---human memory, that is---%
at all times.
This article is about passwords for websites; the passwords many people use on a day to day basis.
Passwords are an archaic type of security measure
(\cite{Honan2012}),
compared to the scheme proposed here.

\Subsection{On human behavior}
The consensus amongst security experts is that passwords should not be written down.
Ever.
And passwords should be unique for each and every website.
Oh, and---I almost forgot---you have to change them as well.
Each month or so will do nicely. 
\par
Yeah, right!
I cannot remember all different passwords I am forced to use, although my memory is quite good.
I \emph{have} to write them down,
	otherwise,
		I am lost.
(Fortunately,
	I have some support for this \cite{Schneier:2005}.)
Therefore,
	I resort to a little black booklet,
		with all my account information.
I don't remember passwords any more,
	I remember where my booklet is.
And I confess that I am compelled to reuse passwords,
	and to keep the ones I am not forced to change,
		so I can actually remember some of them.
So I believe nothing has fundamentally changed in more than 15 years\ldots\cite{Adams:1999:UE:322796.322806}
\par
%Letting humans use passwords is a bad idea because of what humans do:
%forget, steal, conceil, ignore.
%Since their first use they were troublesome,
%which is nicely depicted in the movie WarGames,
%released way back in 1983.
%The `pencil' password for the school's online system was nicely written down on a piece of paper.
%Notice, by the way, that the password was changed regularly!
%\par
I have given up on inventing a scheme for passwords that differ from each other,
can be changed individually at different times, 
and are easy to remember, as not to be forced to write them down.
I believe no such scheme exists, because websites have different requirements regarding passwords.
To see what I mean, watch \cite{youtube:tobyturner}.
\par
For the user, the bad thing with passwords is that you have to keep track of it all.
Remembering difficult passwords is cumbersome for most, and impossible for some.
Tracking things infallibly, and remembering different passwords for each and every site is not something people excel at.

\Subsection{A new login approach}
Using a key,
with up to 128 random bits,
to gain access to a website is far more secure than letting people decide which password they would like to use to do so.
\par
This proposed new way of logging in has several advantages over the current practice of websites requiring passwords.
Instead of having to remember dozens of passwords for numerous sites,
you only need to remember a key number for that site, in the range of 1 to 99.
This key number stays the same for that website at all times, so you \emph{can} remember it.
